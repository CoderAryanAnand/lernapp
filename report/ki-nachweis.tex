\documentclass[12pt, a4paper]{article}
\usepackage[utf8]{inputenc}
\usepackage[T1]{fontenc}
\usepackage[nswissgerman]{babel}
\usepackage{geometry}
\usepackage{graphicx}
\geometry{a4paper, margin=1in}

\title{Quellenverzeichnis Künstliche Intelligenz (KI)}
\author{}
\date{\today}

\begin{document}

\makeatletter
\begin{titlepage}
    \centering
    \vspace*{1cm}
       { \includegraphics[width=6cm]{img/kanti-baden.png}}\\[1cm]

    {\LARGE \textbf{Kanti Koala}}\\
    {\textbf{Die Lern- und Studienhilfsapp für Schüler:innen der Kantonsschule Baden}}\\[1cm]

    {Maturitätsarbeit, Kantonsschule Baden}\\
    {Künstliche Intelligenz (KI) Nachweis}\\[1cm]
    
    \textbf{Erstbetreuer: }{Michael Schneider}\\
    \textbf{Zweitbewerterin: }{Julia Smits}\\[1cm]
    
    \textbf{Geschrieben von: }{Aryan Anand (G22b), Simon Haddon (G22b)}\\[1cm]
    \date{\large Datum: 11. November 2025}
    {\@date\\}
\end{titlepage}
\makeatother


\section*{KI Nachweis}
\addcontentsline{toc}{section}{KI Nachweis}

\textbf{Hinweis}: Die meisten Anfragen wurden in Copilot (Visual Studio Code) gestellt, also mussten wir den Code nicht in den Prompt kopieren. Die Anfragen, bei welchen wir den Code kopieren mussten, sind aber wie normal gekennzeichnet mit einem [...].

\begin{itemize}

    \item \textbf{ChatGPT-4-Turbo}: «erstelle fragen für eine expertin in den jeweiligen fächern: Themen, in welchen wir Auskunft brauchen: - Infos über Lernverhalten (Lerntimer, Lernmethoden) - Stressmanagement \& Pausen - Effektive/Schlaue Zeitplanung - Tipps für Umfrage entwerfen Interview Fragen: (Lernverhalten) - - (Stressmanagement \& Pausen) (Effektive / Schlaue Zeitplanung) (Tipps für Umfrage entwerfen)», 12.03.2025. Antwort als Inspiration.
    
    \item \textbf{ChatGPT-4-Turbo}: «Create a basic Flask authentication system with a home page, login, register, and logout functionality. The system should use SQLite for the database and SQLAlchemy as the ORM. Passwords should be securely hashed using bcrypt. The home page should display 'Login' and 'Register' buttons if the user is not logged in, and a 'Welcome, [username]' message with a 'Logout' button if they are logged in. Ensure the Flask app is well-documented with comments and docstrings. The database tables should be created automatically at startup, using a method compatible with Flask 2.3+. Use session-based authentication for managing user logins. The folder structure should include templates for home, login, and register pages, using simple HTML forms.», 19.03.2025. Antwort als Basis.
    
    \item \textbf{ChatGPT-4o}: «create the css for the app», 28.03.2025. Antwort ganz übernommen.
    
    \item \textbf{ChatGPT-4o}: «add following functions to the agenda: if you click on an event, it will create a popup which has a form, where you can edit the information and color of the event, or delete it. it needs to be connected to the events database. if you click on the agenda, where there is no event, there should be a similar popup where you can create an event at that time.», 02.04.2025. Antwort als Basis.
    
    \item \textbf{ChatGPT-4o}: «comment the entire main.py, and agenda.html, so that everything is explained», 02.04.2025. Antwort ganz übernommen.
    
    \item \textbf{ChatGPT-4o}: «make sure the date matches the event being updated. only the time is changing. if the day is also changed, then that should also be changed for all events (eg. monday to tuesday)», 06.04.2025. Antwort ganz übernommen.
    
    \item \textbf{ChatGPT-4o}: «when i change the date of a recurring event, and want it to change for all recurring events, it: 1. doesnt change the start time of the selected event. 2. doesnt change the time or day of recurring events», 06.04.2025. Antwort ganz übernommen.
    
    \item \textbf{ChatGPT-4o}: «create a button above the agenda, which allows the user to import events in .ics format», 10.04.2025. Antwort als Basis.
    
    \item \textbf{ChatGPT-4o}: «Create a daily tips box on the home page», 09.05.2025. Antwort ganz übernommen.
    
    \item \textbf{ChatGPT-4o}: «how would i store an sql file online/in the cloud for my webapp which im programming in flask», 14.05.2025. Antwort als Inspiration.
    
    \item \textbf{ChatGPT-4o}: «tell me concretely how i can implement the security considerations», 14.05.2025. Antwort als Basis.
    
    \item \textbf{ChatGPT-4o}: «every semester should be a drop down, so i can collapse semesters. make sure there is a button to add grades so i can actually add grades. additonally when adding a grade i need to be able to add its name, its weightage, and if it counts. if it doesnt count it should not affect the average. now the subjects should also be a dropdown. when collapsed i just see the average, when its open i see the names of the tests and their grades and their weightage», 20.08.2025. Antwort ganz übernommen.
    
    \item \textbf{ChatGPT-4o}: «everything on the page should be saved, so that the user can access it whenever they want.», 20.08.2025. Antwort als Basis.
    
    \item \textbf{ChatGPT-4o}: «create a pomodoro timer in html», 09.09.2025. Antwort ganz übernommen.
    
    \item \textbf{ChatGPT-4o}: «Comment the entire code, so its readable, and make it nicer basically (without changing the code)», 01.10.2025. Antwort ganz übernommen.
    
    \item \textbf{ChatGPT-4o}: «Make timezone offset work for edit event too», 01.10.2025. Antwort ganz übernommen.
    
    \item \textbf{ChatGPT-4o}: «Create a login\_required decorator», 01.10.2025. Antwort als Basis.
    
    \item \textbf{Kimi.com (Model K2)}: «create a psedo code algorithm, which is described below: Input: Agenda, with events. Events have priority 1-4, where 1-3 describe how important the test is (1: most important, 3: least important) Saturday learning: Boolean Sunday learning:Boolean Preferred learning time: time of day, which the user would prefer to learn on. Algorithm should try to enter learning time slots for the exam, it should start maximum 2 weeks before. if the agenda is too filled, and that is not possible at all, it can go up to 3 weeks, but not more than that. sunday and Saturday learning only if the Boolean is true. if the exam is more important, you need to learn more for it. Output: Events, which have priority 4, where the user can learn.», 01.10.2025. Antwort als Basis.
    
    \item \textbf{ChatGPT-4o}: «Create a test case for the algorithm», 03.10.2025. Antwort ganz übernommen.
    
    \item \textbf{Gemini (Model 2.5 Flash)}: «how would i display the summary message on a popup?», 04.10.2025. Antwort als Inspiration.
    
    \item \textbf{Gemini (Model 2.5 Flash)}: «the learning time algorithm doesnt work with the test case under populate events, what is the problem?», 04.10.2025. Antwort als Inspiration.
    
    % \item \textbf{Gemini (Model 2.5 Flash)}: «Correct and improve this part of the report», 04.10.2025. Antwort ganz übernommen.
    
    % \item \textbf{Gemini (Model 2.5 Flash)}: «verbessere folgenden text», 05.10.2025. Antwort ganz übernommen.
    
    % \item \textbf{Gemini (Model 2.5 Flash)}: «verbessere folgenden text», 06.10.2025. Antwort ganz übernommen.
    
    \item \textbf{Gemini (Model 2.5 Flash)}: «Comment this entire code for readability, and so that someone who has never seen this code can understand it», 06.10.2025. Antwort ganz übernommen.
    
    % \item \textbf{Gemini (Model 2.5 Flash)}: «with the attached documents, format these ai ``sources'' correctly in latex:», 07.10.2025. Antwort ganz übernommen.
    
    \item \textbf{ChatGPT-4o}: «make the categories collapsable», 09.10.2025. Antwort ganz übernommen.

    \item \textbf{Gemini (Model 2.5 Flash)}: «what are some server options I can use for my flask application?», 10.10.2025. Antwort als Inspiration.

    \item \textbf{Gemini (Model 2.5 Pro)}: «Create csrf protection for this app without flask-wtf», 21.10.2025. Antwort als Basis.

    \item \textbf{Gemini (Model 2.5 Pro)}: «Korrigiere Fehler in den Test-Cases, und verbessere die Sprache», 23.10.2025. Antwort ganz übernommen.

    % \item \textbf{Gemini (Model 2.5 Pro)}: «Verbessere, korrigiere, und schreib folgenden Text besser in der main.tex Datei», 31.10.2025. Antwort ganz übernommen.
    
    % \item \textbf{ChatGPT-5}: «Verbessere, und schreibe diesen Text um, damit er besser mit den Angaben dieses Dokumentes passt.», 02.11.2025. Antwort ganz übernommen.
    
    \item \textbf{Claude (Model Sonnet 4.5)}: «Can you create a button to export .ics Button», 07.11.2025. Antwort ganz übernommen.

    \item \textbf{Gemini (Model 2.5 Pro)}: «Can you create a responsive design with dark mode with tailwind for all pages and make it german [...]», 07.11.2025. Antwort als Basis.

    \item \textbf{Gemini (Model 2.5 Pro)}: «Can you create a navbar which is available for all pages [...]», 07.11.2025. Antwort ganz übernommen.

    \item \textbf{Gemini (Model 2.5 Pro)}: «How do I make dark mode turn on based on system settings? I'm using tailwind», 07.11.2025. Antwort als Basis.

    \item \textbf{Claude (Model Sonnet 4.5)}: «create the content for this page, based on main.tex», 08.11.2025. Antwort als Basis.

    \item \textbf{Claude (Model Sonnet 4.5)}: «design this page, based on the other designs», 08.11.2025. Antwort ganz übernommen.

    \item \textbf{Claude (Model Sonnet 4.5)}: «redesign the create and update», 08.11.2025. Antwort ganz übernommen.
    
    \item \textbf{Gemini (Model 2.5 Pro)}: «Create a scroll to top button», 08.11.2025. Antwort als Basis.
    
    \item \textbf{Gemini (Model 2.5 Pro)}: «make the darkmodejs respect the setting dark mode», 08.11.2025. Antwort ganz übernommen.
    
    \item \textbf{Gemini (Model 2.5 Pro)}: «Create a instruction/tutorial for the app», 09.11.2025. Antwort als Basis.
    
    \item \textbf{Gemini (Model 2.5 Pro)}: «Redesign the home page including the the new info», 09.11.2025. Antwort als Basis.
    
    \item \textbf{Gemini (Model 2.5 Pro)}: «Redesign the page to make the tutorial nice», 09.11.2025. Antwort als Basis.
    
    \item \textbf{Claude (Model Sonnet 4.5)}: «designe eine todo-listen seite», 10.11.2025. Antwort als Basis.
    
    \item \textbf{Claude (Model Sonnet 4.5)}: «create js functions similar to agenda.js and also animations for items», 10.11.2025. Antwort als Basis.
    
    \item \textbf{ChatGPT (Model 4.1)}: «ensure that the user must enter a grade in the rang 1-6», 10.11.2025. Antwort ganz übernommen.
    
    \item \textbf{ChatGPT (Model 4.1)}: «design a nice semester rename popup», 10.11.2025. Antwort als Basis.
    
    \item \textbf{ChatGPT (Model 4.1)}: «add a enter and escape keybind», 10.11.2025. Antwort ganz übernommen.
    
    \item \textbf{ChatGPT (Model 4.1)}: «how do i make rows draggable», 10.11.2025. Antwort als Basis.
    
    \item \textbf{ChatGPT (Model 4.1)}: «create a delete confirmation for events», 10.11.2025. Antwort als Basis.
    
    \item \textbf{ChatGPT (Model 4.1)}: «design the datenschutz page», 10.11.2025. Antwort als Basis.
    
    \item \textbf{ChatGPT (Model 4.1)}: «comment the entire code so that it fits best practices», 11.11.2025. Antwort ganz übernommen.
    
    \item \textbf{ChatGPT (Model 4.1)}: «comment the entire code so that it fits best practices using html comments», 11.11.2025. Antwort ganz übernommen.
    
    \item \textbf{ChatGPT (Model 4.1)}: «comment the entire code so that it fits best practices using js comments», 11.11.2025. Antwort ganz übernommen.
\end{itemize}

\end{document}