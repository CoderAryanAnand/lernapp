\documentclass[12pt,a4paper]{report}

% Questions to ask @Herr Schneider
% 1 -



% Packages
\usepackage{graphicx}
\usepackage{amsmath}
\usepackage{hyperref}
\usepackage[T1]{fontenc}        % proper glyphs for äöüß, hyphenation
\usepackage[utf8]{inputenc}     % source file is UTF-8
\usepackage{lmodern}            % Latin Modern fonts
\usepackage[ngerman]{babel}     % modern German hyphenation
\usepackage[autostyle]{csquotes}
\usepackage{setspace}
\usepackage{titlesec}
\usepackage{tabularx}
\usepackage{booktabs}
\usepackage{array} % for 'm' column type
\usepackage{longtable}
\usepackage{anyfontsize}
\usepackage{xurl}
\usepackage{float}
\usepackage[
backend=biber,
style=apa,
language=ngerman
]{biblatex}
\DeclareLanguageMapping{ngerman}{ngerman-apa}
\DefineBibliographyStrings{ngerman}{
  nodate = {o\adddot\addspace D\adddot}
}
\usepackage{microtype}          % better line breaking / protrusion
\usepackage[htt]{hyphenat}      % allow hyphenation in \texttt / typewriter

% Prevent footnotes from breaking across pages
\interfootnotelinepenalty=10000

% Hyphenation / line-breaking tuning
\pretolerance=50
\tolerance=2000
\emergencystretch=25pt
\hyphenpenalty=100
\exhyphenpenalty=50
\doublehyphendemerits=5000
\finalhyphendemerits=1000
% Manual hyphenation exceptions (add as needed)
\hyphenation{
  Lern-zeit-al-go-rith-mus
  Lern-zeit-al-go-rith-men
  Nutz-er-da-ten
  Authen-ti-fi-zie-rung
  Prioritäts-ein-stel-lun-gen
  Pro-to-typ
  Daten-struk-tur
  Daten-struk-tu-ren
  Web-An-wen-dung
  An-wen-dungs-ar-chi-tek-tur
}
% \DeclareLanguageMapping{german}{german}
\addbibresource{references.bib}
\usepackage{etoolbox}
\makeatletter
\patchcmd{\chapter}{\if@openright\cleardoublepage\else\clearpage\fi}{}{}{}
\makeatother
\makeatletter
\renewcommand{\@makechapterhead}[1]{%
\vspace*{50 pt}%
{\setlength{\parindent}{0pt} \raggedright \normalfont
\bfseries\Huge
\ifnum \value{secnumdepth}>1 
   \if@mainmatter\thechapter.\ \fi%
\fi
#1\par\nobreak\vspace{40 pt}}}
\makeatother

% Create a command to add a footnote for AI assistance, you can add dates or versions if needed
\newcommand{\aifootnote}[1]{\footnote{ChatGPT (Version GPT-5): \enquote{Überarbeite den folgenden Text, damit er sprachlich und stilistisch den Standards einer wissenschaftlichen Maturitätsarbeit entspricht. Achte auf korrekte Grammatik, präzisen Ausdruck, logische Argumentation und sachlichen Stil. Behalte den ursprünglichen Sinn und Stil des Textes bei, aber formuliere ihn wissenschaftlicher und grammatikalisch korrekt. [...]. }, #1. Antwort ganz übernommen.}}
\newcommand{\aifootnotemark}{\footnotemark}
\newcommand{\aifootnotetext}[1]{\footnotetext{ChatGPT (Version GPT-5): \enquote{Überarbeite den folgenden Text, damit er sprachlich und stilistisch den Standards einer wissenschaftlichen Maturitätsarbeit entspricht. Achte auf korrekte Grammatik, präzisen Ausdruck, logische Argumentation und sachlichen Stil. Behalte den ursprünglichen Sinn und Stil des Textes bei, aber formuliere ihn wissenschaftlicher und grammatikalisch korrekt. [...]. }, #1. Antwort ganz übernommen.}}
\newcommand{\aifootnotebasis}[1]{\footnotetext{ChatGPT (Version GPT-5): \enquote{Überarbeite den folgenden Text, damit er sprachlich und stilistisch den Standards einer wissenschaftlichen Maturitätsarbeit entspricht. Achte auf korrekte Grammatik, präzisen Ausdruck, logische Argumentation und sachlichen Stil. Behalte den ursprünglichen Sinn und Stil des Textes bei, aber formuliere ihn wissenschaftlicher und grammatikalisch korrekt. [...]. }, #1. Antwort als Basis.}}

% Fix for font issue
\DeclareRobustCommand{\ttfamily}{\fontencoding{T1}\fontfamily{lmtt}\selectfont}

% newline after paragraph
\newcommand{\myparagraph}[1]{\paragraph{#1}\mbox{}\\}

\onehalfspacing

% Begin Document
\begin{document}


% Title Page
\makeatletter
\begin{titlepage}
    \centering
    \vspace*{1cm}
       { \includegraphics[width=6cm]{img/kanti-baden.png}}
       \addtocounter{figure}{1}
       \addcontentsline{lof}{figure}{\protect\numberline{\thefigure}{Logo der Kantonsschule Baden. Quelle: Wikipedia}}
       \\[1cm]

    {\LARGE \textbf{Kanti Koala}}\\
    {\textbf{Die Lern- und Studienhilfsapp für Schüler:innen der Kantonsschule Baden}}\\[1cm]

    {Maturitätsarbeit, Kantonsschule Baden}\\
    {Schriftlicher Kommentar}\\[1cm]
    
    \textbf{Erstbetreuer: }{Michael Schneider}\\
    \textbf{Zweitbewerterin: }{Julia Smits}\\[1cm]
    
    \textbf{Geschrieben von: }{Aryan Anand (G22b), Simon Haddon (G22b)}\\[1cm]
    \date{\large Datum: 11. November 2025}
    {\@date\\}
\end{titlepage}
\makeatother

\chapter*{Abstract}
\addcontentsline{toc}{chapter}{Abstract}
% LAST SENTENCE OF THIS LINE TO BE EDITED.
Diese Maturitätsarbeit behandelt die Konzeption und prototypische Umsetzung einer webbasierten Applikation für Schüler:innen der Kantonsschule Baden mit dem Ziel, schulische Termine, Aufgaben und Lernzeiten zentral organisierbar zu machen und regelmässige, abgegrenzte Lernphasen zu unterstützen. Eine umfassende Wirkungskontrolle war nicht Teil der Zielsetzung; stattdessen wurde die Bedienbarkeit in einem Usability-Test geprüft.; die Ergebnisse bestätigten grundlegende Bedienbarkeit und identifizierten kleinere Optimierungspunkte (Navigationstiefe, Beschriftungen einzelner Formularfelder).

Zur Ableitung der Anforderungen wurden eine fokussierte Literatur- und Internetrecherche (Lernmethoden, Zeit- und Pausenorganisation, Stressaspekte), Interviews mit zwei PPP-Lehrpersonen sowie eine Umfrage unter Schüler:innen eingesetzt. Die Ergebnisse strukturierten die Funktionsprioritäten (Agenda, Lernblöcke, Noten, Timer, Tipps) und die Parametrisierung des Lernzeitalgorithmus.

Die Applikation wurde mit Python (Flask), einem relationalen Datenmodell und modularer Architektur implementiert. Kernfunktionen sind: eine Agenda mit manueller Ereigniserfassung, Import von \texttt{.ics}-Stundenplänen und algorithmischer Lernblock-Generierung nach Prioritäten; Notenverwaltung; ein konfigurierbarer Pomodoro-basiertes Lerntimer-Modul; kategorisierte tägliche Tipps, sowie auch allgemeine Lerntipps (Zeit-, Pausen-, Stressmanagement, Lernmethoden); Authentifizierung mit Basis-Schutzmassnahmen (u. a. Passwort Hashing). Der Lernzeitalgorithmus verteilt definierte Lernstunden rückwärts vom Prüfungstermin unter Berücksichtigung von Tageslimits, bevorzugten Zeiten und Konfliktvermeidung.

Die Arbeit zeigt die technische Machbarkeit eines integrierten, strukturierten Ansatzes zur Lern- und Organisationsunterstützung; eine weitergehende Wirkungsevaluation (über die durchgeführte Usability-Prüfung hinaus) steht aus und bildet zukünftige Arbeit. Der Prototyp bildet eine erweiterbare Grundlage für nachfolgende empirische Validierung. \aifootnote{06.11.2025}
\clearpage
% Table of Contents
\tableofcontents

% ------------------------------Einleitung------------------------------
\chapter[Einleitung -- unsere Vision]{Einleitung -- unsere Vision\texorpdfstring{\aifootnotemark}{}}
\aifootnotetext{04.11.2025}
Der Schulalltag an der Kantonsschule Baden ist durch eine hohe Arbeitsdichte geprägt. Neben zahlreichen Prüfungen, auf die vorzubereiten ist, fallen kontinuierlich Hausaufgaben an. In Kombination mit ausserschulischen Verpflichtungen entstehen Anforderungen an Planung und Selbstorganisation, die ohne strukturierende Hilfsmittel nur begrenzt überschaubar bleiben. 

\section{Motivation und Relevanz}
Aus unserer Erfahrung als Schüler der Kantonsschule Baden besteht ein wiederkehrendes Problem für Schüler:innen darin, Lernaufwand, Fristen und verfügbare Zeitblöcke koordiniert abzubilden. Fehlende Übersicht kann -- wie in unseren Interviews und der Umfrage als wiederkehrendes Problem identifiziert -- zu subjektivem Zeitdruck, Verdichtung einzelner Tage und Verschiebungen führen. Eine zentral organisierende Applikation, welche Lernzeiten, Termine und Noten gemeinsam verwaltet, ist daher potenziell relevant. In diesem Kontext wird unter Unterstützung des Selbstmanagements ausdrücklich Folgendes verstanden: (1) Reduktion des individuellen Planungsaufwands (Zeit zur manuellen Wochenplanung), (2) Erhöhung der Vorlauftransparenz (Sichtbarkeit bevorstehender Prüfungen in Tagen), (3) strukturierte Segmentierung von Lernblöcken (klar definierte Start-/Endzeiten). Die Plattformwahl (Web oder nativ) bleibt zum Untersuchungszeitpunkt offen; die Konzeption erfolgt plattformneutral.

\section{Name und Logo}
Der Name \enquote{KantiKoala} verbindet \enquote{Kanti} (Kurzform von Kantonsschule) mit dem Koala als sympathischem Maskottchen. Die Alliteration macht den Namen einprägsam; der Koala haben wir als Tier gerne, deswegen wurde er gewählt. Das Logo wurde bewusst schlicht gehalten, um freundlich und zugänglich zu wirken und gleichzeitig in der Kalender- und UI-Darstellung gut lesbar zu bleiben.

\begin{figure}
    \centering
    \includegraphics[width=0.4\textwidth]{img/KantiKoala-Logo-Var2.png}
    \caption[Logo der Applikation Kanti Koala. Eigene Darstellung, 22.05.2025.]{Logo der Applikation Kanti Koala}
    \label{fig:logo}
\end{figure}

\section{Fragestellung und Zielsetzung}
\textbf{Hauptfragestellung:}
\begin{quote}
Kann eine digitale Applikation so konzipiert und prototypisch umgesetzt werden, dass sie Schüler:innen eine übersichtliche Organisation von Terminen, Aufgaben und Lernzeiten ermöglicht und zugleich Funktionen bereitstellt, die regelmässige Lerngewohnheiten unterstützen?
\end{quote}
\textit{Begriffsklärung:} Unter „Organisation erleichtern“ verstehen wir eine bessere Übersicht über anstehende Termine/Aufgaben und eine Verringerung des manuellen Planungsaufwands. „Lerngewohnheiten unterstützen“ meint Funktionen, die regelmässige, zeitlich abgegrenzte Lernphasen fördern (z. B. Lernblöcke und Timer).\\

\noindent
\textit{Zielsetzung:} Entwicklung eines Prototyps, der die zur Messung dieser Indikatoren notwendigen funktionalen Strukturen (Agenda mit Prioritäten, Lernblock-Generierung, Notenerfassung, Timer) bereitstellt.

\section{Aufbau und Begründung des schriftlichen Kommentars}
Der Bericht ist entlang der Entwicklungslogik aufgebaut und führt von der Ausgangslage über die Umsetzung zur Einordnung:
\begin{enumerate}
  \item \textbf{Recherche}: TBA
  \item \textbf{Methodik - Programmierung}: Systemarchitektur, Datenstrukturen, Darstellung der Features, Sicherheitsaspekte und schliesslich noch Testing.
  \item \textbf{Schlussfolgerung und Ausblick}: Bilanz zur Fragestellung, gewonnene Erkenntnisse und offene Arbeitsschritte.
\end{enumerate}
Die Gliederung macht nachvollziehbar, wie aus Problemstellung und Zielsetzung konkrete Komponenten abgeleitet, implementiert und dokumentiert wurden; ein Usability-Test der Bedienbarkeit und Verständlichkeit ist auch Bestandteil der Arbeit und wurde durchgeführt.

% ------------------------------Recherche------------------------------
\chapter{Recherche}
\section{Einführung}
\subsection{Warum brauchen wir eine Recherche?}
Da wir eine Web-App erstellen wollen, welche gut an die Bedürfnisse von Schüler:innen angepasst ist, durften wir uns nicht nur auf unsere eigenen Erfahrungen als Schüler verlassen, sondern mussten auch ein gewisses Mass an Recherche erledigen, damit wir wichtige Entscheidungen sinnvoll begründen konnten. 
Zu diesem Ende haben wir uns entschieden, uns tiefgründig mit unserem Zielpublikum - Schüler:innen der Kantonsschule Baden - auseinanderzusetzen, indem wir Interviews mit PPP-Lehrpersonen führten und eine Umfrage für Schüler:innen gestalteten.

Die Recherche stellt hier nicht das Kernstück unserer Arbeit dar, sondern ist ein unterstützender, aber dennoch sehr wichtiger, Bestandteil für die Entwicklung der Web-Applikation, da sie uns hilft, uns in unsere Zielgrupe zu versetzen und ihre Bedürfnisse
Wir setzten einen starken Wert auf begründete Entscheidungen und strebten eine hohe Qualität an.

\subsection{Was recherchieren wir?}
Nun ging es zuerst einmal darum, herauszufinden was wir überhaupt recherchieren wollten. 
Da unsere Web-Applikation das Lernen fördern soll, stand der Kernpunkt, nämlich das Lernen, schon von Anfang an klar.

Um genaue Recherche-Themen auszusuchen, stellten wir einige W-Fragen zum Lernen:
\begin{itemize}
    \item Wie oder warum lernt man gut?
    \item Wann lernt man gut oder nicht gut?
    \item Was hindert das Lernen?
    \item Was verursacht Stress beim Lernen?
\end{itemize}

Ebenso versuchten wir zu bedenken, was für eine Rolle unsere geplanten Features für die Web-Applikation spielen und was für Themenbereiche für sie wichtig sein könnten.
Beispielsweise ist die Recherche zur Zeitplanung wichtig, damit wir unsere Agenda und den Lernzeitalgorithmus sinnvoll gestalten und an den Bedürfnissen von Schüler:innen anpassen können, Informationen zu Stress beim Lernen oder vor Prüfungen und die Vorbeugung davon können uns helfen, gute Lerntipps zu gestalten.
Somit entstanden unsere vier Hauptbereiche, welche die obigen drei Fragen beinhalten: Lerntechniken, das Pausen- und Zeitmanagement und das Stressmanagement.

Auch bestand die Möglichkeit, dass aus der Recherche zu diesen Themenblöcken weitere Features entstehen könnten.

\subsection{Wie führen wir die Recherche durch}

Uns war klar, dass wir, da wir uns tief in unser Zielpublikum versetzen, uns hauptsächlich auf die Meinungen von Personen in und im Umfeld unserer Zielgruppe.
Somit standen persönlichere Methoden, wie Interviews und Umfragen, schon früh in Erwägung.

Wir entschieden uns also nach Rücksprache mit unserem Erstbewerter, die Recherche in drei wesentliche Teile zu teilen: 
\begin{itemize}
    \item Eine begrenzte Literaturstudie, welche uns in das Thema einführen und ein wenig vertrauter mit der Materie machen soll.
    \item Darauf aufbauend führen wir Interviews mit Expert:innen, um diese Materie konkret zu vertiefen und, wenn möglich, auf das Zielpublikum zu beziehen.
    \item Durch Feedback aus den vorherigen Schritten bereiten wir eine Umfrage für unser Zielpublikum vor, welches uns Daten aus der Sicht der Schüler:innen liefern soll.
\end{itemize}

Somit können wir anhand dieser Strategie das Wissen, welche wir in der Literaturstudie auffinden, in den Interviews vertiefen, mit unserem gewünschten Umfeld vergleichen und dann gezielt in der Umfrage mit der Praxis vergleichen.
Dies sollte uns erlauben, einiges an nützlichen Informationen für unsere Web-Applikation zu erhalten und daraus wichtige Entscheidungen zu treffen.

Das Ziel der Interviews und der Umfrage bestand auch wesentlich daraus, konkrete Empfehlungen für unsere Web-Applikation zu erhalten, welche dann auch als \enquote{Must}-Anforderungen, also Features, welche zwingend empfehlenswert für die Implementation sind, und \enquote{Kann}-Anforderungen, welche definitiv nützlich aber nicht dringend wären.

\section{Literaturstudie}
\subsection{Ziel}
Da die Literaturstudie als Unterstüztung für die anderen, für uns weitaus wichtigeren Elemente unserer Recherche gedacht war, lag nicht besonders viel Fokus darauf.
Ziel war, hier ein kleines Stückchen an grundlegendem Wissen zu erreichen ohne dass wir komplett in der Materie verloren gehen und die Übersicht über was tatsächlich für unsere Arbeit nötig ist verlieren.

\subsection{Vorgehensweise}
Die Literaturstudie ist hauptsächlich aus zwei Teilen aufgebaut: Eine breitere Internet-Recherche um ein gewisses Basiswissen zu erreichen, und einen tieferen Einblick in zwei Sachtexte zum Lernen.
Dieses kombinierte Basiswissen fliesst dann direkt in die Vorbereitung für die Interviews hinein.

\subsection{Internet-Recherche}

Wie erwähnt, ging es hier primär um den Aufbau eines Grundsatzes an Vorwissen, welches wir an unserer bisherigen Erfahrung als Kantischüler anhängen können.
Somit fokussierten wir uns nicht darauf, möglichst breite und diverse Quellen einzuholen, sondern darauf, dass wir dieses Vorwissen einigermassen effizient aufbauen können. 
Daraus merkten wir, dass nicht viel nötig war, um dies zu erlangen.

\subsubsection{Lerntechniken und -methoden}
Eines der ersten interessanten Einblicke welche wir fanden, war der Unterschied zwischen den Fachbegriffen \enquote{Lernmethode} und \enquote{Lerntechnik}.

Nämlich besteht da der wesentliche Unterschied darin, dass Lerntechniken einzelne spezifische Schritte im Lernprozess sind, während eine Lernmethode eine Kollektion von Lerntechniken darstellt und die allgemeine Lernstrategie beschreibt.\parencite{Lerntechnik_1}
Dabei wurden wir auf ein paar wenige Lernmethoden und -techniken aufmerksam, welche angeblich das Lernen vereinfachern sollten, wie beispielsweise die Lernmethoden SQ3R\parencite{SQ3R} oder KWL\parencite{SQ3R}, während eine Lerntechnik beispielsweise das Erschaffen von Verknüpfung zu bestehendem Wissen darstellt. 
Diese wurden hauptsächlich von \textcite{SQ3R} übernommen.

\subsubsection{Zeit- \& Pausenmanagement}
Dieses Thema war schon von Anfang an wichtig für uns, vor allem wegen unseren Agenda- und Lerntimer-features, also haben wir hauptsächlich im Bereich des Lerntimers, auch bekannt als der \enquote{Pomodoro-Timer}, und die Zeiteinplanung recherchiert.

Zur erfolgreichen Zeiteinteilung gehört für uns auch die Fähigkeit, sinnvoll Pausen zu machen, weswegen wir auch Informationen zu wann und wie man Pausen machen soll recherchierten.
Dies war für uns vor allem wichtig, da wir in unserem Umfeld auch dies immer wieder als Problem beobachtet haben.

\subsubsection{Umfragedesign}
Da wir später eine Umfrage für die Schüler:innen der Kantonsschule Baden erstellen wollten, war es für uns wichtig, dass wir auch das Umfragedesign berücksichtigen.
Der Hauptteil der Informationen dafür kam nicht aus dem Internet, sondern aus einem Lehrmittel\parencite{umfrageDesign}, welches wir netterweise von den PPP-Lehrpersonen Frau Suter und Herr Schmocker erhalten hatten.
Da war hauptsächlich das Kapitel \texttt{\enquote{Sozialwissenschaftliche Methoden" (S. 191 - 218)}}
\textcite{umfrageDesign} hatte uns hier schon sehr weitergeholfen, da es uns auch in das Konzept der Beibehaltung der Befragten eingeführt hatte. 
Eine Umfrage darf somit nicht zu komplex sein, da sonst die Befragten zu schnell das Interesse verlieren und die Umfrage nicht abschliessen.

Auch recherchierten wir ein wenig zur Farbpsychologie für Umfragen, welche begründet was für Farben man verwenden sollte, um eine ansprechbare Umfrage zu gestalten. Hierbei bezogen wir uns hauptsächlich auf \textcite{ColorPsychology} und wählten schlussendlich ein blaues Farbschema für unsere Umfrage aufgrund dessen angeblicher fokussierender Wirkung.

\subsection{Literatur}
Für die Literaturstudie liehen wir die folgenden zwei Bücher aus, welche uns einen differenzierten Standpunkt geben sollten:
\begin{itemize}{}
    \item \texttt{Effektiver Lernen für Dummies (2. Auflage) \parencite{Book1} von Dr. Birgit Ebbert}
    \item \texttt{Lernpsychologie (6. Auflage) \parencite{Book2} von Walter Edelmann}
    
\end{itemize}
Diese zwei Bücher sollten uns einen guten Überblick über das Lernen aus zwei verschiedenen Perspektiven - der direkten Anwendung mit \enquote{Effektiver Lernen für Dummies} und der wissenschaftlichen Perspektive mit \enquote{Lernpsychologie} - geben. 
Dabei führten wir laufend Notizen und integrierten diese in unser Interviewfragendossier \& unsere Umfrage, welche in \texttt{Abschnitt 2.3 \& 2.4} näher beschrieben werden.

In den Büchern lernten wir viel über was einen guten Lernerfolg voraussetzt.
Dazu gehören unter anderem verschiedene Lerntechniken \& -strategien, wie man sich erfolgreich auf eine Prüfung vorbereitet und die Bedeutung von einem guten Lernumfeld, d.h. alle Faktoren um das éernen welche dies unterstützen.
Unter anderem beinhaltet dies, dass man einen guten Schlafrhythmus hat, genug sinnvolle Pausen macht, und auch psychologische Faktoren wie ein gutes Mindset. 
Auch ein paar wenige wissenschaftliche Begriffe waren für uns wichtig, darunter das sogenannte \enquote{Assoziationslernen}.\parencite{Book2}. 
Unter dem Assoziationslernen versteht \textcite{Book2} das Lernen durch der Schaffung von Verknüpfungen zu bisher gelerntem Wissen. 
Somit kann das Gehirn einfacher neues Wissen aufnehmen und verarbeiten.

Insgesamt konnten wir so unser Vorwissen zu unseren für uns wichtigen Themenbereichen noch um einiges ausbauen, damit wir uns noch besser auf die Interviews vorbereiten konnten.


\section{Interviews}

\subsection {Vorgehensweise}
Wir wussten, das ein wichtiger Aspekt unserer Recherche Interviews mit Expert:innen sein würden, da sie uns vermutlich am Besten weiterhelfen könnten, da sie sich gut mit dem Thema auskennen und viel persönliche Erfahrungen mitbringen.
Somit können sie auch direkt auf unsere Fragen eingehen und uns auch für die Umfrage persönlich Feedback geben. 

Zu diesem Ende wählten wir zwei PPP-Lehrpersonen der Kantonsschule Baden, Frau Suter und Herr Schmocker, aus. Sie haben beide Erfahrung mit der Lernpsychologie und, dank ihrer Tätigkeit als Lehrpersonen, auch viel Kontakt mit Kantischüler:innen.
Deswegen stellten sie, unserer Meinung nach, gute Interviewpartner für uns dar. Wir hatten bereits durch unserem Erstbetreuer, Herr Schneider, zwei Theorie-Dokumente aus \textcite{umfrageDesign} erhalten, welche uns mit der Umfragetheorie und der Durchführung von Interviews helfen sollte.

\subsubsection {Themenwahl}
Wir wollten uns hauptsächlich auf unsere am Anfang der Recherche festgelegten vier Themenbereiche konzentrieren:
\begin{itemize}
    \item Lernmethoden \& -techniken
    \item Stressmanagement
    \item Pausenmanagement
    \item Zeitmanagement
\end{itemize}

Unser Ziel mit den Interviews war es damit, vieles an nützlichen Informationen zu diesen Themenbereichen und geplanten Features für unsere Web-Applikation zu lernen.
Darunter gehören auch zum Beispiel Tipps für unsere Daily-Tipps- und Lerntipps-Features, aber auch gute Ansätze für unsere Agenda, den Lernzeitalgorithmus und den Lerntimer, auch als Pomodoro-Timer bekannt. 

Bei all diesen Themen haben wir neben dem Vorwissen, welches wir durch die Literaturstudie erworben haben, auch schon einen persönlichen Bezug dank unserer bisherigen Schulkarriere, und können uns so auch auf unsere eigenen Erfahrungen und Unsicherheiten stützen.
Als Letztes haben wir dann auch nach Feedback für unsere Umfrage eingefügt, da wir professionelles Feedback dafür einholen wollten und das Interview dafür die beste Gelegenheit ist, vor allem da die PPP-Lehrpersonen uns schon das Theorie-Dokument zum Umfragedesign\parencite{umfrageDesign} zur Verfügung gestellt hatten.

\subsection {Interviewfragebogen \& Interviewfragen}
Um sowohl uns selbst als auch die Interviewpartner auf das Interview adequat vorzubereiten, erstellten wir einen Interviewfragebogen, in dem all unsere geplanten Fragen aufgelistet sind.
Die Fragen wurden nach den vier Themenblöcken geordnet und nummeriert, um eine klare Struktur zu erstellen, in der das Interview verlaufen soll.

Die jeweiligen Fragen haben wir gesammelt, indem wir uns einerseits überlegten, wo wir nach der Literaturstudie Unsicherheiten sahen oder generell mehr wissen wollte, andererseits wo ein genauer Bezug zu den Schüler:innen der Kantonsschule Baden oder die persönlichen Erfahrungen der Interviewpartner wichtig sein könnten. 
Ebenfalls benutzten wir am Anfang ChatGPT um ein paar Vorschläge zu generieren, jedoch sind alle Fragen selbstständig ausgedacht und formuliert worden. (Vergleich \textit{KI Nachweis})

Wir achteten uns immer darauf, dass die Fragen einen guten Bezug zu unserer Web-Applikation hatten und das Potenzial hatten, relevante Informationen zu liefern.

\subsubsection {Interviewfragen: Lernverhalten}
Als Erstes überlegten wir uns theoretische Fragen zum Lernverhalten allgemein, aufgeteilt in \texttt{Lernmethoden} \& \texttt{Lerntechniken}. 
Hier wurde wieder der Unterschied zwischen den beiden Begriffen wichtig, welcher von \textcite{SQ3R} erklärt wurde.

\myparagraph{Lernmethoden}
Zum Thema Lernmethoden überlegten wir uns zwei sehr spezifische Fragen zur Effektivität von Lernmethoden zum Lernen.
Mit diesen Fragen wollten wir nach spezifischen Lernmethoden nachforschen, wie beispielsweise SQ3R und KWL\parencite{SQ3R}, und die Meinung der Interviewpartner:innen dazu herausfinden. 
Dies da, wenn sich solche als sinnvoll herausstellen würden, diese eventuell in die Web-Applikation integriert oder, beispielsweise, speziell erklärt werden könnten.
Somit konnten wir auch die erwähnten Lernmethoden aus der Internet-Recherche hineinarbeiten.

\myparagraph{Lerntechniken}
Das Segment der Lerntechniken stellten mit fünf Fragen das umfangreichste Segment des Interviewfragebogens dar.

Hier ging es uns darum, anstatt spezifische Techniken, wie beispielsweise das Assoziationslernen\parencite{Book2}, genauer zu erforschen, was für eine individuelle Person am Besten funktionieren würde und ob es gute, allgemeine Lösungen gibt.
Jenachdem, wie unterschiedlich das Lernen für verschiedene Personen sein kann, hätte dies unseren Ansatz für die Lerntipps der Web-Applikation stark verändern können.
Das grundlegende Ziel war deswegen immernoch, herauszufinden, wie man dies in die Web-Applikation integrieren kann. 
Beispielsweise wäre es schlau, wenn nun ganz klar eine spezifische Technik empfohlen wird, diese Technik, ähnlich wie beispielsweise die Pomodoro-Technik für das Zeitmanagement, einzubauen.

Besonders interessiert waren wir am sogenannten \enquote{Mindset} und den persönlichen Tipps der Interviewpartner:innen, da diese uns womöglich einen guten Einblick in die Materie aufgrund ihrer eigenen Erfahrung als Lehrperson geben könnten.
Natürlich sind alle Fragen auch besonders relevant für unsere Lerntipps.

\subsubsection{Interviewfragen: Pausenmanagement}
Dieser Abschnitt ist vor allem für unseres geplantes Lerntimer-Feature und unsere Agenda wichtig, hat aber auch eine Relevanz für unsere Lerntipps. 
Somit stellten wir Fragen, welche die Meinung der Interviewpartner:innen zu Pausen, deren Dauer und wie man gute Pausen macht, erforschen.
Beispielsweise könnte 


\subsubsection{Interviewfragen: Stressmanagement}
Das Stressmanagement war ein für uns durch persönliche Erfahrungen bereits sehr vertrautes Problem und eines, welches wir so gut wie möglich in unsere Lerntipps einbauen wollten. 
Ob und wie andere Schüler:innen oft Stress empfinden, wollten wir mit der Umfrage herausfinden, weswegen es auch hier hauptsächlich um die Erfahrungen der Lehrpersonen mit ihren Schüler:innen und um ihre Empfehlungen, wie man Stress abbauen kann, geht.
Prüfungs- und Lernstress vorzubeugen sind für uns wichtige Aspekte, welche wir auch so gut wie möglich mithilfe unseres Lernzeitalgorithmus und den Lerntipps in die Web-Applikation einbauen wollten.


\subsubsection{Interviewfragen: Zeitmanagement}
Als letztes Segment mit konkreten Fragen kam das Zeitmanagement. 
Auch dies stellte sich für uns als ein vertrautes Problemfeld dar, vor allem wegen der Prokrastination, etwas mit dem wir schon seit langem oftmals kämpfen.
Ob dies auch andere Schüler:innen persönlich betrifft, ist hauptsächlich Thema für die Umfrage.
Hier ging es uns aber im Interview hauptsächlich darum, herauszufinden wie man die Zeit ausserhalb des Stundenplans einteilen sollte, was unter anderem für unseren Agenda-Algorithmus von sehr grosser Bedeutung ist. 
Auch fragten wir hier sehr spezifisch zur Pomodoro-Technik nach, da wir auch die persönlichen Meinungen der Interviewpartner:innen in Bezug nehmen wollten, da sie als Lehrpersonen das Umfeld an der Kanti Baden gut kennen sollten und die Nützlichkeit davon einschätzen könnten.
Dies ist sehr wichtig für unseren Lerntimer, welcher auf der Pomodoro-Technik basiert.

\subsubsection{Umfragedesign}
Ganz am Schluss wollten wir noch das Umfragedesign besprechen. 
Hierzu gibt es keine konkreten Fragen, sondern es ging uns darum, genaues Feedback von den Interviewpartner:innen zu unserer bisher erstellten Umfrage einzuholen, damit wir diese so verbessern können.
Dabei achteten wir uns vor allem auch auf das Umfrage-Layout und die Art der Fragen, welche wir in der Umfrage einbauten, da unser Ziel war, eine zugängliche und, für die Befragten, unkomplizierte Umfrage zu erstellen.
Beispielsweise wollten wir wissen, ob unsere Wahl eines bestimmten Fragentyps, wie beispielsweise die sogenannten Likert-Fragen, und die Gliederung der Themen sinnvoll war, damit die Befragten die Umfrage auch abschliessen.
Hier stützten wir uns sehr stark auf das erlernte Vorwissen aus \textcite{umfrageDesign}.

\subsection {Durchführung}
Wir wussten schon früh, wer wir als unsere Interviewpartner:innen haben wollten. 
Wir haben uns zwei PPP-Lehrpersonen der Kantonsschule Baden ausgesgesucht, nämlich Herr Schmocker und Frau Suter, von welchen wir auch durch unseren Erstbewerter Theoriedokumente zum Umfrage- und Interviewdesign erhalten haben\parencite{umfrageDesign}.

Somit luden wir diese zwei Lehrpersonen per E-Mail zum Interview ein, vereinbarten ein Datum und schickten ihnen jeweils etwa eine Woche vor dem Interview den fertiggestellten Interviewfragebogen und einen Link zu unserer Umfrage, damit sie sich gut vorbereiten konnten.
Die vereinbarten Daten waren der \texttt{24. April 2025} mit Frau Suter und der \texttt{8. Mai 2025} mit Herr Schmocker.

Die Interviews führten wir in einem reservierten Klassenzimmer an der Kantonsschule Baden, jeweils am Mittag um \texttt{12:15} durch, mit einer Dauer von je etwa einer Stunde.
Diese wurden mit der Erlaubnis der Interviewpartner aufgenommen, damit sie später besser transkribiert und analysiert werden können und schrieben nebenbei reichliche Notizen.
Ein Fehler, welcher uns hier unterlief, war, dass wir die Interviews auf Mundart führten, welches diese spätere Analyse um einiges erschwerte.

Nach dem Interview mit Frau Suter tauschten wir uns noch per E-Mail aus, um weiteres Feedback für die Umfrage, welche nach dem Feedback aus den Interviews ergänzt wurde, zu gewinnen.

\subsection {Transkription}
Als erster Schritt der formellen Analyse der Interviews transkribierten wir die jeweiligen Audio-Aufnahmen der Interviews auf Papier als Word-Dateien, um die spätere detailierte Analyse zu erleichtern.
Dies war, wie in der \texttt{\hyperlink{Durchführung}{Durchführung}} erwähnt, aufgrund der Durchführung auf Mundart nicht sehr einfach, da dies aufgrund des Mangels an Mundart-Übersetzern nun komplett manuell vorlaufen musste.

Somit wurde das Gesprochene auf Hochdeutsch übersetzt und allfällige sprachliche Füller wie beispielsweise \enquote{ähm} wurden entfernt.
Wir achteten uns immer genau darauf, dass klar ist wer wann spricht. 
Zu diesem Ende wurden Textabschnitte geformt, indem alles, was eine Person zu einem Zeitpunkt ununterbrochen sagt, zusammengefügt wurde.
Dies sah dann beispielsweise folgendermassen aus:

\begin{quote}
    \textbf{[3:27]
Frau Suter:} Von den effektiven Zeiten bin ich ein wenig überfragt. Ich kann mir vorstellen, dass es auch wieder draufankommt, um was es nun genau geht. 

\end{quote}

Die Zeitangabe signifiert, wann der gesprochene Abschnitt anfängt, damit man ihn zur Überprüfung leicht wiederfinden kann und die Person wer die Aussage getroffen hat.
Ebenso haben wir uns als Hilfe notiert, wann in etwa welche Frage / welches Thema diskutiert wird.

So arbeiteten wir uns durch beide Interviews durch und transkribierten sie vollständig. Die Transkriptionen sind im Anhang zu finden.

\subsection{Analyse}
\subsubsection{Vorgehensweise}
In der Analyse ging es uns um dreierlei: Direkte Antworten und Aussagen zu unseren Fragen, allgemeine nützliche Informationen zum Lernen und allgemeine nützliche Informationen zur Web-Applikation.
Diese sammelten wir durch genaue Analyse der schriftlichen Transkription, dabei wurde immer für spätere Referenz auch auf die genaue Textquelle vermerkt. 
Auch wurden die jeweiligen Antworten von Frau Suter und Herr Schmocker zu den gegebenen Fragen verglichen.

Unser Ziel mit der Analyse bestand darin, möglichst viele konkrete Empfehlungen für unsere Web-Applikation zu erhalten, welche aus diesen nützlichen Informationen zusammengefasst wurden und nach den Features, für welche sie nützlich sind, geordnet wurden.

Das Interviewanalyse-Dokument samt allen Erkenntnissen ist im Anhang vorzufinden.
\subsubsection{Ergebnisse}
Zuerst fassen wir hier die wichtigsten Erkenntnisse zu unseren festgelegten Themenbereichen stark zusammen:

\myparagraph{Lernverhalten}
Bei den Lernmethoden und -techniken bestand die Hauptaussage der Interviewpartner:innen darin, dass es keine perfekte, allgemeingültige Methode oder Technik gibt, um das Lernen einfacher zu machen. 
Es geht darum, dass das Gehirn gut aktiviert wird und auch im Idealfall mit mehreren \enquote{Sinnen} lernt. Konzepte wie die \enquote{Lerntypen} sind eigentlich veraltet und nicht wissenschaftlich belegt.

\enquote{Geheimtipps} der Lehrpersonen empfehlen beispielsweise gute Reflektion über den Lernprozess selbst und dass, grundsätzlich, früheres Lernen besser ist.
Auch hat die eigene Einstellung, also das \enquote{Mindset}, der Lernenden einen grossen Einfluss auf den Lernerfolg hat.

Dies bedeutete für unsere Web-Applikation, dass wir fürs Lernen selbst keine spezifischen Lernmethoden einbauen können ausser das der Lernzeitalgorithmus frühes und wiederholtes Lernen fördern sollte, jedoch haben wir reichlich Lerntipps daraus gewonnen.

\myparagraph{Pausenmanagement}
Bei den Pausen emphasierte Frau Suter, dass man vor allem auf seinen Körper hören sollte. 
Das heisst, wenn man merkt, dass man zu müde wird um selbst einfache Aufgaben zu lösen, sollte man spätestens eine Pause machen.
Herr Schmocker hingegen empfahl, dass fix regulierte Pausen, wie beispielsweise bei der Pomodoro-Technik, besser seien.

Beide aber betonen auch, dass Disziplin und Selbstregulation hier eine grosse Rolle spielen und gaben genauere Empfehlungen für wie lange Pausen sein sollten.

Diese Erkenntnisse helfen auch wieder hauptsächlich bei unseren Lerntipps, aber auch mit dem Lerntimer-Feature.

\myparagraph{Stressmanagement}
Hier wurde von beiden Interviewpartner:innen betont, dass Stress vor allem durch gute Planung verhindert werden kann, und auch genau der Mangel dessen oftmals hier an der Kantonsschule Baden zu diesem Stress führt, welches den Nutzen für unsere Agenda und den Lernzeitalgorithmus unterstützt.
Sie emphasierten aber auch, dass es auch an den Lehrpersonen liegen kann, vor allem wenn diese Rücksichtslos Prüfungen und Termine einplanen, weswegen auch sie eine grosse Verantwortung tragen.
Auch wurden, in Verbindung mit dem Pausenmanagement, gute Aktivitäten zum Stressabbau empfohlen, wie beispielsweise Sport.

Somit haben wir auch hier reichlich Material für unsere Daily- und Lerntipps gewonnen, aber auch wichtige Validation für unsere Agenda. 
Der Lernzeitalgorithmus sollte so auch durch gute Planung zum Stressabbau beitragen können.

\myparagraph{Zeitmanagement}
Beide Interviewpartner:innen befanden die Pomodoro-Technik als eine gute Methode um die Disziplin beim Lernen zu fördern, sie brauchen die Technik auch oftmals selbst. 
Auch verstärkten sie den Nutzen von guter Planung mit Wochenplänen und Agenden, damit die verfügbare Zeit gut benutzt wird und eine gute Balance zwischen Lernen und Freizeit etabliert werden kann.
Herr Schmocker beispielsweise empfiehlt, dass man mindestens einen arbeitsfreien Tag in der Woche haben sollte, egal wie beschäftigt man sonst ist.

Auch dieser Themenblock unterstützt so unsere Agenda und den Lerntimer und liefert uns einige gute Vorschläge, wie wir diese Features verbessern können.

\subsubsection{Konkrete Empfehlungen für die Web-Applikation}
Nun wurden diese Ergebnisse zu konkreten Empfehlungen für unsere Web-Applikation zusammengefasst, also wie genau gewisse Features gestaltet und ergänzt werden sollten.
Diese Empfehlungen drehen sich hauptsächlich um die Lerntipps, die Agenda und dem Lernzeitalgorithmus, dem Lerntimer und ein paar extra Features.

Alle diese Empfehlungen sind im Anhang in unserem Interviewanalyse-Dokument genauer beschrieben und ausschliesslich aus den Erkenntnissen der Interviews abgeleitet.
Dort wird auch genauer aufgelistet, welche dieser Empfehlungen für uns als \enquote{Muss}-Anforderungen und welche als \enquote{Kann}-Anforderungen gelten, wie in der Vorgehensweise der Recherche definiert.

\myparagraph{Daily- und Lerntipps}
Hier geht es hauptsächlich darum, dass Erkenntnisse welche sich sonst nicht direkt in die Web-Applikation einbauen lassen, den Endnutzer:innen vermittelt werden können.
Davon haben wir einiges aus allen unserer Themenbereichen gesammelt, grundsätzlich alles, in dem es um die persönliche Einstellung oder Handlungen geht.

\texttt{Empfohlene Tipps sind beispielsweise:}
\begin{itemize}
    \item Multisensorisches Lernen, da dies verschiedene Bereiche des Gehirns aktiviert, was zu effizienterem Lernen führt.
    \item Kurze Erklärungen bekannter Lernmethoden, wie die bereits erwähnten SQ3R- und KWL-Methoden\parencite{SQ3R}.
    \item Selbstreflektion über den Lernprozess und Selbsttests sind wichtig, ebenso wie externes Feedback.
    \item Tipps zum \enquote{richtigen} Mindset und Motivationssprüche.
    \item Tipps für eine gute Zeit- und Pausenplanung.
    \item Tipps zum Stressabbau, wie Sport und Entspannungstechniken.
\end{itemize}

Diese Tipps werden dann, gestützt auf den Aussagen von Frau Suter und Herr Schmocker, in volle Sätze formuliert und in die Daily- und Lerntipps-Features integriert.

\myparagraph{Agenda und Lernzeitalgorithmus}
Im Laufe der Interviews kamen einige gute Empfehlungen für die Agenda und den Lernzeitalgorithmus heraus, hauptsächlich betreffend wann man lernen soll und wann nicht.
Diese Empfehlungen sind für uns sehr wichtig, da diese Features den Kern unserer Web-Applikation darstellen.
So sollte der/die Endnutzer:in beispielsweise einstellen können, wie viel vom Wochenende vom Lernzeitalgorithmus aufgebraucht werden darf, bis wie spät am Abend gelernt werden darf, ob es einen idealen Lernzeitpunkt gibt, welcher zuerst gefüllt werden sollte.
Auch soll der Lernzeitalgorithmus ebenso eine Warnung geben, wenn ein Tag zu voll wird und möglichst wiederholtes Lernen fördern.

Dies und weiteres soll den Benutzer:innen helfen, ihre Zeit sinnvoll mithilfe unserer Agenda einzuteilen und Lernstress vermeiden.

\myparagraph{Lerntimer}
Der Lerntimer, welcher auf der Pomodoro-Technik basiert, wurde auch durch die Interviews stark validiert, welches an sich schon eine gute Empfehlung darstellt.
Jedoch können wir aufgrund diesen Erkenntnissen auch auf der Pomodoro-Technik aufbauen und das Konzept dessen ein wenig weiterentwickeln.
Eine Empfehlung ist beispielsweise, dass der Timer auch Lernintervalle bis um eine gewisse Uhrzeit unterstützen soll, damit längere Aufgaben wo man nicht ständig unterbrochen werden will und sich gut konzentrieren kann, durchgeführt werden können.

\myparagraph{Miscellaneous}
Es gab auch ein paar wenige Empfehlungen, welche nicht direkt in die obigen Kategorien passen. 
Diese zwei Empfehlungen schlagen neue Features vor, nämlich:
\begin{itemize}
    \item Ein \textbf{Wochenplaner} mit welchen man sich (optionale) tägliche Aufgaben aufschreiben und erledigen können soll.
    \item Eine \textbf{KI-Page} welche einen gesunden Umgang mit KI empfiehlt und auch vor den Gefahren für den Lernerfolg warnt, wenn man sich zu sehr darauf verlässt.
\end{itemize}

Dieses sind beides nicht sehr wichtige Empfehlungen, deswegen auch als 'Kann'-Anforderungen klassifiziert und können jederzeit ergänzt werden.

\section{Umfrage}
\subsection{Ziel}
Das wesentliche Ziel unserer Umfrage bestand darin, unsere Web-Applikation und unseren Features aus unserer Zielgruppe, welche ja dann auch die Web-Applikation benutzen sollten, so gut wie möglich zu validieren.
Dazu gehört, zum Beispiel, herauszufinden ob unsere Annahmen über die Bedürfnisse der Schüler:innen tatsächlich stimmen, beispielsweise im Bereich 
Hier fliessen die Erkenntnisse der Literaturstudie und der Interviews hinein, damit wir unsere gewählten Themenbereiche gut abdecken können.

\subsection{Entwicklung}
Hauptsächlich wurden hier wieder die selben Themenbereiche wie bei den Interviews abgedeckt, wie bei den anderen Recherche-Segmenten, jedoch mit leichten Änderungen.
Hier interessierte uns hauptsächlich, ob die Probleme, welche unser Produkt lösen soll, tatsächlich in unserer Zielgruppe existieren, d.h. wir suchten Validation für unsere Web-Applikation.
Deshalb wurden die Fragen auf das Verhalten der Befragten gerichtet, wie beispielsweise ob sie oftmals Prüfungsstress erleben und weshalb, wie früh sie mit dem Lernen anfangen, ob sie regelmässig Pausen machen und ob sie interesse an einer Lern-Applikation hätten.

Zuerst mussten wir aber wählen, mit was für einem Programm wir die Umfrage erstellen wollten. 
Als Host für die Umfrage wählten wir Microsoft Forms, da wir bereits mit Forms vertraut waren und dies verwenden konnten. 
Dies machte Kollaboration einfach, da wir unsere Microsoft-Accounts von der Schule brauchen konnten. 
Ebenso lässt Microsoft Forms zu, dass die Daten einfach in einer verbundenen Excel-Datei geöffnet werden können, was die spätere Analyse vereinfachte.

Somit konnten wir mit der Entwicklung der Umfrage beginnen.
Wir stützten uns dabei stark auf das bisherig erwähnte Theorie-Dokument, welches uns von den PPP-Lehrpersonen zur Verfügung gestellt wurde: \textcite{umfrageDesign}.
Dieses Dokument brachte uns einige Konzepte bei, darunter, wie man die Beibehaltung von den Befragten erhöht, indem man die Umfrage kurz und möglichst unkompliziert entwirft, für eine gute Vergleichbarkeit möglichst geschlossene Fragen verwendet.
Deswegen lagen unsere Kernziele für das Design der Umfrage folgendermassen:
\begin{itemize}
    \item Die Umfrage ist unkompliziert, sodass die Motivation der Befragten die Umfrage auszufüllen und somit die Antwortenquote bleibt, und sodass wenig missverstanden werden kann.
    \item Eine klare Gruppierung der Themenbereiche.
    \item Keine unnötigen Fragen - die Fragen sind stets nach unseren Zielen in der Recherche gerichtet.
\end{itemize}

\subsubsection{Grundlegende Informationen}
Als Erstes mussten wir natürlich ein paar grundlegende Daten über unsere Befragten sammeln, damit wir gut Vergleiche zwischen Gruppen ziehen können.
Die Umfrage ist jedoch anonym - die einzigen Informationen, welche wir hier sammeln wollten sind die Mittelschule, also Gymnasium, WMS \& IMS, und die Stufe (1. - 4. Klasse) in der die Befragten sind.
Somit wollten wir aber später gut vergleichen können, wie sich beispielsweise das Lernverhalten zwischen den Stufen und Mittelschulen unterscheidet. 

\subsubsection{Lernverhalten}
Hier interessierte uns hauptsächlich, wann die Schüler:innen mit dem Lernen anfangen, da dies klar mit der Theorie aus der Literaturstudie und den Interviews vergleichbar ist.
Wir lernten dort, dass früheres und wiederholtes Lernen zu einem besseren Lernergebnis führen soll, aber wir wissen auch aus eigener Erfahrung, dass dies viel mehr Aufwand darstellt und die Versuchung, erst sehr spät anzufangen, deswegen sehr gross ist.
Gleich danach stellten wir zwei einfache Fragen, wie zufrieden die Befragten mit ihrem Lernaufwand seien und was sich ändern müsste, damit sie zufriedener wären.
Somit wollten wir vergleichen können, ob es eine Korrelation zwischen der Lernzufriedenheit und dem Lernzeitpunkt gibt, und was die Schüler:innen selbst denken, dass sie machen könnten bzw. bräuchten.

\subsubsection{Stress- und Pausenmanagement}
Wir haben in den Interviews zum Stressmanagement hauptsächlich einiges über Methoden, um Stress abzubauen gelernt, und die Perspektive der Lehrpersonen zu was die häufigsten Gründe für Prüfungsstress seien.
Hier konnten wir nun aber selbst nachfragen, ob die Schüler:innen oftmals sich vor Prüfungen gestresst fühlen und weshalb dies so sei.
So erhofften wir uns herauszufinden, was die häufigsten Motivationen für diesen Stress sind, damit wir uns auch Gedanken machen konnten, ob und wie das in der Applikation adressiert werden könnte.
Wir benutzten hierfür eine sogenannte Likert-Frage, wo mehrere Aussagen mit gewissen Häufigkeitsgraden beantwortet werden können, in unserer Umfrage verwendeten wir: \enquote{Nie}, \enquote{Selten}, \enquote{Manchmal}, \enquote{Oft} und \enquote{Immer}.
\begin{figure}
    \centering
    \includegraphics[width=\linewidth]{img/StressGründeFrage.png}
    \caption[Darstellung einer Likert-Frage zu den Gründen für Prüfungsstress.  Eigener Screenshot von KantiKoala, 9.11.25]{Darstellung einer Likert-Frage zu den Gründen für Prüfungsstress}
    \label{fig:stress_gründe}
\end{figure}

Da wir möglichst geschlossene Fragen haben wollten, gaben wir vier für uns plausible Gründe für Prüfungsstress basiert auf den Interviews und unseren persönlichen Erfahrungen.
Nach Feedback von Frau Suter beschlossen wir uns jedoch auch, ein \enquote{Weiteres}-Feld einzubauen, da es wohl wahrscheinlich noch andere Gründe für Prüfungsstress geben könnte.

Da das Stressmanagement kein besonders Feld darstellt, beschlossen wir uns, das Pausenmanagement auf die gleiche Seite zu packen, da Pausen als Entspannung und Stressabbau für uns eine Antithese zum Stress darstellten.
So wollten wir auch untersuchen, ob und wie die Befragten bisher Pausen führten, da dies von Relevanz für unsere Lerntipps und unseren Lerntimer sein könnte, und als Vergleich mit den theoretischen Grundlagen aus den Interviews interessant sein könnte.

Wegen unserem Ziel, die Umfrage unkompliziert darzustellen, beschlossen wir uns, das sogenannte \enquote{Branching} zu verwenden, damit nur die Befragten, welche die ersten Ja/Nein-Fragen zum Prüfungsstress und dem Pausenmanagement mit \enquote{Ja} beantworten, überhaupt die weiterführenden Fragen sehen.
Die folgenden Abbildungen demonstrieren, wie solches \enquote{Branching} aussieht.

\begin{figure}
    \centering
    \includegraphics[width=\linewidth]{img/BranchingOn.png}
    \caption[Demonstration des Branching-Features vosn Microsoft Forms in unserer Umfrage (Bild 1). Eigener Screenshot von KantiKoala, 9.11.25]{Ansicht der Seite wenn bei \enquote{Fühlen Sie sich vor Prüfungen oft gestresst?} mit \enquote{Nein} beantwortet wird.}
    \label{fig:branching-1}
\end{figure}
\begin{figure}
    \centering
    \includegraphics[width=\linewidth]{img/BranchingOff.png}
    \caption[Demonstration des Branching-Features von Microsoft Forms in unserer Umfrage (Bild 2). Eigener Screenshot von KantiKoala, 9.11.25]{Ansicht der Seite wenn mit \enquote{Ja} beantwortet wird.}
    \label{fig:branching-2}
\end{figure}

\break

\subsubsection{Zeitmanagement}


\subsubsection{Musik}
Ein weiteres Feature, welches uns interessierte, wäre die Option, Musik in die Lernapp zu integrieren, welche man beim Lernen hören könnte, ähnlich wie diese \enquote{Lofi hip hop radio beats to relax/study to} videos, welche man auf YouTube vorfinden kann \parencite{LofiVideo}.
Das Musik störend für das Lernen ist, ist eine sehr kontroverse These \parencite{Music}.
Da wir dies aber selbst gern machen, wollten wir untersuchen, ob Musik von unserem Zielpublikum selbst beim Lernen gebraucht wird, und ob sie als störend oder nicht störend empfunden wird.

\subsubsection{Lern-App}


Die abgeschlossene Umfrage kann im Anhang vorgefunden werden.

\subsection{Durchführung}

\subsection{Analyse}
\subsubsection{Vorgehensweise}

\subsubsection{Ergebnisse}


\section{Fazit der Recherche}
% ------------------------------Programmieren------------------------------
\chapter{Methodik -- Programmieren der Web-Applikation}

\section[Erste Entscheidungen]{Erste Entscheidungen\texorpdfstring{\aifootnotemark}{}}
\aifootnotetext{04.11.2025}
Zu Beginn stand die grundsätzliche Plattformwahl im Zentrum: native Applikation (z.\,B. für Smartphones) oder webbasierte Lösung. Unter Berücksichtigung von Geräteunabhängigkeit, Verteil- und Updateaufwand, Entwicklungsressourcen sowie des verfügbaren Zeitrahmens erwies sich eine Web-Applikation als zweckmässig. Sie ist plattformagnostisch im Browser nutzbar, benötigt keine Installation und lässt sich zentral aktualisieren. Zudem reduziert eine einheitliche Codebasis den Implementierungs- und Wartungsaufwand gegenüber mehreren nativen Anwendungen für unterschiedliche Betriebssysteme.

Auf Basis dieser Entscheidung fiel die Wahl der Technologie auf Python mit dem Microframework Flask \parencite{flask_docs}. Ausschlaggebend waren vorhandene Vorkenntnisse. Als Entwicklungsumgebung wurde Visual Studio Code verwendet, da es durch integrierte Funktionen wie Code-Assistenz (z. B. Copilot) einen effizienten Entwicklungsprozess ermöglicht und schnelle Unterstützung bietet.

Für die kollaborative Arbeit kamen Git als Versionsverwaltung und GitHub als zentrales Remote-Repository zum Einsatz. Der Quellcode wurde dort gemeinsam versioniert und ausgetauscht; regelmässige Synchronisationen (Push/Pull) stellten einen konsistenten, aktuellen Projektstand sicher.

\subsection{Server- und Hosting-Entscheidung}
Für das Hosting der Produktivumgebung fiel die Wahl auf DigitalOcean. Ausschlaggebend waren: (1) Verfügbarkeit kostenloser Credits über das GitHub Student Developer Pack, (2) integrierte PostgreSQL-Datenbank-Hosting-Optionen, (3) Unterstützung für Gunicorn/WSGI-Deployment. Die Domain \texttt{kantikoala.app} wurde ebenfalls über das Student Pack mit kostenloser \texttt{.app}-TLD registriert.

\section[Anforderungen]{Anforderungen\texorpdfstring{\aifootnotemark}{}}
\aifootnotetext{04.11.2025}
Bevor wir mit dem Programmieren der Web-Applikation beginnen konnten, mussten wir uns zuerst über die Anforderungen an die Applikation klar werden. Da es sich um eine Web-Applikation handelt, welche den Schüler:innen der Kantonsschule Baden helfen soll, mussten wir uns überlegen, welche Funktionen die Applikation beinhalten sollte und wie diese umgesetzt werden könnten.

Die Anforderungen an die Kanti Koala Web-Applikation sind wie folgt:
\begin{itemize}
    \item \textbf{Home-Screen}: Von dem Home-Screen sollte man auf seinen Account und die Agenda zugreifen können. Zusätzlich sollte hier jeden Tag ein allgemeiner Tipp für die Kantonsschule angezeigt werden.
    \item \textbf{Account Management}: Die Nutzer:innen sollten sich registrieren, einloggen, ihr Passwort zurücksetzen und ihre Account-Einstellungen ändern können. Sie sollten die Möglichkeit haben, ihr Passwort zu ändern und allfälligerweise ihr Account zu löschen.
    \item \textbf{Agenda}: Die Nutzer:innen sollten ihren Stundenplan eintragen können, sowohl manuell wie auch durch den Import einer .ics-Datei. Ebenso sollte man neue Ereignisse eintragen können. Die Ereignisse sollten veränderbar sein. Die Farbe der Ereignisse sollten auch frei bestimmbar sein. Die Agenda sollte auch einen Lernzeitalgorithmus beinhalten, welcher automatisch Lernzeiten basierend auf den eingetragenen Ereignissen und den Prioritätseinstellungen der Nutzer:innen plant.
    \item \textbf{Notenverwaltung}: Die Nutzer:innen sollten ihre Noten für jedes Fach eintragen können. Die Noten sollten veränderbar und löschbar sein. Die Nutzer:innen sollten auch ihre Semester verwalten können, indem sie neue Semester hinzufügen, bestehende Semester bearbeiten und löschen können.
    \item \textbf{Lerntimer}: Die Nutzer:innen sollten einen Pomodoro-Timer verwenden können, um ihre Lernzeiten zu strukturieren. Der Timer sollte anpassbar sein, sodass die Nutzer:innen die Länge der Lern- und Pausenintervalle einstellen können.
    \item \textbf{UI}: Die Web-Applikation sollte ein ansprechendes UI haben, welches die Kernfunktionen (Agenda, Noten, Timer) ohne Schulung zugänglich macht.
\end{itemize}

\section[Systemarchitektur]{Systemarchitektur\texorpdfstring{\aifootnotemark}{}}
\aifootnotetext{04.11.2025}
Dieser Abschnitt beschreibt die technische Grundlage der Kanti Koala Web-Applikation, einschliesslich der Systemarchitektur, der Code-Struktur und der Deployment-Infrastruktur.

\subsection{Überblick und Framework-Wahl}
Die Kanti Koala App ist als monolithische Webanwendung konzipiert, die auf einem zentralen Backend-Server läuft. Das Kernstück der Anwendung ist das Python-Microframework Flask \parencite{flask_docs}. Es steuert das Routing (die Zuordnung von URLs zu Funktionen), verarbeitet HTTP-Anfragen (GET, POST, usw.) und rendert die HTML-Templates für den Benutzer.

\subsection{Wichtige Komponenten und Pakete}
\begin{itemize}
    \item \textbf{Flask-SQLAlchemy} \parencite{flask_sqlalchemy}: Dient als Object-Relational Mapper (ORM) für die Datenbank. Es ermöglicht die Definition von Datenbanktabellen als Python-Klassen (Models) und vereinfacht Datenbankabfragen.
    \item \textbf{Flask-Bcrypt} \parencite{flask_bcrypt_docs}: Wird für die Sicherheit der Benutzerpasswörter eingesetzt. Es hasht und verifiziert Passwörter mithilfe des bcrypt-Algorithmus.
    \item \textbf{Flask-Migrate} \parencite{flask_migrate}: Erleichtert Schema-Migrationen der Datenbank, wenn sich die Modelle (Tabellenstruktur) ändern.
    \item \textbf{Resend} \parencite{resend_docs}: Dient als E-Mail-API für den Versand von systemgenerierten E-Mails, insbesondere für die \enquote{Passwort vergessen}-Funktion.
    \item \textbf{icalendar} \parencite{icalendar_docs}: Eine Python-Bibliothek, die zum Parsen und Importieren von \texttt{.ics}-Kalenderdateien verwendet wird, um den Schulnetz-Stundenplan zu importieren.
    \item \textbf{itsdangerous} \parencite{itsdangerous_docs}: Wird verwendet, um sichere, zeitlich begrenzte Tokens zu generieren, die für die \enquote{Passwort zurücksetzen}-Links benötigt werden.
\end{itemize}

\subsection{Code-Struktur und Application Factory Pattern}
Um die Wartbarkeit und Skalierbarkeit der Anwendung zu verbessern, wurde die ursprüngliche Code-Struktur von einer einzigen \texttt{app.py}-Datei in ein modulares Python-Paket namens \texttt{kkoala} umstrukturiert. Dieser Ansatz folgt dem \enquote{Application Factory}-Pattern, einer bewährten Methode für Flask-Anwendungen \parencite{flask_structure_best_practices}.

\subsubsection{Application Factory (\texttt{kkoala/\_\_init\_\_.py})}
Das Herzstück des Pakets ist die \texttt{create\_app}-Funktion. Anstatt einer globalen App-Instanz wird die Anwendung durch diesen \enquote{Factory}-Aufruf erzeugt. Dies ermöglicht es, verschiedene Konfigurationen (z.B. für Entwicklung, Test oder Produktion) dynamisch zu laden und macht die Anwendung robuster. In dieser Datei werden auch die Flask-Erweiterungen initialisiert und die Blueprints registriert.

\subsubsection{Konfiguration (\texttt{kkoala/config.py})}
Diese Datei enthält Konfigurationsklassen (z.B. \texttt{DevConfig}, \texttt{ProdConfig}). Sie verwalten wichtige Einstellungen wie den \texttt{SECRET\_KEY}, die Datenbank-URL und API-Schlüssel. Die Konfiguration wird je nach Umgebungsvariable beim Start der App ausgewählt.

\subsubsection{Blueprints für Routen (\texttt{kkoala/routes/})}
Die Routen der Anwendung sind in \enquote{Blueprints} aufgeteilt, die eine Gruppierung von zusammengehörigen Endpunkten ermöglichen:
\begin{itemize}
    \item \textbf{\texttt{auth.py}}: Enthält alle Routen für die Benutzerauthentifizierung (Login, Registrierung, Passwort zurücksetzen).
    \item \textbf{\texttt{events.py}}: Verwaltet die API-Endpunkte für die Agenda, einschliesslich des Erstellens, Bearbeitens und Löschens von Kalendereinträgen sowie den Start des Lernalgorithmus.
    \item \textbf{\texttt{grades.py}}: Beinhaltet die API für das Notenmanagement.
    \item \textbf{\texttt{main.py}}: Definiert die Hauptrouten der Webseite, wie die Startseite.
    \item \textbf{\texttt{settings.py}}: Steuert die Einstellungsseite und die zugehörige Speicherlogik.
\end{itemize}

\subsubsection{Kernlogik und Hilfsfunktionen}
\begin{itemize}
    \item \textbf{\texttt{kkoala/algorithms.py}}: Enthält ausschliesslich die komplexe Logik des Lernzeitalgorithmus (LZA).
    \item \textbf{\texttt{kkoala/utils.py}}: Beinhaltet wiederverwendbare Hilfsfunktionen und Decorators:
    \begin{itemize}
        \item \textbf{\texttt{@login\_required}}: Prüft automatisch, ob ein Benutzer angemeldet ist; andernfalls Umleitung zur Login-Seite.
        \item \textbf{\texttt{@csrf\_protect}}: Schützt vor Cross-Site Request Forgery Angriffen.
    \end{itemize}
    \item \textbf{\texttt{kkoala/extensions.py}}: Initialisiert Flask-Erweiterungen, um zirkuläre Importfehler zu vermeiden.
\end{itemize}

\subsection{Deployment und WSGI-Schnittstelle}
Die Datei \texttt{wsgi.py} im Hauptverzeichnis ist der standardisierte Einstiegspunkt für den Webserver. Sie importiert die \texttt{create\_app}-Factory und erstellt das Flask-\texttt{application}-Objekt, das der WSGI (Web Server Gateway Interface) Spezifikation entspricht \parencite{chaitanya_srivastav_wsgi}.

Für den produktiven Einsatz verwenden wir Gunicorn (\enquote{Green Unicorn}), einen robusten WSGI-HTTP-Server. Während der eingebaute Entwicklungsserver von Flask für Tests ausreicht, ist er nicht für hohe Last ausgelegt. Gunicorn agiert als leistungsfähiger \enquote{Middleman} zwischen dem Internet und unserer Flask-Anwendung, verwaltet mehrere Worker-Prozesse und stellt so Leistung und Stabilität sicher \parencite{codesignal_gunicorn}.

\subsection{Frontend-Struktur}
Die Benutzeroberfläche wird dynamisch auf dem Server generiert und als fertige HTML-Seiten an den Browser gesendet.

\begin{itemize}
    \item \textbf{Templates (\texttt{kkoala/templates/})}: Enthalten alle HTML-Dateien. Flask verwendet die Template-Engine Jinja, um Python-Code direkt in HTML einzubetten (Schleifen, bedingte Blöcke, Variablenausgabe), was die Seiten dynamisch und personalisiert macht \parencite{jinja_templates, flask_templating}.
    \item \textbf{Statische Dateien (\texttt{kkoala/static/})}: Enthält CSS-Dateien für das Styling und Bilder (Logos, grafische Elemente), die vom Browser direkt geladen werden.
\end{itemize}

\section{Datenstruktur und Datenbankmodelle}

\subsection{Datenbankwahl und Konfiguration}
Die Datenstruktur ist in der Datei \texttt{models.py} durch SQLAlchemy-Modelle definiert. Für die lokale Entwicklung wird SQLite verwendet; für die Produktionsumgebung kommt PostgreSQL zum Einsatz, das besser für Mehrbenutzerbetrieb und hohe Last geeignet ist \parencite{astera_postgres_vs_sqlite, flask_database_tutorial}.

\subsection[Datenbankmodelle und Schema]{Datenbankmodelle und Schema\texorpdfstring{\aifootnotemark}{}}
\footnotetext{ChatGPT (Version GPT-5): \enquote{Describe the models from models.py in LaTeX in German, without using tables. Just use text. }, 04.11.2025. Antwort ganz übernommen.}
Die Datenbank besteht aus sieben Hauptmodellen, die die Nutzerdaten und die Planungslogik abbilden.

\myparagraph{User}
Dieses Modell speichert die Authentifizierungsdetails und dient als zentraler Ankerpunkt für alle anderen Daten des Nutzers.
\begin{description}
    \item[\textbf{id}] Eindeutige ID des Nutzers (Primary Key).
    \item[\textbf{username}] Der gewählte Benutzername (eindeutig, notwendig).
    \item[\textbf{password}] Das gehashte Passwort (notwendig).
    \item[\textbf{email}] Die E-Mail-Adresse des Nutzers (eindeutig, notwendig).
\end{description}

\myparagraph{Settings}
Speichert globale Einstellungen für den Lernalgorithmus, die dem \texttt{User} zugeordnet sind.
\begin{description}
    \item[\textbf{id}] Eindeutige ID (Primary Key).
    \item[\textbf{user\_id}] Fremdschlüssel zur \texttt{User}-Tabelle (notwendig).
    \item[\textbf{learn\_on\_saturday}] Boolesche Variable, ob am Samstag gelernt werden soll (Standard: False).
    \item[\textbf{learn\_on\_sunday}] Boolesche Variable, ob am Sonntag gelernt werden soll (Standard: False).
    \item[\textbf{preferred\_learning\_time}] Bevorzugte Startzeit für Lernblöcke (Standard: 18:00).
    \item[\textbf{study\_block\_color}] Hex-Code für die Farbe der Lernblöcke (Standard: \#0000FF).
\end{description}

\myparagraph{PrioritySetting}
Definiert die spezifischen Parameter für jede Prioritätsstufe des Lernalgorithmus.
\begin{description}
    \item[\textbf{id}] Eindeutige ID (Primary Key).
    \item[\textbf{settings\_id}] Fremdschlüssel zur \texttt{Settings}-Tabelle (notwendig).
    \item[\textbf{priority\_level}] Die Prioritätsstufe (Integer, notwendig).
    \item[\textbf{color}] Die dem Prioritätslevel zugeordnete Farbe (Hex-Code, notwendig).
    \item[\textbf{max\_hours\_per\_day}] Maximale Lernstunden pro Tag für diese Priorität.
    \item[\textbf{total\_hours\_to\_learn}] Die gesamte zu lernende Stundenanzahl für diese Priorität.
\end{description}

\myparagraph{Event}
Speichert Kalendereinträge des Nutzers sowie Metadaten für den Planungsalgorithmus.
\begin{description}
    \item[\textbf{id}] Eindeutige ID (Primary Key).
    \item[\textbf{user\_id}] Fremdschlüssel zur \texttt{User}-Tabelle (notwendig).
    \item[\textbf{title}] Titel des Ereignisses.
    \item[\textbf{start}] Startzeitpunkt im ISO-Format (notwendig).
    \item[\textbf{end}] Endzeitpunkt im ISO-Format (optional).
    \item[\textbf{color}] Farbe des Ereignisses.
    \item[\textbf{priority}] Prioritätsstufe (Integer).
    \item[\textbf{recurrence}] Wiederholungsregel des Ereignisses.
    \item[\textbf{recurrence\_id}] Eindeutige ID zur Gruppierung wiederkehrender Ereignisse.
    \item[\textbf{all\_day}] Boolesche Variable, ob das Ereignis ganztägig ist (Standard: False).
    \item[\textbf{locked}] Boolesche Variable für den Algorithmus; \texttt{True} bedeutet, das Ereignis ist fixiert (Standard: True).
    \item[\textbf{exam\_id}] ID des zugehörigen Examens, falls zutreffend.
\end{description}

\myparagraph{Semester, Subject und Grade}
Diese Modelle bilden die akademische Hierarchie ab.
\begin{description}
    \item[\textbf{Semester}] Speichert akademische Abschnitte. Enthält \textbf{user\_id} (Fremdschlüssel) und \textbf{name}.
    \item[\textbf{Subject}] Speichert Fächer innerhalb eines Semesters. Enthält \textbf{semester\_id} (Fremdschlüssel) und \textbf{name}.
    \item[\textbf{Grade}] Speichert Bewertungen für Fächer. Enthält \textbf{subject\_id} (Fremdschlüssel), \textbf{name}, \textbf{value}, \textbf{weight} und \textbf{counts}.
\end{description}

\subsection{Beziehungsstruktur}
Die Tabellen in der Datenbank sind durch Fremdschlüssel (Foreign Keys) miteinander verbunden, um die Beziehungen zwischen den verschiedenen Datenmodellen abzubilden. Beispielsweise verknüpft der Fremdschlüssel \texttt{user\_id} in der \texttt{Event}-Tabelle jedes Ereignis mit dem zugehörigen Nutzer. Dadurch wird sichergestellt, dass jeder Kalendertermin eindeutig einem Benutzerkonto zugeordnet ist. Ähnliche Verknüpfungen existieren auch zwischen \texttt{Settings} und \texttt{User}, \texttt{PrioritySetting} und \texttt{Settings}, \texttt{Semester} und \texttt{User}, \texttt{Subject} und \texttt{Semester} sowie \texttt{Grade} und \texttt{Subject}.

Die Abhängigkeiten und Kaskadenlöschungen (z.B. ein gelöschter \texttt{User} löscht alle seine \texttt{Events}, \texttt{Semesters} und \texttt{Settings}) sind über Fremdschlüsselverweise in allen untergeordneten Tabellen implementiert. Die zentralen Verbindungen sind:
\begin{itemize}
    \item \texttt{User} $\to$ \texttt{Settings} (1:1), \texttt{Events} (1:n), \texttt{Semester} (1:n)
    \item \texttt{Settings} $\to$ \texttt{PrioritySetting} (1:n)
    \item \texttt{Semester} $\to$ \texttt{Subject} (1:n)
    \item \texttt{Subject} $\to$ \texttt{Grade} (1:n)
\end{itemize}

\section{Kernfunktionen (Features)}

\subsection[Authentifizierung]{Authentifizierung\texorpdfstring{\aifootnotemark}{}}
\aifootnotebasis{04.11.2025}
Um eine App mit verschiedenen Nutzern zu haben, brauchen wir ein gutes Authentifizierungssystem. Das bedeutet, dass es eine Anmelde- und Registrierungsfunktion sowie eine Option zum Vergessen des Passworts, eine Option zum Ändern des Passworts und schliesslich auch eine Option zum Löschen des Kontos geben muss.

Natürlich können wir ein Passwort nicht im Klartext speichern, denn das wäre ein Sicherheitsrisiko und ein ethisches Risiko, weil wir als Entwickler dann die Passwörter der einzelnen Konten einsehen können. Die einfachste Lösung für dieses Problem besteht darin, das Kennwort zu hashen. Unter Passwort-Hashing versteht man die algorithmische Umwandlung eines Passworts in Chiffretext oder eine unumkehrbar verschleierte Version seiner selbst \parencite{password_hashing}. Ein anderes wichtiges Konzept ist das Salting. Salting ist die Praxis, zufällige Daten (ein \enquote{Salt}) zu einem Passwort hinzuzufügen, bevor es gehasht wird. Dies verhindert Angriffe mit vorgefertigten Tabellen (Rainbow Tables), da das gleiche Passwort mit unterschiedlichen Salts zu unterschiedlichen Hashes führt \parencite{password_salting_rainbow_tables}. Glücklicherweise kümmert sich das Flask-Bcrypt-Modul um all diese Dinge für uns, sodass wir uns keine Sorgen machen müssen, wie wir das Passwort hashen und salten \parencite{flask_bcrypt_docs}.

Sobald man sich anmeldet oder registriert, wird man auf die Startseite der Website weitergeleitet.

Sehr wichtig bei jedem Anmeldesystem ist natürlich eine Option das Passwort zurückzusetzen wenn man es vergisst. Um so ein System zu haben, ist es wichtig, dass man den Link, um das Passwort zurückzusetzen, nur einmal verwenden kann. Um dies zu erreichen, haben wir den Token, welches gebraucht wird, um zu überprüfen, dass das Passwort für den richtigen User zurückgesetzt wird, mit dem Hash vom alten Passworts generiert. Wir sind dank \textcite{password_reset} auf diese Idee gekommen. Das Prinzip funktioniert, da das Passwort sich ändern wird, und somit kann man den gleichen Token nicht wieder verwenden. Zudem brauchen wir, um die E-Mail zu verschicken, eine API, damit wir sie mit unserer eigenen Domaine verschicken können. Wir haben uns für die API von Resend entschieden, da sie eine kostenlose Stufe hat, die für unser Projekt ausreicht.

Im Falle, dass ein Nutzer zum Beispiel mit einer E-Mail, die schon gebraucht wird, sich registrieren will, oder mit einem falschen Passwort sich anmelden will, bekommt der Nutzer eine Fehlermeldung angezeigt, die ihm sagt, was falsch gelaufen ist, mithilfe von Flash-Nachrichten, welche eine Funktion von Flask ist \parencite{realpython_flask_flashes}.

\begin{figure}[!htp]
    \centering
    \includegraphics[width=0.6\textwidth]{img/login-screenshot.png}
    \caption[Screenshot des Anmeldeprozesses. Eigener Screenshot von Kanti Koala, 08.11.2025.]{Screenshot des Anmeldeprozesses}
    \label{fig:auth_flow}
\end{figure}

\subsection[Agenda]{Agenda\texorpdfstring{\aifootnotemark}{}}
\aifootnotebasis{04.11.2025}
Die Agenda ist eine zentrale Kernfunktion der Anwendung. Sie sollte die Operationen Erstellen, Bearbeiten und Löschen von Terminen mit minimalem Aufwand ermöglichen.

Jedes Ereignis umfasst die Attribute Titel, Start- und Endzeit, eine optionale Wiederholungsregel, eine Farbe sowie eine Priorität, die für den Lernzeitalgorithmus relevant ist. Zusätzlich kann ein Ereignis als ganztägig gekennzeichnet werden. Die Zuordnung zum jeweiligen Konto erfolgt über die aktuell authentifizierte Sitzung; die Benutzer-ID wird beim Anlegen eines Ereignisses automatisch hinterlegt.

Für die Darstellung setzen wir die JavaScript-Bibliothek FullCalendar ein \parencite{full_calendar}. Sie ist etabliert, funktionsreich und gut dokumentiert und ermöglicht unterschiedliche Ansichten (z.\,B. Monat, Woche, Liste), Drag-and-Drop-Anpassungen sowie eine direkte Synchronisation mit dem Backend über JSON-Endpunkte.

Die Agenda unterstützt zudem den Import des individuellen Stundenplans aus \texttt{schulnetz}, der von der Kantonsschule zur Anzeige von Stundenplänen genutzt wird. Da \texttt{schulnetz} Kalenderdaten im \texttt{.ics}-Format exportiert, kann die Datei zunächst dort exportiert und anschliessend in unsere Anwendung importiert werden. Dies reduziert manuellen Erfassungsaufwand.

\subsubsection{Farbkonzept für Ereignisse}
Die Farbgebung in der Agenda folgt einem durchdachten System zur visuellen Unterscheidung verschiedener Ereignistypen (vgl. \texttt{kkoala/consts.py}, \texttt{kkoala/routes/events.py}, \texttt{kkoala/templates/settings.html}):

\myparagraph{Importierte Ereignisse (Stundenplan)}
Beim Import von \texttt{.ics}-Dateien werden alle Ereignisse automatisch mit der Standardfarbe \texttt{\#6C757D} (mittleres Grau) eingefärbt. Diese Wahl basiert auf mehreren Überlegungen:
\begin{itemize}
    \item \textbf{Visuelle Hierarchie}: Grau ist neutral und tritt gegenüber den farbcodierten Prioritätsstufen (rot, orange, grün) und Lernblöcken (blau) zurück, sodass wichtigere Ereignisse visuell dominieren.
    \item \textbf{Lesbarkeit}: \texttt{\#6C757D} bietet ausreichenden Kontrast zu weissen Hintergründen, ohne visuell dominant zu wirken.
\end{itemize}

Diese Standardfarbe kann in den Einstellungen global angepasst werden, falls Nutzer:innen persönliche Präferenzen haben. Alle bereits importierten Ereignisse übernehmen dann die neue Farbe, was eine konsistente visuelle Darstellung des gesamten Stundenplans gewährleistet.

\myparagraph{Prüfungsereignisse nach Prioritätsstufen}
Prüfungen werden basierend auf ihrer Prioritätsstufe automatisch eingefärbt, um die Dringlichkeit visuell zu kommunizieren (vgl. \texttt{kkoala/consts.py}). Die Farbwahl folgt einem Ampel-System mit abgestufter Intensität:

\begin{itemize}
    \item \textbf{Priorität 1 (Höchste Dringlichkeit): \texttt{\#770000} -- Dunkles Rot}
    \begin{itemize}
        \item Rot wird im Ampel-System für \enquote{Stopp} verwendet und signalisiert hohe Dringlichkeit.
        \item Das dunkle Rot \texttt{\#770000} ist weniger intensiv als reines Rot (\texttt{\#FF0000}), bleibt aber gut sichtbar.
    \end{itemize}
    
    \item \textbf{Priorität 2 (Mittlere Dringlichkeit): \texttt{\#ca8300} -- Orange}
    \begin{itemize}
        \item Orange liegt im Ampel-System zwischen Rot und Grün und signalisiert mittlere Dringlichkeit.
        \item Bildet eine Abstufung zwischen Priorität 1 (Rot) und Priorität 3 (Grün).
    \end{itemize}
    
    \item \textbf{Priorität 3 (Niedrigste Dringlichkeit): \texttt{\#097200} -- Dunkles Grün}
    \begin{itemize}
        \item Grün wird im Ampel-System für \enquote{Freigabe} verwendet und signalisiert niedrige Dringlichkeit.
        \item Das dunkle Grün \texttt{\#097200} bietet ausreichenden Kontrast zum weissen Hintergrund.
    \end{itemize}
\end{itemize}

Die Farbwahl orientiert sich am Ampel-System (Rot → Orange → Grün), das aus dem Verkehrswesen bekannt ist. Die Farbtöne sind abgedunkelt, um eine gleichmässige visuelle Darstellung zu gewährleisten. Alle Prioritätsfarben können in den Einstellungen angepasst werden.

\myparagraph{Algorithmisch generierte Lernblöcke}
Die vom Lernzeitalgorithmus erzeugten Lernblöcke erhalten standardmässig die Farbe \texttt{\#0000FF} (reines Blau). Diese Entscheidung wurde aus folgenden Gründen getroffen:
\begin{itemize}
    \item Blau liegt ausserhalb des Ampel-Farbspektrums (Rot-Orange-Grün) und der Stundenplanfarbe (Grau), wodurch Lernblöcke klar erkennbar sind.
    \item Reines Blau bietet hohen Kontrast zum weissen Hintergrund.
\end{itemize}

Diese Standardfarbe kann in den Einstellungen angepasst werden, falls Nutzer:innen eine andere Farbe bevorzugen (vgl. \texttt{kkoala/templates/settings.html}).

\subsubsection{Import und Export}
Wie oben beschrieben, kann der Stundenplan aus \texttt{schulnetz} im \texttt{.ics}-Format importiert werden. Zusätzlich besteht die Möglichkeit, die gesamte Agenda als \texttt{.ics}-Datei zu exportieren, um sie in anderen Kalenderanwendungen zu nutzen. Wenn man den Kalendar aus unserer Applikation exportiert und dann auch wieder in unserer Applikation importiert, bleiben Priorität und Farbe erhalten.


\begin{figure}
    \centering
    \includegraphics[width=\linewidth]{img/agenda.png}
    \caption[Beispiel einer Agenda mit Lernblöcken der LZA. Eigener Screenshot von Kanti Koala, 31.10.2025.]{Beispiel einer Agenda mit Lernblöcken der LZA}
    \label{fig:placeholder}
\end{figure}

\subsection[Der Lernzeitalgorithmus]{Der Lernzeitalgorithmus\texorpdfstring{\aifootnotemark}{}}
\footnotetext{ChatGPT (Version GPT-5): \enquote{Korriegiere Grammatik und Rechtschreibfehler im folgenden Text. [...]. }, 04.11.2025. Antwort ganz übernommen.}
Der Lernzeitalgorithmus (LZA) ist der Kern unserer Web-App. Er automatisiert die Planung der notwendigen Lernzeiten für die anstehenden Prüfungen eines Nutzers.
Wir bezeichnen unseren Mechanismus als Algorithmus, da er die formalen Kriterien eines Algorithmus erfüllt: Jeder Planungsschritt ist ausführbar (existierende Funktionen), deterministisch und determiniert (gleiche Eingabedaten führen stets zur gleichen Priorisierung und Planung). Zudem ist die Anzahl der zur Erstellung des Lernplans notwendigen Schritte endlich (Finitheit), wodurch der Mechanismus garantiert terminiert und eine strukturierte Ausgabe (den Lernplan) basierend auf den Eingabedaten (Benutzereinstellungen und Prüfungstermine) liefert \parencite{studyflix_algorithmus}.

\subsubsection{Eingabeparameter und Planungsziel}
Der Lernzeitalgorithmus (LZA) verwendet globale Benutzereinstellungen sowie prüfungsspezifische Prioritätseinstellungen als Eingabeparameter, um eine regelbasierte Lernplanung zu ermöglichen.

Das zentrale Planungsziel des LZA ist es, die definierten totalen Lernstunden für jede Prüfung innerhalb des gültigen Lernfensters zu erreichen, während das tägliche Maximum und alle bestehenden Kalenderkonflikte des Nutzers strikt eingehalten werden.

Wir haben die Eingabeparameter so gewählt, weil wir sie so aus der Recherche als wichtig erachteten, um eine gute Lernplanung zu ermöglichen. 

\myparagraph{Globale Parameter}
Diese Einstellungen gelten für den gesamten Planungszeitraum:
\begin{itemize}
    \item \textbf{Lernen am Samstag}: Definiert, ob der Algorithmus Lernblöcke an Samstagen planen darf.
    \item \textbf{Lernen am Sonntag}: Definiert, ob der Algorithmus Lernblöcke an Sonntagen planen darf.
    \item \textbf{Bevorzugte Lernzeit}: Die bevorzugte Uhrzeit am Tag, zu der die Platzierung von Lernblöcken primär angestrebt wird.
\end{itemize}

\myparagraph{Prüfungsspezifische Parameter (Pro Prioritätstufe)}
Diese Werte werden basierend auf der Priorität jeder Prüfung zugewiesen:
\begin{itemize}
    \item \textbf{Tägliches Maximum}: Die maximale Stundenzahl, die pro Tag für Prüfungen dieser Priorität geplant werden darf.
    \item \textbf{Total Lernstunden}: Die gesamte Anzahl an Lernstunden, die für Prüfungen dieser Priorität absolviert werden muss.
\end{itemize}

\subsubsection{Ablauf und Planungsstrategie}
Der Algorithmus arbeitet iterativ und bearbeitet alle als Prüfung markierten Ereignisse in aufsteigender Reihenfolge ihrer Priorität. Eine niedrigere Prioritätsnummer kennzeichnet dabei eine höhere Wichtigkeit.

\textbf{Hinweis zum Umgang mit fehlenden Endzeiten:} 
Falls ein Ereignis keine Endzeit besitzt, behandelt der Algorithmus dieses Ereignis als sogenanntes \emph{Punkt-Ereignis}, d.\,h. das Start- und Endzeitpunkt sind identisch. Dies ist insbesondere relevant, da importierte Kalenderdaten manchmal keine Endzeit angeben. Eine Korrektur dieses Verhaltens bereits beim Import könnte zu Darstellungsproblemen in der Kalenderansicht führen. Daher wurde entschieden, diese Anpassung erst im Algorithmus selbst vorzunehmen, um sowohl die Integrität der Planung als auch die Kompatibilität mit der Kalenderdarstellung zu gewährleisten.

\begin{enumerate}
    \item \textbf{Zyklische Neuberechnung der Anforderungen (Recycling)}:
    \begin{itemize}
        \item \textbf{Flexibilitätsbereinigung}: Alle vom System selbst geplanten, aber nicht gesperrten Lernblöcke für die aktuelle Prüfung werden gelöscht. Dies ermöglicht eine Neuplanung, falls sich die Rahmenbedingungen geändert haben.
        \item \textbf{Soll-Stunden-Ermittlung}: Die noch zu erbringende Lernzeit wird neu berechnet unter Berücksichtigung bereits absolvierter oder manuell gesperrter Stunden.
    \end{itemize}
    \item \textbf{Rückwärts-Iterative Planung}:
    \begin{itemize}
        \item Die Planungsstrategie beginnt beim Prüfungstermin und arbeitet sich tageweise, jedoch maximal drei Wochen, bis zum aktuellen Datum vor. Dies stellt sicher, dass Lernblöcke mit höchster Dringlichkeit (nächste zur Prüfung) zuerst belegt werden.
        \item An jedem Tag wird die maximale Lernzeit für diese spezifische Prüfung ermittelt, um das tägliche Zeitlimit einzuhalten.
    \end{itemize}
    \item \textbf{Platzierung und strikte Konfliktvermeidung}:
    \begin{itemize}
        \item \textbf{Bevorzugter Slot}: Primär wird versucht, einen Lernblock in der vom Nutzer festgelegten bevorzugten Lernzeit zu platzieren.
        \item \textbf{Konfliktprüfung}: Die Verfügbarkeit wird gegen alle Kalendereinträge geprüft. Ein 30-minütiger Puffer vor und nach jedem Ereignis verhindert zeitliche Überlappungen.
        \item \textbf{Alternative Slots}: Falls die bevorzugte Zeit belegt ist, sucht eine dedizierte Funktion den grössten verfügbaren, konfliktfreien Zeitabschnitt.
        \item \textbf{Echtzeit-Aktualisierung}: Nach erfolgreicher Generierung wird der Lernblock sofort zur Liste der aktuellen Ereignisse hinzugefügt, um nachfolgende Überlappungen auszuschliessen.
    \end{itemize}
             \item \textbf{Ergebnisrückgabe und Zusammenfassung}:
    \begin{itemize}
        \item Der Algorithmus gibt eine detaillierte Zusammenfassung zurück: Gesamtzahl hinzugefügter Lernblöcke, geplante Gesamtstunden und Planungsstatus pro Prüfung.
    \end{itemize}
\end{enumerate}

\subsection{Notenverwaltung}
In der Kantonsschule muss man immer wieder Prüfungen schreiben. Die Noten, die man erhält, sind dann wichtig für die Promotion in die nächste Stufe. Deswegen haben wir ein Feature, in dem man seine Noten pro Semester speichern kann. Jedes Semester hat schon die jeweiligen Fächer, die man dann in diesem Semester haben würde, geladen. Man kann natürlich immer noch Fächer löschen und hinzufügen. Man kann mit diesem Feature dann seine Durchschnitte pro Fach und Semester sehen. Ebenfalls kann man mit dem Notenrechner sehen, welche Note man in einem Fach brauchen würde, um einen bestimmten Schnitt in diesem Fach zu haben. Die Idee von diesem Feature stammt aus der App \enquote{Pluspoints} \parencite{pluspoints_app}. Das Feature, das uns dort fehlte, war die automatische Fächeraddition pro Semester, welches wir in unserer App implementiert haben.

\begin{figure}[!htp]
    \centering
    \includegraphics[width=\linewidth]{img/notenorg-screenshot.png}
    \caption[Screenshot der Notenverwaltungs-Seite. Eigener Screenshot von Kanti Koala, 08.11.2025.]{Screenshot der Notenverwaltungs-Seite}
    \label{fig:notenverwaltung_page}
\end{figure}

\subsection{Lerntimer}
Ein weiteres wichtiges Feature unserer App ist der Lerntimer. Dieser Timer basiert auf der Pomodoro-Technik, welche im Recherche Teil dieses Berichts erklärt wurde. Mit der Recherche, die wir führten, fanden wir es wichtig, einen solchen Lerntimer zu implementieren, da es eine sehr effektive Lerntechnik ist. Der Timer hat die Standard-Einstellungen von 25 Minuten Lernen und 5 Minuten Pause, welche man aber auch ändern kann, falls man das möchte.

\begin{figure}[!htp]
    \centering
    \includegraphics[width=\linewidth]{img/lerntimer-screenshot.png}
    \caption[Screenshot der Lerntimer-Seite. Eigener Screenshot von Kanti Koala, 08.11.2025.]{Screenshot der Lerntimer-Seite}
    \label{fig:lerntimer_page}
\end{figure}

\subsection{Daily Tipps}
Ein relativ wichtiges Feature unserer App sind die Daily Tipps. Es sollte jeden Tag ein neuer Tipp an den Nutzer gezeigt werden auf der Startseite, entweder über die Kantonsschule oder allgemeine Lerntipps. Die einfachste Möglichkeit, dies zu implementieren, ist eine simple Modulo-Rechnung:
\begin{equation}
\text{Tipp des Tages (Index)} = (\text{Tag des Jahres}) \bmod (\text{Anzahl der Tipps})
\end{equation}
Somit wird an einem bestimmten Tag nur ein Tipp gezeigt und über das ganze Jahr sollten alle Tipps in einem Zyklus gezeigt werden.

Der Titel dieses Features ist auf Englisch (\enquote{Tip of the Day}), weil der Ausdruck als prägnanter UI-Anker wiedererkennbar ist; der eigentliche Inhalt (der Tipp) bleibt vollständig auf Deutsch, sodass Verständlichkeit und lokaler Bezug erhalten bleiben.

\subsection{Lerntipps}
Ein weiteres Feature unserer App sind die Lerntipps. Diese sind in verschiedene Kategorien aufgeteilt, u. a. Zeitmanagement, allgemeine Tipps der Kanti Baden, Stressmanagement und Lernmethoden. In jeder Kategorie gibt es verschiedene Tipps, welche wir aus unserer Recherche und den Interviews gesammelt haben. Diese Tipps sollen den Nutzer:innen helfen, ihr Lernverhalten zu verbessern.
Beim Design für dieses Feature haben wir uns entschieden, dass die Bullet-Points der Tipps ein kleines Koala-Symbol sein sollen, um das Thema der App widerzuspiegeln. Der Koala wurde von uns selber entworfen.
Wir haben auch ein Scroll-To-Top Button implementiert, da wir von einer Rückmeldung von den Usability Tests die Rückmeldung bekamen, dass es ein gewünschtes Feature sei, um schnell wieder nach oben zu gelangen.

\begin{figure}[!htp]
    \centering
    \includegraphics[width=0.2\linewidth]{img/koala-lerntipps-cropped.png}
    \caption[Koala Bullet Point für Lerntipps. Eigene Darstellung, 06.11.2025.]{Koala Bullet Point für Lerntipps}
    \label{fig:koala_bullet_point}
\end{figure}

% Screenshot of the lerntipps page
\begin{figure}[!htp]
    \centering
    \includegraphics[width=\linewidth]{img/lerntipps-screenshot.png}
    \caption[Screenshot der Lerntipps-Seite. Eigener Screenshot von Kanti Koala, 08.11.2025.]{Screenshot der Lerntipps-Seite}
    \label{fig:lerntipps_page}
\end{figure}

\subsection{Einstellungen}
In den Einstellungen kann der Nutzer verschiedene globale Einstellungen für die App vornehmen, wie zum Beispiel das Farbschema der App (hell oder dunkel), die bevorzugte Lernzeit, ob am Wochenende gelernt werden soll und die Farben der Prioritätsstufen. Diese Einstellungen werden in der Datenbank gespeichert und beim Laden der App abgerufen, um eine personalisierte Erfahrung zu bieten. Zusätzlich kann man hier auch sein Konto löschen und sein Passwort ändern.

\begin{figure}[!htp]
    \centering
    \includegraphics[width=\linewidth]{img/einstellungen-screenshot.png}
    \caption[Screenshot der Einstellungen-Seite. Eigener Screenshot von Kanti Koala, 08.11.2025.]{Screenshot der Einstellungen-Seite}
    \label{fig:settings_page}
\end{figure}


\section[Sicherheitskonzept]{Sicherheitskonzept\texorpdfstring{\aifootnotemark}{}}
\footnotetext{ChatGPT (Version GPT-5): \enquote{Korriegiere Grammatik und Rechtschreibfehler im folgenden Text. [...]. }, 04.11.2025. Antwort ganz übernommen.}
Neben der reinen Authentifizierung wurden weitere grundlegende Sicherheitsmassnahmen implementiert, um die Daten der Nutzer und die Integrität der Anwendung zu schützen.

\subsection[Datenspeicherung und Passwort-Sicherheit]{Datenspeicherung und Passwort-Sicherheit\texorpdfstring{\aifootnotemark}{}}
\footnotetext{ChatGPT (Version GPT-5): \enquote{Korriegiere Grammatik und Rechtschreibfehler im folgenden Text. [...]. }, 04.11.2025. Antwort ganz übernommen.}
Die Sicherheit der Benutzerdaten hat höchste Priorität. Wie im Abschnitt zur Authentifizierung beschrieben, werden Passwörter niemals im Klartext gespeichert. Stattdessen wird die Flask-Bcrypt-Bibliothek verwendet, um von jedem Passwort einen kryptografischen Hash zu erzeugen. Beim Login-Vorgang wird das eingegebene Passwort ebenfalls gehasht und dieser Hash wird mit dem in der Datenbank gespeicherten Hash verglichen. Da dieser Prozess unumkehrbar ist, kann selbst bei einem direkten Zugriff auf die Datenbank das ursprüngliche Passwort nicht wiederhergestellt werden.

\subsection[Transportverschlüsselung (HTTPS)]{Transportverschlüsselung (HTTPS)\texorpdfstring{\aifootnotemark}{}}
\footnotetext{ChatGPT (Version GPT-5): \enquote{Korriegiere Grammatik und Rechtschreibfehler im folgenden Text. [...]. }, 04.11.2025. Antwort ganz übernommen.}
Die gesamte Kommunikation zwischen dem Browser des Nutzers und unserem Server wird durch das HTTPS-Protokoll verschlüsselt. Dies wird durch ein SSL/TLS-Zertifikat realisiert, das auf unserem Server bei DigitalOcean installiert ist. Die Verschlüsselung stellt sicher, dass alle übertragenen Daten, von Login-Informationen über Kalendereinträge bis hin zu Noten, vor dem Abhören durch Dritte geschützt sind. Ein Angreifer in einem öffentlichen WLAN könnte beispielsweise die Daten nicht mitlesen. Der Browser zeigt dies durch ein Schlosssymbol in der Adressleiste an und stellt so eine verschlüsselte Verbindung zur Domain \texttt{kantikoala.app} sicher \parencite{cloudflare_https}.

\subsection[CSRF-Schutz]{CSRF-Schutz\texorpdfstring{\aifootnotemark}{}}
\footnotetext{ChatGPT (Version GPT-5): \enquote{Korriegiere Grammatik und Rechtschreibfehler im folgenden Text. [...]. }, 04.11.2025. Antwort ganz übernommen.}
Neben der reinen Authentifizierung ist es entscheidend, die Aktionen eines angemeldeten Benutzers abzusichern. Eine häufige Schwachstelle in Webanwendungen ist die Cross-Site Request Forgery (CSRF) \parencite{rapid7_csrf}. Bei einem CSRF-Angriff bringt ein Angreifer den Browser eines authentifizierten Benutzers dazu, eine unerwünschte Aktion in einer Webanwendung auszuführen, bei der der Benutzer gerade angemeldet ist. Dies geschieht, ohne dass der Benutzer es merkt. So könnte ein Angreifer beispielsweise einen Benutzer dazu verleiten, auf einen bösartigen Link zu klicken, der im Hintergrund unbemerkt das Passwort des Benutzers ändert oder sein Konto löscht \parencite{TestDriven_CSRF}.

Um dies zu verhindern, haben wir das \enquote{Synchronizer Token Pattern} implementiert, eine von \textcite{OWASP_CSRF} empfohlene Methode. Das Prinzip ist einfach, aber sehr wirksam:
\begin{enumerate}
    \item Für jede Benutzersitzung wird ein einzigartiges, geheimes und unvorhersehbares Token generiert und auf dem Server gespeichert.
    \item Dieses Token wird in alle Formulare, die eine Zustandsänderung bewirken (z.B. das Ändern von Einstellungen oder das Löschen eines Kontos), als verstecktes Feld eingebettet.
    \item Wenn der Benutzer das Formular abschickt, wird das Token zusammen mit den anderen Formulardaten an den Server gesendet.
    \item Der Server vergleicht das vom Client gesendete Token mit dem in der Sitzung gespeicherten Token. Stimmen die beiden nicht überein, wird die Anfrage abgelehnt.
\end{enumerate}
Da ein Angreifer auf einer fremden Website dieses geheime Token nicht kennen kann, schlägt der Fälschungsversuch fehl.

In unserer Flask-Anwendung haben wir diese Logik mithilfe eines eigenen Decorators (\texttt{@csrf\_protect}) umgesetzt. Dieser Decorator wird auf alle Routen angewendet, die Daten durch \texttt{POST}-, \texttt{PUT}- oder \texttt{DELETE}-Anfragen ändern. Bei Standard-HTML-Formularen wird das Token als verstecktes \texttt{<input>}-Feld übergeben. Für unsere dynamischen Agenda-Funktionen, die auf AJAX basieren, wird das Token aus einem Meta-Tag ausgelesen und in einem benutzerdefinierten HTTP-Header (\texttt{X-CSRF-Token}) mit jeder Anfrage gesendet. Dies stellt sicher, dass jede datenverändernde Aktion, die in unserer Applikation ausgeführt wird, legitim vom Benutzer und von unserer eigenen Webseite stammt.

Eine wichtige Anmerkung ist, dass es schon Module wie \texttt{Flask-WTF} gibt, die CSRF-Schutz bieten. Wir haben uns jedoch entschieden, unseren eigenen Decorator zu schreiben, um ein tieferes Verständnis für die Funktionsweise von CSRF-Schutzmechanismen zu erlangen.

\section{Design und Benutzeroberfläche}
Das Design der Web-App haben wir mit der CSS-Bibliothek Tailwind CSS umgesetzt \parencite{tailwind_css}. Tailwind ermöglicht es uns, schnell und konsistent ansprechende Oberflächen zu gestalten, indem wir vordefinierte Utility-Klassen verwenden. Dies erleichtert die Umsetzung eines einheitlichen Designs und beschleunigt den Entwicklungsprozess erheblich.

\subsection{Farbkonzept}
Die Farbwahl folgt praktischen Erwägungen: gut unterscheidbare Zustände, ruhige Grundflächen und konsistente Akzente. Manche Farbwahlen wurden schon vorher im Abschnitt zur Agenda erläutert. Hier eine Übersicht des generellen Farbkonzepts:
\begin{itemize}
  \item Neutrale Hintergründe (hell/dunkel): dezente Grau-/Zinc-Töne als Bühne für Inhalte.
  \item Prioritäten in der Agenda: abgestufte Rot–Orange–Grün-Töne für eine nachvollziehbare Dringlichkeitsskala.
  \item Lernblöcke: Blau ausserhalb des Rot–Grün-Spektrums zur klaren Abgrenzung.
  \item Importierte Stundenplan-Elemente: neutrales Grau, damit sie Informationen liefern ohne zu dominieren.
  \item Rückmeldungen: Grün für erfolgreiche Aktionen, Rot für Fehlerzustände.
  \item Akzente: Primärfarbe Blau und Grün für interaktive Elemente (Buttons, Links) zur Hervorhebung.
\end{itemize}
Die Töne sind bewusst leicht gedämpft gewählt, um eine ruhige, einheitliche Anmutung über helle und dunkle Oberflächen zu erhalten.

\subsection{Dark Mode}
Wir haben die Rückmeldung aus den Usability-Tests aufgenommen und einen wechselbaren Dark Mode (das heisst, er wird nicht automatisch aktiviert, sondern man kann ihn an- und ausschalten) implementiert. Nutzer:innen können in den Einstellungen zwischen Light Mode, Dark Mode und automatischem Modus (abhängig von Systemeinstellungen) wählen. Das Design wurde so angepasst, dass alle Elemente in beiden Modi gut lesbar und ansprechend sind. Tailwind CSS erleichtert die Umsetzung durch integrierte Dark-Mode-Klassen. Flask kann mit einem Content Processor so konfiguriert werden, dass jede Seite die gebrauchte Einstellung erhält.

\subsection{Fehlerseiten (404/500): Gestaltung und Zweck}
\textbf{404} signalisiert, dass eine angeforderte Ressource nicht existiert; \textbf{500} steht für einen internen Fehler.
Die Seiten sind reduziert gestaltet:
\begin{itemize}
  \item Zentrale Karte mit Statuscode, kurzer Erklärung und primärer Aktion (zurück zur Startseite).
  \item Konsistentes Branding (Logo, Basisfarben) für Wiedererkennung.
  \item Keine technischen Details; klare Orientierung statt Verunsicherung.
\end{itemize}
Ziel ist, Nutzer:innen schnell zurück in einen funktionierenden Kontext zu führen und gleichzeitig sensible Interna nicht offenzulegen.

\begin{figure}[!htp]
    \centering
    \includegraphics[width=0.6\linewidth]{img/404-page.png}
    \caption[Screenshot der 404-Fehlerseite. Eigener Screenshot von Kanti Koala, 08.11.2025.]{Screenshot der 404-Fehlerseite}
    \label{fig:404_page}
\end{figure}

% ------------------ keep existing structure below ------------------

\section{Tests}
Um die Qualität und Zuverlässigkeit der Kanti Koala Web-App sicherzustellen, wurden verschiedene Testfälle (funktional + Usability) durchgeführt. Die funktionalen Tests decken Backend-Logik und Frontend-Funktionalität ab (Authentifizierung, Agenda, Notenverwaltung, Lerntools, UI). Der formative Usability-Test evaluierte Bedienlichkeit, Verständlichkeit und Fehlertoleranz bei der App. Die vollständigen Testprotokolle der funktionalen Tests sind im Anhang dokumentiert.

\subsection{Usability-Test}
Um die Anonymität der Tester:innen zu gewährleisten, lassen wir die Namen weg. Insgesamt nahmen drei Personen am Test teil, alle Schüler:innen der Kantonsschule. Die Tester:innen wurden gebeten, verschiedene Aufgaben in der App zu erledigen und danach Feedback zu geben. 

\subsubsection{Testbericht Person 1 -- 08.11.2025}
\textbf{Pros:}
\begin{itemize}
\item Alles an einem Ort, was ich für mein Lernen brauche. Ich muss nicht 100 Tabs offen haben.
\item Ein schönes User-Interface.
\end{itemize}

\textbf{Cons:}
\begin{itemize}
    \item Es fehlt mir eine Introduktion in die App. So eine Anleitung, wie man es benutzt und was man machen kann.
\end{itemize}

\subsubsection{Testbericht Person 2 -- 08.11.2025}
\textbf{Pros:}
\begin{itemize}
    \item Simples und intuitives Design.
    \item Design ist sehr schön.
\end{itemize}
\textbf{Cons:}
\begin{itemize}
    \item Ich fände es cool, zwischen Light und Dark Mode wechseln zu können, anstatt dass es nur automatisch geht.
    \item Ich fände es cool, wenn ihr irgendwie ein Bild von euch in der \enquote{Über Uns} Page hättet.
    \item Ich würde auch gerne ein Scroll-To-Top Button bei den Lerntipps haben.
    \item Mir ist es unklar, wieso es bei den Lerntipps ein Section \enquote{Lernen} und dann noch \enquote{Lernmethoden} gibt. Es fühlt sich wie das Gleiche an.
\end{itemize}

% Schlussfolgerung und Ausblick
%\clearpage
\chapter[Schlussfolgerung und Ausblick]{Schlussfolgerung und Ausblick\texorpdfstring{\aifootnotemark}{}}
\aifootnotetext{06.11.2025}
\section*{Zusammenfassung}
% Folgende Linie zu überarbeiten:
Ausgehend von der in der Einleitung formulierten Zielsetzung -- Übersicht über Termine/Aufgaben/Lernzeiten schaffen und regelmässige, abgegrenzte Lernphasen unterstützen -- wurde ein prototypischer Web-Ansatz umgesetzt. Die zuvor durch Literaturstudie (Lern-, Zeit-, Pausen-, Stressaspekte), Interviews mit PPP-Lehrpersonen und eine Schüler:innen-Umfrage gewonnenen Erkenntnisse dienten der Ableitung der funktionalen Kernanforderungen (zentrale Agenda, automatische Lernblock-Erzeugung, strukturierte Zeitsegmente, einfache Notenerfassung, Basis-Schutzmassnahmen).

Umgesetzt wurden:
\begin{itemize}
  \item \textbf{Agenda} als zentrales Organisationsinstrument mit manueller Event-Erfassung, Import von \texttt{.ics}-Stundenplänen und Prioritätsattributen als Grundlage für Lernblöcke (reduziert Mehrfacherfassung).
  \item \textbf{Lernzeitalgorithmus} mit rückwärtsgerichteter, konfliktvermeidender Platzierung von Lernblöcken innerhalb täglicher Obergrenzen und gesetzter Prioritäten (unterstützt regelmässige, zeitlich definierte Phasen).
  \item \textbf{Notenverwaltung} (Semester-Fach-Note) zur strukturierten Ablage schulischer Leistungsdaten (fördert konsistente Dokumentation).
  \item \textbf{Pomodoro-basierter Lerntimer} zur zeitlichen Segmentierung von Lernintervallen und Pausen (Rahmen für wiederholbare Lerngewohnheiten).
  \item \textbf{Tägliche Tipps} aus recherchierten Bereichen (Zeit-, Pausen-, Stressmanagement, Lernmethoden) als ergänzende Impulse.
  \item \textbf{Authentifizierung} (Registrierung, Login, Passwort-Reset) mit gehashten/gesalzten Passwörtern (bcrypt) zur sicheren Zugangstrennung.
  \item \textbf{Basis-Sicherheitsmassnahmen} (CSRF-Token, HTTPS-Kommunikation) zur Integrität von Sitzungsaktionen und Schutz der Transportstrecke.
\end{itemize}
Damit liegt ein kohärenter Prototyp vor, der die in der Einleitung definierten organisatorischen und lerngewohnheitsbezogenen Zielkomponenten funktional realisiert; Aussagen zur tatsächlichen Wirkung im Nutzungskontext stehen aus.

\section*{Bilanz zur Fragestellung}
Die Hauptfragestellung (Erleichterung von Organisation und Unterstützung regelmässiger Lerngewohnheiten) wurde technisch adressiert:
\begin{itemize}
  \item \textbf{Organisation erleichtern}: Zentrale Erfassung von Ereignissen, algorithmische Erstellung von Lernblöcken, Prioritätssteuerung und .ics-Import reduzieren manuellen Planungsaufwand (keine Mehrfacherfassung, automatische zeitliche Segmentierung vom Algorithmus).
  \item \textbf{Unterstützung von Lerngewohnheiten}: Strukturierte Lernblöcke (mit Start-/Endzeiten) und konfigurierbarer Lerntimer bilden wiederkehrende Einheiten ab; Lerntipps liefern ergänzende Handlungsimpulse.
\end{itemize}
Nicht beantwortet ist die weitergehende empirische Wirkung (Langzeitnutzung, Lernverhaltensänderung); die Bedienbarkeit wurde jedoch formativ geprüft und Basisakzeptanz bestätigt. Wir dachten uns, dass eine langfristige Evaluation nur über ein Semester Sinn machen würde und im Rahmen einer Maturitätsarbeit nicht realisierbar ist.

\section*{Erkenntnisse aus dem Arbeitsprozess}
\begin{itemize}
  \item Früh definierte Datenbeziehungen erleichtern Erweiterungen (z.B. zusätzliche Einstellungen).
  \item Iterative Anpassung des Lernzeitalgorithmus notwendig (Rückwärtsplanung erwies sich zur Priorisierung naher Prüfungen als praktikabel; Konfliktpuffer reduzierten Engstellen).
  \item Klare Prioritätsparameter (tägliches Maximum, Gesamtstunden) verhindern Überlastung, erforderten aber konsistente Validierung. % ?
  \item Modularisierung (Blueprints, \texttt{extensions.py}, Factory) vereinfachte die Übersichtlichkeit und spätere Anpassungen.
  \item Sicherheitsthemen (CSRF, Passwort-Hashing) müssen früh integriert werden, um Nacharbeiten zu vermeiden.
  \item Transkription in Mundart führte zu hohem Aufwand; Standardisierung der Interviewsprache wäre effizienter.
\end{itemize}

\section*{Ausblick und offene Punkte}
\begin{itemize}
  \item Vertiefte Wirkungsstudie (Langzeit-Logdaten, Lernblock-Nutzung).
  \item Erweiterte Analytics (Verhältnis geplante vs. absolvierte Lernblöcke).
  \item Dynamische Prioritätsanpassung nach Notenfortschritt.
  \item Native App-Evaluierung (Offline, Push-Benachrichtigungen).
\end{itemize}

\section*{Fazit}
Der Prototyp schafft eine funktionale Grundlage zur strukturierten Erfassung und Planung von schulischen Aktivitäten und Lernzeiten. Die technische Machbarkeit der Kernfunktionen ist gezeigt; die tatsächliche Wirkung auf Lernorganisation und Gewohnheiten bleibt Gegenstand zukünftiger Evaluation. Damit liegt eine belastbare Ausgangsbasis für eine empirische Validierungs- und Erweiterungsphase vor.


% References

%\addcontentsline{toc}{chapter}{Bibliography}
%\nocite{*}
%\printbibliography
\clearpage
\addcontentsline{toc}{chapter}{Literaturverzeichnis} % Changed from Bibliography/Quellen
\nocite{*}
\printbibliography[title={Literaturverzeichnis}]

\clearpage
\listoffigures
\addcontentsline{toc}{chapter}{Abbildungsverzeichnis}

\clearpage
\chapter*{Anhang}
\addcontentsline{toc}{chapter}{Anhang}
\begin{itemize}
    \item \textbf{Code:} Der vollständige Code der Kanti Koala Web-App ist auf GitHub verfügbar unter: \url{https://github.com/CoderAryanAnand/lernapp}.\\
                         Der Code wird aber auch noch separat beigefügt.
    \item \textbf{KI-Nachweis}
    \item \textbf{Tests}
    \item \textbf{Unsere Umfrage}
    \item \textbf{Interviewfragebogen}
    \item \textbf{Interviewtranskripte}
    \item \textbf{Interview- und Umfrageanalysen}
\end{itemize}


\end{document}
