\documentclass[11pt, a4paper]{article}

% --- UNIVERSAL PREAMBLE BLOCK ---
\usepackage[a4paper, top=2.5cm, bottom=2.5cm, left=2cm, right=2cm]{geometry}

% Main language is German
\usepackage[german, provide=*]{babel}

\babelprovide[import, onchar=ids fonts]{german}
\babelprovide[import, onchar=ids fonts]{english}

\usepackage{csquotes}

% Set default/Latin font to Sans Serif in the main (rm) slot

% Fix lists for non-English main language
\usepackage{enumitem}

% Packages for structure and tables
\usepackage{array}
\usepackage{longtable}
\usepackage{booktabs}
\usepackage{xcolor} % For colored status text
\usepackage{hyperref} % Keep this last

% ---------------------------------

% Define colors for status
\definecolor{passcolor}{rgb}{0.1, 0.6, 0.1}
\definecolor{failcolor}{rgb}{0.8, 0.1, 0.1}

\newcommand{\aifootnotemark}{\footnotemark}
\newcommand{\aifootnotetext}{\footnotetext{Gemini (Model 2.5 Pro): \enquote{Korrigiere Fehler in den Test-Cases, und verbessere die Sprache [...]. }, 23.10.2025. Antwort ganz übernommen.}}
\newcommand{\aifootnoteintro}{\footnotetext{Gemini (Model 2.5 Pro): \enquote{Korrigiere Grammatik und Rechtschreibfehler, und verbessere die Sprache [...]. }, 23.10.2025. Antwort ganz übernommen.}}


% Define a new environment for Test Cases
\newenvironment{testcase}[1]{%
    \par\vspace{1em}\noindent
    \textbf{Testfall ID:} #1 \\
    \vspace{0.2em}
}{\par\vspace{1em}}

\title{\textbf{Manuelle Testspezifikation: KantiKoala Studienhilfsapp}}
\author{Dokumentation zur Maturaarbeit}
\date{11. November 2025}

\begin{document}
\maketitle
\thispagestyle{empty}

\section[Einführung]{Einführung\texorpdfstring{\aifootnotemark}{}}
\aifootnoteintro
Dieses Dokument dient als Protokoll für die manuelle Qualitätssicherung der KantiKoala Webanwendung. Der Testplan ist darauf ausgerichtet, die Kernfunktionen der App zu validieren, darunter Benutzer-Authentifizierung, Notenverwaltung, Kalenderfunktionen, Lern-Tools, To-Do Listen, Informationsseiten und die User Interface. Jeder Testfall ist klar definiert mit Ziel, Schritten zur Durchführung, erwarteten Resultaten und dem Status (Bestanden/Nicht Bestanden).

\section[Abdeckung der Bewertungskriterien (Produkt)]{Abdeckung der Bewertungskriterien (Produkt)\texorpdfstring{\aifootnotemark}{}}
\aifootnoteintro
Die manuelle Testspezifikation zielt primär auf die Überprüfung der folgenden Bewertungskriterien ab:

\begin{itemize}
    \item \textbf{Funktionalität (Vollständigkeit und Korrektheit)}: Überprüfung, ob alle implementierten Features (Noten, Kalender, Timer, Login, etc.) wie erwartet funktionieren.
    \item \textbf{Ästhetik (Gestaltung, Layout, Typografie)}: Visuelle Überprüfung der Konsistenz, Responsive Design und ansprechenden Darstellung auf verschiedenen Geräten.
\end{itemize}

\section{Manuelle Testfälle (Testprotokoll)}

Die Testfälle sind nach funktionalen Bereichen gruppiert.

\subsection[Authentifizierung und Benutzerverwaltung]{Authentifizierung und Benutzerverwaltung\texorpdfstring{\aifootnotemark}{}}
\aifootnotetext

\begin{testcase}{AUTH-01}
    \textbf{Ziel:} Erfolgreiche Registrierung eines neuen Benutzers und anschliessender Login.\\
    \textbf{Schritte:}
    \begin{enumerate}[label=\arabic*.]
        \item Navigieren Sie zur Registrierungsseite.
        \item Geben Sie eine gültige E-Mail-Adresse, einen Benutzernamen und ein sicheres Passwort ein.
        \item Klicken Sie auf den Registrierungsbutton.
        \item Melden Sie sich mit den neuen Zugangsdaten an.
    \end{enumerate}
    \textbf{Erwartetes Resultat:} Registrierung wird bestätigt, Benutzer wird zur Login-Seite weitergeleitet. Login ist erfolgreich, und der Benutzer wird zum Dashboard/Home-Screen umgeleitet.\\
    \vspace{0.5em}\textbf{Status:} \textcolor{passcolor}{Bestanden}
\end{testcase}

\begin{testcase}{AUTH-02}
    \textbf{Ziel:} Korrekte Handhabung von ungültigen Login-Versuchen.\\
    \textbf{Schritte:}
    \begin{enumerate}[label=\arabic*.]
        \item Versuchen Sie sich mit einem unbekannten Benutzernamen und einem Passwort anzumelden.
        \item Versuchen Sie sich mit korrektem Benutzernamen, aber falschem Passwort anzumelden.
    \end{enumerate}
    \textbf{Erwartetes Resultat:} In beiden Fällen wird eine klare, nicht diskriminierende Fehlermeldung angezeigt.\\
    \vspace{0.5em}\textbf{Status:} \textcolor{passcolor}{Bestanden}
\end{testcase}

\begin{testcase}{AUTH-03}
    \textbf{Ziel:} Abmelden (Logout) funktioniert korrekt.\\
    \textbf{Schritte:}
    \begin{enumerate}[label=\arabic*.]
        \item Melden Sie sich erfolgreich an.
        \item Klicken Sie auf den "Abmelden"-Link oder -Button.
        \item Versuchen Sie, über die URL-Eingabe auf eine geschützte Seite (z.B. \texttt{/agenda}) zuzugreifen.
    \end{enumerate}
    \textbf{Erwartetes Resultat:} Der Benutzer wird sofort zur Login-Seite weitergeleitet. Der Zugriff auf geschützte Seiten ist nicht mehr möglich, sondern wird zur Login-Seite umgeleitet.\\
    \vspace{0.5em}\textbf{Status:} \textcolor{passcolor}{Bestanden}
\end{testcase}

\begin{testcase}{AUTH-04}
    \textbf{Ziel:} Ein Benutzer kann seine Profileinstellungen (z.B. Prioritätseinstellungen für die Agenda) ändern.\\
    \textbf{Schritte:}
    \begin{enumerate}[label=\arabic*.]
        \item Melden Sie sich an und navigieren Sie zur Einstellungsseite (\texttt{/settings}).
        \item Ändern Sie die Prioritätseinstellungen für die Agenda.
        \item Speichern Sie die Änderungen.
        \item Überprüfen Sie, ob die neuen Prioritätseinstellungen in der Agenda beim Erstellen eines neuen Termins korrekt wiedergegeben werden.
    \end{enumerate}
    \textbf{Erwartetes Resultat:} Die Änderungen werden erfolgreich gespeichert, und die neuen Prioritätseinstellungen werden in der Agenda beim Erstellen eines neuen Termins korrekt wiedergegeben.\\
    \vspace{0.5em}\textbf{Status:} \textcolor{passcolor}{Bestanden}
\end{testcase}

\begin{testcase}{AUTH-05}
    \textbf{Ziel:} Ein eingeloggter Benutzer kann sein Passwort sicher ändern.\\
    \textbf{Schritte:}
    \begin{enumerate}[label=\arabic*.]
        \item Navigieren Sie zur Einstellungsseite und dort zum Abschnitt "Passwort ändern".
        \item Geben Sie das aktuelle Passwort korrekt ein.
        \item Geben Sie ein neues, sicheres Passwort ein und bestätigen Sie es.
        \item Melden Sie sich ab und versuchen Sie, sich mit dem neuen Passwort anzumelden.
    \end{enumerate}
    \textbf{Erwartetes Resultat:} Die Passwortänderung wird bestätigt. Der Login mit dem alten Passwort schlägt fehl, während der Login mit dem neuen Passwort erfolgreich ist.\\
    \vspace{0.5em}\textbf{Status:} \textcolor{passcolor}{Bestanden}
\end{testcase}

\begin{testcase}{AUTH-06}
    \textbf{Ziel:} Ein Benutzer kann sein Passwort über die "Passwort vergessen"-Funktion zurücksetzen.\\
    \textbf{Schritte:}
    \begin{enumerate}[label=\arabic*.]
        \item Navigieren Sie zur Login-Seite und klicken Sie auf "Passwort vergessen?".
        \item Geben Sie die mit dem Konto verknüpfte E-Mail-Adresse ein.
        \item Folgen Sie dem Link in der E-Mail zum Zurücksetzen des Passworts.
        \item Geben Sie ein neues Passwort ein.
    \end{enumerate}
    \textbf{Erwartetes Resultat:} Der Benutzer erhält eine E-Mail mit einem gültigen Link. Nach Eingabe eines neuen Passworts kann sich der Benutzer erfolgreich damit anmelden.\\
    \vspace{0.5em}\textbf{Status:} \textcolor{passcolor}{Bestanden}
\end{testcase}

\begin{testcase}{AUTH-07}
    \textbf{Ziel:} Verhinderung doppelter Registrierung mit identischem Benutzernamen oder identischer E-Mail-Adresse.\\
    \textbf{Schritte:}
    \begin{enumerate}[label=\arabic*.]
        \item Registrieren Sie einen neuen Benutzer mit Benutzername "testuser" und E-Mail "test@example.com".
        \item Melden Sie sich ab.
        \item Versuchen Sie, einen weiteren Benutzer mit demselben Benutzernamen ("testuser"), aber einer anderen E-Mail zu registrieren.
        \item Versuchen Sie, einen weiteren Benutzer mit derselben E-Mail ("test@example.com"), aber einem anderen Benutzernamen zu registrieren.
    \end{enumerate}
    \textbf{Erwartetes Resultat:} In beiden Fällen wird die Registrierung abgelehnt und eine klare Fehlermeldung angezeigt. Die Datenbank-Integritätsbedingungen (UNIQUE-Constraints) werden korrekt durchgesetzt.\\
    \vspace{0.5em}\textbf{Status:} \textcolor{passcolor}{Bestanden}
\end{testcase}

\begin{testcase}{AUTH-08}
    \textbf{Ziel:} Zeitliche Begrenzung des Passwort-Reset-Tokens (max\_age).\\
    \textbf{Schritte:}
    \begin{enumerate}[label=\arabic*.]
        \item Fordern Sie über die "Passwort vergessen"-Funktion einen Reset-Link an.
        \item Notieren Sie den Token aus der URL des erhaltenen Links.
        \item Warten Sie eine Zeitspanne, die länger ist als die im Code definierte \texttt{max\_age}, welche 15 Minuten beträgt (z.B. 15 Minuten + 1 Minute).
        \item Versuchen Sie, den Link zu verwenden oder den Token manuell zu validieren.
    \end{enumerate}
    \textbf{Erwartetes Resultat:} Der Token wird als abgelaufen zurückgewiesen, und eine entsprechende Fehlermeldung wird angezeigt.\\
    \vspace{0.5em}\textbf{Status:} \textcolor{passcolor}{Bestanden}
\end{testcase}

\begin{testcase}{AUTH-09}
    \textbf{Ziel:} Sicherstellen, dass die Registrierung fehlschlägt, wenn die Passwörter nicht übereinstimmen.\\
    \textbf{Schritte:}
    \begin{enumerate}[label=\arabic*.]
        \item Navigieren Sie zur Registrierungsseite.
        \item Füllen Sie alle Felder aus, aber geben Sie unterschiedliche Werte in die Felder "Passwort" und "Passwort bestätigen" ein.
        \item Klicken Sie auf den Registrierungsbutton.
    \end{enumerate}
    \textbf{Erwartetes Resultat:} Die Registrierung wird blockiert. Eine klare Fehlermeldung wird angezeigt, und der Benutzer bleibt auf der Registrierungsseite.\\
    \vspace{0.5em}\textbf{Status:} \textcolor{passcolor}{Bestanden}
\end{testcase}

\begin{testcase}{AUTH-10}
    \textbf{Ziel:} Ein Benutzer kann sein Konto dauerhaft über die Einstellungsseite löschen.\\
    \textbf{Schritte:}
    \begin{enumerate}[label=\arabic*.]
        \item Melden Sie sich mit einem Testkonto an.
        \item Navigieren Sie zur Einstellungsseite (\texttt{/settings}).
        \item Klicken Sie auf den "Konto löschen"-Button.
        \item Bestätigen Sie die Löschung, da ein Bestätigungsdialog erscheint.
        \item Versuchen Sie, sich erneut mit den Zugangsdaten des gelöschten Kontos anzumelden.
    \end{enumerate}
    \textbf{Erwartetes Resultat:} Das Konto wird aus der Datenbank entfernt. Der Benutzer wird ausgeloggt und zur Start- oder Login-Seite weitergeleitet. Ein erneuter Anmeldeversuch mit den alten Daten schlägt fehl.\\
    \vspace{0.5em}\textbf{Status:} \textcolor{passcolor}{Bestanden}
\end{testcase}

\begin{testcase}{AUTH-11}
    \textbf{Ziel:} Sicherstellen, dass ausgeloggte (nicht angemeldete) Benutzer keinen Zugriff auf geschützte (login-pflichtige) Seiten haben.\\
    \textbf{Schritte:}
    \begin{enumerate}[label=\arabic*.]
        \item Stellen Sie sicher, dass Sie ausgeloggt sind.
        \item Versuchen Sie, eine geschützte Seite direkt über die URL aufzurufen, z.B.:
            \begin{itemize}
                \item \texttt{/agenda}
                \item \texttt{/noten}
                \item \texttt{/todo}
                \item \texttt{/settings}
            \end{itemize}
        \item Wiederholen Sie dies für alle weiteren Seiten, die eine Anmeldung voraussetzen.
    \end{enumerate}
    \textbf{Erwartetes Resultat:} In allen Fällen wird der Benutzer zur Login-Seite weitergeleitet. Kein Zugriff auf geschützte Inhalte ohne vorherige Anmeldung möglich.\\
    \vspace{0.5em}\textbf{Status:} \textcolor{passcolor}{Bestanden}
\end{testcase}

\subsection[Notenverwaltung]{Notenverwaltung\texorpdfstring{\aifootnotemark}{}}
\aifootnotetext

\begin{testcase}{GRADE-01}
    \textbf{Ziel:} Korrekte Berechnung des gewichteten Durchschnitts.\\
    \textbf{Schritte:}
    \begin{enumerate}[label=\arabic*.]
        \item Navigieren Sie zur Notenverwaltungsseite (\texttt{/noten}).
        \item Erstellen Sie ein neues Semester.
        \item Fügen Sie folgende Noten einem beliebigen Fach hinzu:
            \begin{itemize}
                \item Note 1: Wert 5.0, Gewichtung 1 (z.B. Hausaufgaben)
                \item Note 2: Wert 4.0, Gewichtung 2 (z.B. Prüfung)
            \end{itemize}
        \item Fügen Sie eine Note hinzu, die NICHT in den Durchschnitt zählt (setzen Sie das 'counts'-Flag auf False).
    \end{enumerate}
    \textbf{Erwartetes Resultat:} Der gewichtete Durchschnitt für das Fach sollte korrekt berechnet werden: $$ \frac{(5.0 \cdot 1) + (4.0 \cdot 2)}{1+2} = \frac{13}{3} \approx 4.33 $$ Nicht zählende Noten dürfen die Berechnung nicht beeinflussen.\\
    \vspace{0.5em}\textbf{Status:} \textcolor{passcolor}{Bestanden}
\end{testcase}

\begin{testcase}{GRADE-02}
    \textbf{Ziel:} CRUD-Funktionalität für Noten, Fächer und Semester.\\
    \textbf{Schritte:}
    \begin{enumerate}[label=\arabic*.]
        \item \textbf{Erstellen:} Fügen Sie ein Semester hinzu (maximal 8 Semester sind insgesamt möglich, jeder Benutzer kann nur bis zu 8 Semester anlegen).
        \item Überprüfen Sie, ob das neue Semester automatisch mit den vordefinierten Fächern befüllt wird.
        \item Fügen Sie in einem dieser Fächer drei Noten hinzu.
        \item \textbf{Lesen/Anzeigen:} Überprüfen Sie, ob alle hinzugefügten Daten korrekt auf der Übersichtsseite angezeigt werden.
        \item \textbf{Aktualisieren:} Bearbeiten Sie den Wert einer Note und den Namen eines Faches. Überprüfen Sie, ob die Änderung korrekt gespeichert wird und die Durchschnittsberechnung sich anpasst (siehe GRADE-01).
        \item \textbf{Löschen:} Löschen Sie eine einzelne Note. Löschen Sie das gesamte Semester.
    \end{enumerate}
    \textbf{Erwartetes Resultat:} Es können maximal 8 Semester angelegt werden, jedes neue Semester enthält automatisch die notwendigen Fächer. Alle Operationen (Erstellen, Lesen, Aktualisieren, Löschen) sind erfolgreich und die Datenkonsistenz bleibt erhalten.\\
    \vspace{0.5em}\textbf{Status:} \textcolor{passcolor}{Bestanden}
\end{testcase}

\begin{testcase}{GRADE-03}
    \textbf{Ziel:} Automatisches Hinzufügen vordefinierter Fächer zu neu erstellten Semestern.\\
    \textbf{Schritte:}
    \begin{enumerate}[label=\arabic*.]
        \item Melden Sie sich als Benutzer an.
        \item Navigieren Sie zur Notenverwaltungsseite (\texttt{/noten}).
        \item Erstellen Sie ein neues Semester manuell mit dem \enquote{Add Semester} Knopf.
        \item Überprüfen Sie, ob automatisch vordefinierte Standard-Fächer zum Semester hinzugefügt wurden.
        \item Erstellen Sie ein weiteres Semester und überprüfen Sie erneut, ob die Standard-Fächer hinzugefügt wurden.
    \end{enumerate}
    Die Standard-Fächer für das erste Semester sind folgende:
    \begin{itemize}
        \item Mathematik
        \item Deutsch
        \item Englisch
        \item Französisch
        \item Physik
        \item Chemie
        \item Biologie
        \item Geschichte
        \item Geografie
        \item Informatik
        \item Sport
    \end{itemize}
    Für das zweite Semester sind die Standard-Fächer:
    \begin{itemize}
        \item Mathematik
        \item Deutsch
        \item Englisch
        \item Französisch
        \item Physik
        \item Chemie
        \item Biologie
        \item Geschichte
        \item Geografie
        \item Informatik
        \item Wirtschaft und Recht
        \item Sport
    \end{itemize}
    \textbf{Erwartetes Resultat:} Beim Erstellen eines neuen Semesters werden automatisch die in der Konfiguration definierten Standard-Fächer hinzugefügt, sodass der Benutzer direkt mit der Noteneingabe beginnen kann. Die Logik ist in der Backend-Route zur Semestererstellung implementiert.\\
    \vspace{0.5em}\textbf{Status:} \textcolor{passcolor}{Bestanden}
\end{testcase}

\begin{testcase}{GRADE-04}
    \textbf{Ziel:} Fächer innerhalb eines Semesters können per Drag-and-Drop umsortiert werden.\\
    \textbf{Schritte:}
    \begin{enumerate}[label=\arabic*.]
        \item Navigieren Sie zur Notenverwaltungsseite (\texttt{/noten}).
        \item Erstellen Sie ein Semester.
        \item Bewegen Sie den Mauszeiger auf das Drag-Handle-Symbol eines Faches.
        \item Ziehen Sie das Fach an eine andere Position innerhalb des Semesters und lassen Sie es dort los.
        \item Wiederholen Sie den Vorgang mit weiteren Fächern.
        \item Laden Sie die Seite neu und überprüfen Sie, ob die Reihenfolge der Fächer erhalten bleibt.
    \end{enumerate}
    \textbf{Erwartetes Resultat:} Die Reihenfolge der Fächer kann per Drag-and-Drop beliebig verändert werden. Nach dem Neuladen bleibt die neue Reihenfolge erhalten.\\
    \vspace{0.5em}\textbf{Status:} \textcolor{passcolor}{Bestanden}
\end{testcase}

\subsection[Kalender und Agenda]{Kalender und Agenda\texorpdfstring{\aifootnotemark}{}}
\aifootnotetext

\begin{testcase}{CAL-01}
    \textbf{Ziel:} Erstellung und Verwaltung eines Einzelereignisses.\\
    \textbf{Schritte:}
    \begin{enumerate}[label=\arabic*.]
        \item Navigieren Sie zum Kalender (\texttt{/agenda}).
        \item Erstellen Sie ein neues Ereignis mit Titel, Priorität, Datum, Start- und Endzeit.
        \item Speichern Sie das Ereignis.
        \item Überprüfen Sie die korrekte Darstellung im Kalender.
        \item Bearbeiten Sie das Ereignis (z.B. die Zeit).
        \item Löschen Sie das Ereignis.
    \end{enumerate}
    \textbf{Erwartetes Resultat:} Das Ereignis wird korrekt gespeichert, angezeigt, bearbeitet und gelöscht.\\
    \vspace{0.5em}\textbf{Status:} \textcolor{passcolor}{Bestanden}
\end{testcase}

\begin{testcase}{CAL-02}
    \textbf{Ziel:} Erstellung und Verwaltung von wiederkehrenden Ereignissen (Recurrence).\\
    \textbf{Schritte:}
    \begin{enumerate}[label=\arabic*.]
        \item Erstellen Sie ein Ereignis (z.B. "Sporttraining").
        \item Aktivieren Sie die Wiederholungsfunktion und setzen Sie diese auf "Wöchentlich".
        \item Speichern Sie das Ereignis.
        \item Überprüfen Sie, ob wöchentlich einzelne Instanzen des Ereignisses im Kalender erscheinen (für ein ganzes Jahr).
        \item Löschen Sie die Hauptserie (oder eine einzelne Instanz, falls diese Funktion implementiert ist).
    \end{enumerate}
    \textbf{Erwartetes Resultat:} Die Ereignisse werden wöchentlich generiert (für ein ganzes Jahr). Das Löschen der Serie entfernt alle Instanzen.\\
    \vspace{0.5em}\textbf{Status:} \textcolor{passcolor}{Bestanden}
\end{testcase}

\begin{testcase}{CAL-03}
    \textbf{Ziel:} Import von ICS-Dateien.\\
    \textbf{Schritte:}
    \begin{enumerate}[label=\arabic*.]
        \item Besorgen Sie sich eine einfache, gültige \texttt{.ics}-Datei mit einigen Terminen.
        \item Nutzen Sie die gekennzeichnete Importfunktion über der Agenda..
        \item Laden Sie die \texttt{.ics}-Datei hoch.
    \end{enumerate}
    \textbf{Erwartetes Resultat:} Die Ereignisse aus der Datei werden erfolgreich in den Kalender importiert und korrekt angezeigt. Die ursprüngliche \texttt{icalendar}-Struktur wird korrekt interpretiert.\\
    \vspace{0.5em}\textbf{Status:} \textcolor{passcolor}{Bestanden}
\end{testcase}

\begin{testcase}{CAL-04}
    \textbf{Ziel:} Export von Kalenderereignissen als ICS-Datei.\\
    \textbf{Schritte:}
    \begin{enumerate}[label=\arabic*.]
        \item Erstellen Sie mehrere Ereignisse im Kalender (z.B. Einzeltermine, wiederkehrende Termine, ganztägige Ereignisse).
        \item Nutzen Sie die Export-Funktion in der Agenda (z.B. Button "Kalender exportieren" oder ähnlich).
        \item Laden Sie die generierte \texttt{.ics}-Datei herunter.
        \item Öffnen Sie die Datei in einem externen Kalender-Tool (z.B. Google Calendar, Outlook, Apple Calendar).
    \end{enumerate}
    \textbf{Erwartetes Resultat:} Alle Ereignisse werden korrekt in die \texttt{.ics}-Datei exportiert. Die Datei kann in gängigen Kalenderanwendungen ohne Fehler geöffnet werden.\\
    \vspace{0.5em}\textbf{Status:} \textcolor{passcolor}{Bestanden}
\end{testcase}

\subsection{Lern-Tools (Timer und Tipps)}

\begin{testcase}{TOOL-01}
    \textbf{Ziel:} Funktionalität des Pomodoro-Timers (\texttt{/lerntimer}).\\
    \textbf{Schritte:}
    \begin{enumerate}[label=\arabic*.]
        \item Navigieren Sie zur Lerntimer-Seite.
        \item Starten Sie den Timer für die Arbeitsphase (z.B. 25 Minuten).
        \item Vergewissern Sie sich, dass der Countdown korrekt läuft.
        \item Pausieren und setzen Sie den Timer fort.
        \item Lassen Sie den Timer bis zum Ende laufen.
        \item Überprüfen Sie, ob ein akustisches oder visuelles Signal das Ende der Arbeitszeit anzeigt und die Pause beginnt.
    \end{enumerate}
    \textbf{Erwartetes Resultat:} Der Timer funktioniert zuverlässig, kann pausiert/fortgesetzt werden, und signalisiert den Phasenwechsel.\\
    \vspace{0.5em}\textbf{Status:} \textcolor{passcolor}{Bestanden}
\end{testcase}

\begin{testcase}{TOOL-02}
    \textbf{Ziel:} Korrekte Anzeige der Lerntipps (\texttt{/lerntipps}).\\
    \textbf{Schritte:}
    \begin{enumerate}[label=\arabic*.]
        \item Navigieren Sie zur Lerntipps-Seite.
        \item Überprüfen Sie, ob alle Tipps, die in der zugrunde liegenden JSON-Datei (\texttt{tips/learn\_tips.json}) definiert sind, korrekt und leserlich dargestellt werden.
    \end{enumerate}
    \textbf{Erwartetes Resultat:} Alle Lerntipps werden mit ihren Titeln und Beschreibungen vollständig und optisch ansprechend angezeigt.\\
    \vspace{0.5em}\textbf{Status:} \textcolor{passcolor}{Bestanden}
\end{testcase}

\begin{testcase}{TOOL-03}
    \textbf{Ziel:} Die Funktion "Tipp des Tages" zeigt dynamische Inhalte auf dem Dashboard an.\\
    \textbf{Schritte:}
    \begin{enumerate}[label=\arabic*.]
        \item Melden Sie sich an, um das Dashboard anzuzeigen.
        \item Notieren Sie den angezeigten "Tipp des Tages".
        \item Melden Sie sich am nächsten Tag an.
    \end{enumerate}
    \textbf{Erwartetes Resultat:} Auf dem Dashboard wird ein Lerntipp angezeigt. Der Tipp ändert sich an verschiedenen Tagen.\\
    \vspace{0.5em}\textbf{Status:} \textcolor{passcolor}{Bestanden}
\end{testcase}

\begin{testcase}{TOOL-05}
    \textbf{Ziel:} Der Lernalgorithmus plant automatisch Lernblöcke für eine bevorstehende Prüfung und respektiert dabei bestehende Termine.\\
    \textbf{Schritte:}
    \begin{enumerate}[label=\arabic*.]
        \item Navigieren Sie zur Agenda (\texttt{/agenda}).
        \item Erstellen Sie eine Prüfung (z.B. "Mathe-Prüfung") mit Priorität 1, die in 10 Tagen stattfindet.
        \item Erstellen Sie einen weiteren, regulären Termin (z.B. "Zahnarzt") an einem Tag innerhalb des Lernfensters (z.B. in 8 Tagen von 14:00 bis 15:00 Uhr).
        \item Führen Sie den Lernalgorithmus aus (z.B. über einen Button "Lernplan erstellen").
        \item Überprüfen Sie die Agenda-Ansicht.
    \end{enumerate}
    \textbf{Erwartetes Resultat:} Neue Lernblöcke für die "Mathe-Prüfung" erscheinen im Kalender. Die Blöcke werden in den Tagen vor der Prüfung geplant, aber es wird kein Lernblock mit dem "Zahnarzt"-Termin oder in dessen Pufferzeit (30 Min. davor/danach) geplant. Die Gesamtstundenzahl der Blöcke entspricht den Vorgaben aus den Einstellungen für Priorität 1.\\
    \vspace{0.5em}\textbf{Status:} \textcolor{passcolor}{Bestanden}
\end{testcase}

\begin{testcase}{TOOL-06}
    \textbf{Ziel:} Detaillierte Funktionsprüfung des Lernalgorithmus (Prioritäten, Puffer, Konfliktvermeidung).\\
    \textbf{Schritte:}
    \begin{enumerate}[label=\arabic*.]
        \item Erstellen Sie in den Einstellungen (\texttt{/settings}) Prioritätsprofile:
            \begin{itemize}
                \item Priorität 1: 10 Stunden Lernzeit, Lernfenster 14 Tage vor Prüfung.
                \item Priorität 2: 5 Stunden Lernzeit, Lernfenster 7 Tage vor Prüfung.
            \end{itemize}
        \item Erstellen Sie folgende Termine im Kalender:
            \begin{itemize}
                \item Prüfung A (Priorität 1) in 14 Tagen.
                \item Prüfung B (Priorität 2) in 10 Tagen.
                \item Regulärer Termin "Sport" jeden Montag 18:00-19:30 (wiederkehrend).
            \end{itemize}
        \item Führen Sie den Lernalgorithmus aus.
        \item Überprüfen Sie die generierten Lernblöcke auf:
            \begin{itemize}
                \item Korrekte Gesamtstundenzahl (14h für Prüfung A, 7h für Prüfung B).
                \item Verteilung innerhalb des jeweiligen Lernfensters, und wenn möglich, den Lernblock um die bevorzugte Uhrzeit (18:00).
                \item Keine Überschneidungen mit bestehenden Terminen.
                \item Einhaltung des min. 30-Minuten-Puffers vor und nach dem "Sport"-Termin.
            \end{itemize}
        \item Ändern Sie die Uhrzeit von "Sport" und führen Sie den Algorithmus erneut aus.
    \end{enumerate}
    \textbf{Erwartetes Resultat:} Alle Lernblöcke werden regelkonform platziert. Es treten keine Konflikte oder Überschneidungen auf. Der Algorithmus passt sich dynamisch an Änderungen bestehender Termine an. Die Summe der Lernblockdauern stimmt mit den Einstellungen überein.\\
    \vspace{0.5em}\textbf{Status:} \textcolor{passcolor}{Bestanden}
\end{testcase}

\subsection[To-Do Listen]{To-Do Listen\texorpdfstring{\aifootnotemark}{}}
\aifootnotetext

\begin{testcase}{TODO-01}
    \textbf{Ziel:} Kategorien für To-Do Listen können erstellt und gelöscht werden.\\
    \textbf{Schritte:}
    \begin{enumerate}[label=\arabic*.]
        \item Navigieren Sie zur To-Do Listen Seite (\texttt{/todo}).
        \item Klicken Sie auf den Button "Neue Kategorie".
        \item Geben Sie einen Namen für die Kategorie ein und bestätigen Sie.
        \item Überprüfen Sie, ob die neue Kategorie erscheint.
        \item Löschen Sie die Kategorie wieder.
    \end{enumerate}
    \textbf{Erwartetes Resultat:} Kategorien können erfolgreich erstellt und gelöscht werden. Nach dem Löschen verschwindet die Kategorie aus der Liste.\\
    \vspace{0.5em}\textbf{Status:} \textcolor{passcolor}{Bestanden}
\end{testcase}

\begin{testcase}{TODO-02}
    \textbf{Ziel:} Aufgaben können zu Kategorien hinzugefügt, abgehakt (erledigt) und gelöscht werden.\\
    \textbf{Schritte:}
    \begin{enumerate}[label=\arabic*.]
        \item Erstellen Sie eine neue Kategorie.
        \item Fügen Sie mehrere Aufgaben (To-Do Items) hinzu.
        \item Haken Sie eine Aufgabe als erledigt ab.
    \end{enumerate}
    \textbf{Erwartetes Resultat:} Aufgaben werden korrekt hinzugefügt, können als erledigt markiert werden und somit auch gelöscht werden. Die Anzeige aktualisiert sich entsprechend.\\
    \vspace{0.5em}\textbf{Status:} \textcolor{passcolor}{Bestanden}
\end{testcase}

\begin{testcase}{TODO-03}
    \textbf{Ziel:} Kategorien lassen sich ein- und ausklappen (Accordion-Funktion).\\
    \textbf{Schritte:}
    \begin{enumerate}[label=\arabic*.]
        \item Erstellen Sie mehrere Kategorien mit Aufgaben.
        \item Klicken Sie auf den Kopfbereich einer Kategorie.
        \item Überprüfen Sie, ob die Aufgabenliste der Kategorie ein- bzw. ausgeklappt wird.
    \end{enumerate}
    \textbf{Erwartetes Resultat:} Die Accordion-Funktion funktioniert zuverlässig. Kategorien können unabhängig voneinander ein- und ausgeklappt werden.\\
    \vspace{0.5em}\textbf{Status:} \textcolor{passcolor}{Bestanden}
\end{testcase}

\subsection[Ästhetik/User Interface]{Ästhetik/User Interface\texorpdfstring{\aifootnotemark}{}}
\aifootnotetext

\begin{testcase}{UI-01}
    \textbf{Ziel:} Responsive Design und Barrierefreiheit der Navigation.\\
    \textbf{Schritte:}
    \begin{enumerate}[label=\arabic*.]
        \item Testen Sie die Anwendung auf einem Desktop-Browser und einem mobilen Emulator/Gerät (oder verkleinern Sie das Browserfenster).
        \item Überprüfen Sie, ob alle Elemente (Navigation, Formulare, Tabellen) sich an die Bildschirmgröße anpassen.
        \item Überprüfen Sie, ob die primäre Navigation auf dem Mobilgerät über ein Hamburger-Menü erreichbar ist.
    \end{enumerate}
    \textbf{Erwartetes Resultat:} Die App ist auf allen Geräten vollständig funktional und ästhetisch ansprechend (\textbf{Ästhetik}, \textbf{Benutzerfreundlichkeit}).\\
    \vspace{0.5em}\textbf{Status:} \textcolor{passcolor}{Bestanden}
\end{testcase}

\begin{testcase}{UI-02}
    \textbf{Ziel:} Visuelle Konsistenz und Farbkonzept.\\
    \textbf{Schritte:}
    \begin{enumerate}[label=\arabic*.]
        \item Navigieren Sie durch alle Hauptseiten (Home, Grades, Calendar, Timer, Tips).
        \item Achten Sie auf die Einheitlichkeit von Farben, Schriftarten (Typografie) und Button-Stilen.
    \end{enumerate}
    \textbf{Erwartetes Resultat:} Die visuelle Gestaltung ist über die gesamte App hinweg konsistent und entspricht einem definierten Designkonzept.\\
    \vspace{0.5em}\textbf{Status:} \textcolor{passcolor}{Bestanden}
\end{testcase}

\begin{testcase}{UI-03}
    \textbf{Ziel:} Überprüfung der Dark-Mode-Funktionalität und visuellen Konsistenz.\\
    \textbf{Schritte:}
    \begin{enumerate}[label=\arabic*.]
        \item Navigieren Sie zu einer beliebigen Hauptseite der Anwendung (z.B. Dashboard, Notenverwaltung, Agenda).
        \item Wechseln Sie in den Einstellungen (\texttt{/settings}) das Aussehen auf "Dunkel" (Dark Mode).
        \item Überprüfen Sie, ob die gesamte Benutzeroberfläche in den dunklen Farben dargestellt wird.
        \item Navigieren Sie durch verschiedene Seiten und prüfen Sie, ob alle Komponenten (Navigation, Buttons, Tabellen, Popups, Formulare) korrekt im Dark Mode angezeigt werden.
        \item Wechseln Sie zurück auf "Hell" (Light Mode) und überprüfen Sie, ob die Oberfläche wieder korrekt dargestellt wird.
        \item Testen Sie die Einstellung "Systemeinstellung" und prüfen Sie, ob sich die App an die Systemeinstellung des Betriebssystems anpasst.
    \end{enumerate}
    \textbf{Erwartetes Resultat:} Die App wechselt zuverlässig zwischen Light und Dark Mode. Alle UI-Elemente sind im Dark Mode gut lesbar, die Farben sind konsistent, und es treten keine Darstellungsfehler auf. Die Einstellung "Systemeinstellung" funktioniert wie erwartet.\\
    \vspace{0.5em}\textbf{Status:} \textcolor{passcolor}{Bestanden}
\end{testcase}

\subsection[Informationsseiten]{Informationsseiten\texorpdfstring{\aifootnotemark}{}}
\aifootnotetext

\begin{testcase}{INFO-01}
    \textbf{Ziel:} Die Seite "Über uns" (\texttt{/about}) ist erreichbar und zeigt die vorgesehenen Informationen an.\\
    \textbf{Schritte:}
    \begin{enumerate}[label=\arabic*.]
        \item Navigieren Sie zur Seite \texttt{/about} über das Menü oder direkt per URL.
        \item Überprüfen Sie, ob die Seite geladen wird und die Informationen über das KantiKoala-Projekt angezeigt werden.
    \end{enumerate}
    \textbf{Erwartetes Resultat:} Die Seite "Über uns" wird korrekt angezeigt und enthält die relevanten Informationen zum Projekt.\\
    \vspace{0.5em}\textbf{Status:} \textcolor{passcolor}{Bestanden}
\end{testcase}

\begin{testcase}{INFO-02}
    \textbf{Ziel:} Die Hilfeseite (\texttt{/hilfe}) ist erreichbar und stellt Unterstützung für Benutzer bereit.\\
    \textbf{Schritte:}
    \begin{enumerate}[label=\arabic*.]
        \item Navigieren Sie zur Seite \texttt{/hilfe} über das Menü oder direkt per URL.
        \item Überprüfen Sie, ob die Seite geladen wird und Hilfetexte bzw. Anleitungen angezeigt werden.
    \end{enumerate}
    \textbf{Erwartetes Resultat:} Die Hilfeseite wird korrekt angezeigt und bietet verständliche Unterstützung für Benutzer.\\
    \vspace{0.5em}\textbf{Status:} \textcolor{passcolor}{Bestanden}
\end{testcase}

\begin{testcase}{INFO-03}
    \textbf{Ziel:} Die Datenschutzerklärung (\texttt{/datenschutzerklaerung}) ist erreichbar und vollständig einsehbar.\\
    \textbf{Schritte:}
    \begin{enumerate}[label=\arabic*.]
        \item Navigieren Sie zur Seite \texttt{/datenschutzerklaerung} über den Footer oder direkt per URL.
        \item Überprüfen Sie, ob die Datenschutzerklärung vollständig angezeigt wird.
    \end{enumerate}
    \textbf{Erwartetes Resultat:} Die Datenschutzerklärung ist vollständig und korrekt sichtbar.\\
    \vspace{0.5em}\textbf{Status:} \textcolor{passcolor}{Bestanden}
\end{testcase}

\end{document}
