\documentclass[a4paper,12pt]{article}
\usepackage[utf8]{inputenc}
\usepackage[T1]{fontenc}
\usepackage{geometry}
\usepackage{ifthen}
\usepackage{makecell}
\usepackage{graphicx}
\usepackage{hyperref}
\geometry{left=2.5cm,right=2.5cm,top=2.5cm,bottom=2.5cm}
\usepackage[ngerman]{babel}
\usepackage[font=small,labelfont=bf]{caption} % needed for \captionof
\usepackage[subrefformat=parens]{subcaption}
% Command for a journal entry (no separator)
\newcommand{\JournalEntry}[7][false]{
    \subsection*{#2}
    \begin{tabular}{|l|l|}
        \hline
        \textbf{Gearbeitete Stunden} & #3 \\ \hline
    \end{tabular}

    \subsubsection*{\textbf{Arbeitsschritte}}
    #4

    \subsubsection*{\textbf{Ergebnisse}}
    #5

    \subsubsection*{\textbf{Begründung und Reflexion}}
    #6

    \subsubsection*{\textbf{Beleg}}
    #7

    \vspace{1cm}

    % Optional: Add a horizontal line to separate entries
    \ifthenelse{\equal{#1}{true}}{
        \hrule
    
    \vspace{1cm}
    }{}
}

% Command for a day block with separator
\newcommand{\JournalDay}[1]{
    \section*{Datum: #1}
    \vspace{0.3cm}
}

% Helper command to set the date heading for the entry
% #1: Date (e.g., 19.03)
\newcommand{\StandortHeader}[1]{
    \subsection*{Standortbestimmung vom #1}
}

% Helper command to add the horizontal separator line
\newcommand{\StandortFooter}{
    \vspace{0.5cm}
    \hrule
    \vspace{0.5cm}
}

\newcommand{\image}[2]{%
  \begin{center}%
      \centering
      \includegraphics[width=0.5\linewidth]{img/arbeitsjournal/#1}%
      \captionof{figure}{#2}%
  \end{center}%
}
\newcommand{\gitcommit}[1]{
    \textbf{Git Commit SHA:} \href{https://github.com/CoderAryanAnand/lernapp/commit/#1}{#1} \\
}

\begin{document}

\makeatletter
\begin{titlepage}
    \centering
    \vspace*{1cm}
       { \includegraphics[width=6cm]{img/kanti-baden.png}}\\[1cm]

    {\LARGE \textbf{Kanti Koala}}\\
    {\textbf{Die Lern- und Studienhilfsapp für Schüler:innen der Kantonsschule Baden}}\\[1cm]

    {Maturitätsarbeit, Kantonsschule Baden}\\
    {Arbeitsjournal}\\[1cm]
    
    \textbf{Erstbetreuer: }{Michael Schneider}\\
    \textbf{Zweitbewerterin: }{Julia Smits}\\[1cm]
    
    \textbf{Geschrieben von: }{Aryan Anand (G22b), Simon Haddon (G22b)}\\[1cm]
 
    \date{\large Datum: 11. November 2025}
    {\@date\\}
\end{titlepage}
\makeatother

% 12.03.2025
\JournalDay{12.03.2025}
\JournalEntry[true]{Projektstart: Vertrag und Interviewvorbereitung}{Je $\sim$3 Lektionen (Simon \& Aryan)}{Vertrag ausgefüllt. Projektplan entworfen. INTERVIEW: Angefangen, Themen und Fragen zu entwerfen (z.T. mit Hilfe von ChatGPT Ideen für Fragen geholt).}{Erster Projektplan und Vertragsentwurf erstellt. Erste Ideen für Interviewfragen gesammelt.}{Ideen von ChatGPT sind eigentlich okay, aber wahrscheinlich nicht spezifisch genug.}{\image{12032025.png}{Interview Fragen Ideen von ChatGPT}}

% 19.03.2025
\JournalDay{19.03.2025}
\JournalEntry[true]{Vertrag, Interviewfragen und Github-Initialisierung}{Je $\sim$3 Lektionen (Simon \& Aryan)}{Weiter am Vertrag gearbeitet. Interview-Fragen gesammelt und begründet. Eine erste Mail für die PPP-Lehrpersonen entworfen. Lizenz für Webseite gewählt. Github Repository initialisiert. Flask aufgestellt.}{Vertragsentwurf, Interviewfragen, E-Mail-Draft und Lizenzwahl abgeschlossen. Github Repository mit Flask-Grundgerüst aufgesetzt. Basic Account Management Features sind funktionsfähig.}{Wir sind gut vorangekommen, mittlerweile noch gut im Zeitplan. Basic Account Management Features waren einfacher aufzustellen, als erwartet.}{\gitcommit{60391597276b8b42d60f4650989da4aeea664b4f}\image{beleg1903.png}{Interview Fragen weiterformuliert}}

% 21.03.2025
\JournalDay{21.03.2025}
\JournalEntry[true]{Theorie und Lagebesprechung}{Simon (ohne Stundenangabe)}{Theorie-Dokumente zu Interviews angeschaut, welche von Herr Schneider zur Verfügung gestellt wurden. Mail für Frau Suter / Herr Schmocker ausgearbeitet. Lagebesprechung mit Herr Schneider.}{Theoretische Grundlagen für Interviews und konkretes Feedback zum Arbeitsvertrag erhalten.}{Konkretes Feedback erhalten und die theoretische Grundlage für die Interviews erarbeitet.}{\image{Beleg2103.png}{Erste Version der Mail, die wir verschicken wollten.}}

% 26.03.2025
\JournalDay{26.03.2025}
\JournalEntry[true]{Code und Recherche}{Je $\sim$3 Lektionen (Simon \& Aryan)}{Aryan: Code, «Forgot Password» feature, Kalender mit Flask, E-Mail erstellt. Simon: Recherche.}{«Forgot Password»-Funktion und Flask-Kalender-Grundgerüst begonnen.}{Gute Fortschritte in der Programmierung und Recherche gemacht.}{\gitcommit{e8767ec76e2a1c2399ba3833db67da7a4bcb6f14}\image{Beleg2603.png}{Teil unserer Internetrecherche}}

% 28.03.2025
\JournalDay{28.03.2025}
\JournalEntry[true]{Programmierung}{Etwa 90 Minuten (Aryan)}{Programmieren.}{Fortschritte in der Programmierung erzielt.}{}{\gitcommit{042001c5cd1d9a716facaa989158b014c4681674}}

% 02.04.2025
\JournalDay{02.04.2025}
\JournalEntry[true]{Arbeitsvertrag, Interviewfragebogen und Agenda-Funktionalität}{Je $\sim$3 Lektionen (Aryan \& Simon)}{Simon: Arbeitsvertrag fast fertig, Interviewfragebogen angefangen (in \LaTeX\space / Overleaf). Aryan: Agenda Events bearbeiten, erstellen, und löschen; Strategie für eine .ical implementation.}{Arbeitsvertrag fast fertig. Interviewfragebogen in \LaTeX\space begonnen. Funktionen zum Bearbeiten, Erstellen und Löschen von Agenda-Events implementiert, sowie eine Strategie für die .ical-Funktionalität entwickelt.}{Wir haben gemerkt, dass das Umfragedesign nicht im Interview vorkommen muss, da das Internet und die Theoriedokumente bereits genügend gute Antworten liefern. D.h., dass wir da mehr Recherche machen als konkret fragen, was paralleles Arbeiten ermöglicht. Wir haben gemerkt, dass die Zeit schneller vergeht, als wir meinten. Copilot ist nützlich für zeitaufwendige Aufgaben wie Kommentieren und Grundlagen aufbauen.}{\gitcommit{b0e3e3e13ea43fb0f3addfb06928fa09748f1b96}\image{Beleg0204.png}{Erste Überlegungen zum Interviewfragebogen, Link zur Overleaf-Datei}}

% 04.04.2025
\JournalDay{04.04.2025}
\JournalEntry[true]{Bug Fixes und Wiederholbare Events}{Aryan: 2h (Aryan \& Simon)}{Simon: Interviewfragebogen ausgearbeitet, Mail zum Abschicken weiter vorbereitet. Aryan: Bug fix von Account deletion, Logik von wiederholbaren Events.}{Interviewfragebogen weiter ausgearbeitet und Mail vorbereitet. Bugfix für Account Deletion implementiert. Logik für wiederholbare Events begonnen.}{SQL hat "cascade delete", welches sehr nützlich ist um zwei Datenbanken zu verbinden. Der Bug fix lief viel besser als erwartet.}{\gitcommit{70ab566b77bbf63e572c50e7c2d5e4a75772e39a}\image{Beleg0404.png}{Weitere Arbeit an der Mail}}

% 06.04.2025
\JournalDay{06.04.2025}
\JournalEntry[true]{Repeatable Events Implementierung}{2h (Aryan)}{Repeatable Events (ausser löschen) implementiert. Europäisches Datum Format implementiert. Anderer View von Agenda (Monat view) implementiert. Implementieren von repeatable Events Logik mithilfe von ChatGPT.}{Repeatable Events (ausser Löschen), europäisches Datumsformat und Monatsansicht der Agenda implementiert.}{Ich habe gelernt, wie Fullcalendar extra Infos speichert, auch die `datetime`/`dateutil`-Logik für repeatable events (wurde primär von Copilot gemacht).}{\gitcommit{9781b20631258235f76e13fc13b43abad8bc4c01}}

% 08.04.2025
\JournalDay{08.04.2025}
\JournalEntry[true]{Beginn des schriftlichen Kommentars}{1.5h (Aryan)}{Fing an, den schriftlichen Kommentar zu schreiben.}{Der schriftliche Kommentar wurde begonnen.}{Lief viel besser als erwartet (kein chatGPT!). Ich weiss jedoch nicht, wo ich Quellen angeben muss.}{\image{08042025.png}{Screenshot von Overleaf}}

% 10.04.2025
\JournalDay{10.04.2025}
\JournalEntry[true]{Bericht und .ical Feature}{2.5h (Aryan)}{Alles im Bericht, das bisher programmiert wurde, dokumentiert. .ical import feature implementiert.}{Der Bericht wurde um die bisherige Programmierung ergänzt. Das .ical Import Feature ist funktionsfähig.}{Lief alles sehr gut, das Schreiben vom Bericht ist schneller als erwartet.}{\gitcommit{db7c0375762d1c685b0fdecc6e8aabbf5a6193cb}\image{10042025.png}{Screenshot vom Bericht}}

% 13.04.2025
\JournalDay{13.04.2025}
\JournalEntry[true]{Umfrage, Server und Prioritäten}{0.5h (Aryan \& Simon)}{Simon: Arbeit an der Umfrage (via Microsoft Forms). Aryan: Überlegung vom Server, Priorität von Events implementiert.}{Arbeit an der Umfrage begonnen. Gedanken zur Serverimplementierung gemacht. Priorität von Events implementiert.}{Der Server wird vielleicht schwer zu implementieren und kann Kosten verursachen.}{\gitcommit{daabd930eefc0c6886b806d0635ed4bf2c978946}\image{Beleg1304.png}{Erste Arbeit an der Umfrage.}}

% 14.04.2025
\JournalDay{14.04.2025}
\JournalEntry[true]{Umfrage und Farbentheorie}{~2 Stunden (Simon)}{Weitere Arbeit an der Umfrage - Layout, Fragen, Technisches. Farbentheorie für das Umfragedesign recherchiert.}{Die Umfrage ist sehr weit fortgeschritten. Farbentheorie-Recherche für das Design durchgeführt.}{}{\image{Beleg1404.png}{Rechercheergebnisse für die Farbenlehre.}}

% 15.04.2025
\JournalDay{15.04.2025}
\JournalEntry[true]{Logo-Skizzen}{Ohne Stundenangabe (Simon)}{Erste Arbeit am Logo, erste Skizzen und Überlegungen gemacht.}{Erste Skizzen und Überlegungen zum Logo-Design erstellt.}{}{\image{Beleg1504.png}{Eine erste Skizze für das Logo.}}

% 21.04.2025
\JournalDay{21.04.2025 - 24.04.2025}
\JournalEntry[true]{Recherche Lerntheorie}{$\sim$5 Stunden (Simon)}{Bücher über Lerntheorie vor den Interviews ausgeliehen (Auf Empfehlung von Herr Schmocker) und gelesen. Wichtigste Punkte daraus notiert.}{Wichtige Erkenntnisse aus der Buchrecherche zur Lerntheorie gewonnen.}{Das Ausleihen ging sehr einfach, hätten wir vielleicht schon vorher machen sollen.}{\image{Beleg2104.png}{Einige Notizen zu unserer Literaturrecherche.}}

% 24.04.2025
\JournalDay{24.04.2025}
\JournalEntry[true]{Erstes Interview}{Etwa 1 $\frac{1}{4}$ Stunden (Simon \& Aryan)}{Erstes Interview durchgeführt, mit Frau Suter, sehr positiv verlaufen.}{Das erste Interview mit Frau Suter wurde erfolgreich abgeschlossen.}{}{\image{Beleg2404.png}{Ein Teil der Notizen aus dem Interview mit Frau Suter.}}

% 30.04.2025
\JournalDay{30.04.2025}
\JournalEntry[true]{Recherche Daily Tipps}{0.5h (Aryan)}{Ein bisschen für Daily Tipps recherchiert.}{Recherche für Daily Tipps begonnen.}{Im MA-Archiv gibt es gute Arbeiten, aber es ist schwierig daraus Tipps zu finden.}{\image{30042025.png}{MA-Archiv Recherche}}

% 07.05.2025
\JournalDay{07.05.2025}
\JournalEntry[true]{Programmierung und Recherche}{Je $\sim$3 Lektionen (Simon \& Aryan)}{Simon: weitere Bücherrecherche, Erkenntnisse notiert. Aryan: Forgot password implementiert (ohne Testing), repeatable events löschen.}{Bücherrecherche fortgesetzt. «Forgot password»-Funktion implementiert. Logik zum Löschen von wiederholbaren Events implementiert.}{Die Logik für «Forgot Password» war viel komplexer als gedacht, jedoch die für «repeatable events» viel einfacher.}{\gitcommit{d951c9dc237e70be95b8eec2594d8bdf9b9045df}\image{Beleg0705.png}{Weitere Notizen aus der Literaturrecherche.}}

% 08.05.2025
\JournalDay{08.05.2025}
\JournalEntry[true]{Zweites Interview}{~1 Stunde (Simon \& Aryan)}{Zweites Interview - jetzt mit Herr Schmocker - durchgeführt.}{Das zweite Interview mit Herr Schmocker wurde erfolgreich durchgeführt.}{Die Interviews auf Mundart durchzuführen war ein Fehler. ABER wir haben sowohl von Frau Suter wie auch Herr Schmocker sehr gutes Feedback erhalten, u.a. auch auf unsere Professionalität $\rightarrow$ Das ist gut.}{\image{Beleg0805.png}{Notizen aus unserem Interview mit Herr Schmocker.}}

% 09.05.2025
\JournalDay{09.05.2025}
\JournalEntry[true]{Umfrage, Testing und Daily Tipps}{3h (Simon \& Aryan)}{Simon: Umfrage-Feedback von Herr Schmocker ergänzt, Umfrage überarbeitet. Bis dahin: Auch noch mehr Bücherrecherche durchgeführt. Aryan: Testing und fixing von Forgot password, Daily tipps template.}{Umfrage überarbeitet, Bücherrecherche abgeschlossen. «Forgot password»-Funktion getestet und gefixt. Daily Tipps Template erstellt.}{Lief alles perfekt.}{\gitcommit{ab6a2237a50f63ec74682e17c2e43b0b5aa40dbc}\image{Beleg0905.png}{Notizen aus unserem Meeting mit Herr Schneider.}}

% 14.05.2025
\JournalDay{14.05.2025}
\JournalEntry[true]{Umfrage Testläufe und Serververbindung}{90 min (Aryan \& Simon)}{Simon: Umfrage nochmals überarbeitet, Klassenkameraden um erste Testläufe gefragt, um dies «praktisch» zu testen. Aryan: Probierte Serververbindung, aber nur Security stuff geschafft.}{Umfrage überarbeitet und erste Testläufe mit Klassenkameraden gestartet. Erste Schritte zur Serververbindung (Security) unternommen.}{Obwohl der gesamte Recherche-Prozess länger gebraucht hatte als gedacht, haben wir gute Resultate erzielt. In der Zukunft sollten wir aber mehr Zeit für Recherche einplanen. Serververbindung ist viel schwieriger als erwartet, aber trotzdem wichtiges Security Zeugs gemacht.}{\gitcommit{03a2f7e4cdf8ddf5cf17f9719de87a87683c4e3f}\image{Beleg1405.png}{Screenshot der Umfrage}}

% 16.05.2025
\JournalDay{16.05.2025}
\JournalEntry[true]{Server-Implementierung}{2h (Aryan)}{Immernoch Server am versuchen.}{Fortschritte bei der Serverimplementierung.}{Noch nicht funktioniert, aber bin sehr nah an einer Lösung.}{\gitcommit{bdc9fedc274a2b7a69f3aa0b61e43fd3df38ad91}}

% 18.05.2025
\JournalDay{18.05.2025}
\JournalEntry[true]{Serververbindung und Quellenformatierung}{2.5h (Aryan)}{Endlich Serververbindung implementiert. Quellen für das schriftliche Kommentar formatiert (\LaTeX).}{Funktionierende Serververbindung implementiert. Quellen für den Bericht in \LaTeX\space formatiert.}{Gut, dass es funktionierte.}{\gitcommit{5d67a4aa875793cddb767b5541634ba29c874bbb}\image{18052025.png}{Quellenformatierung in Overleaf}}

% 20.05.2025
\JournalDay{20.05.2025}
\JournalEntry[true]{Umfrage-Feedback-Verarbeitung}{~30 min. (Simon)}{Weiteres Feedback von Frau Suter ergänzt/umgestaltet.}{Umfrage nach Feedback von Klassenkameraden und Frau Suter umgestaltet.}{Leider haben wir von unseren Klassenkameraden nur 3 Antworten bekommen, aber trotzdem wertvolles Feedback. Auswertung des Feedbacks unserer Klassenkameraden hat einiges gezeigt, und wir sind hauptsächlich mit den Resultaten zufrieden, wenn auch sie nicht immer die tiefgründigsten Antworten gegeben haben, weswegen wir nun versucht haben, dies ein wenig mit Umformulierungen entgegenzuwirken.}{\image{Beleg2005.png}{Feedback, welches wir von Frau Suter erhalten haben.}}

% 21.05.2025
\JournalDay{21.05.2025}
\JournalEntry[true]{Umfrage-Abschluss}{~1h (Simon)}{Finale Änderungen an der Umfrage gemacht, an Frau Hofmann geschickt.}{Die finalen Änderungen an der Umfrage wurden vorgenommen und sie wurde zur Freigabe geschickt.}{Die Arbeit an der Umfrage \& Recherche hat viel länger gedauert, als erhofft / zuerst angenommen.}{\image{Beleg2105.png}{Screenshot der Mail an Frau Hofmann}}

% 22.05.2025
\JournalDay{22.05.2025}
\JournalEntry[true]{Logo-Design}{Ohne Stundenangabe (Simon)}{Weitere Arbeit am Logo, von Skizzen weitergearbeitet und angefangen, erste Versionen auszuarbeiten.}{Erste Versionen des Logos basierend auf Skizzen ausgearbeitet.}{Auseinandersetzung mit «Was macht ein gutes Logo aus», wir haben, glaube ich, noch keine endgültige Version getroffen.}{\image{Beleg2205.png}{Eine erste, farbige Version des Logos.}}

% 27.05.2025
\JournalDay{27.05.2025}
\JournalEntry[true]{Umfrage-Versand}{Ohne Stundenangabe (Simon)}{Nach letztem Feedback von Frau Hofmann wurde die Umfrage (ENDLICH) losgeschickt.}{Die Umfrage wurde nach Freigabe versandt.}{}{\image{Beleg2705.png}{Screenshot der Sammelmail, welche unsere Umfrage enthielt.}}

% 28.05.2025
\JournalDay{28.05.2025 - 04.06.2025}
\JournalEntry[true]{Interviews Transkribieren}{Ca. 4 Stunden (Simon)}{Interviews werden transkribiert und direkt auf Hochdeutsch übersetzt. Ebenso werden, so gut wie möglich, sprachliche Aussetzer («ähm», bspw.) entfernt, sollte aber sonst so getreu wie möglich bleiben.}{Transkription und Übersetzung der Interviews ins Hochdeutsche begonnen.}{SEHR aufwendig, in der Zukunft sollten die Interviews auf Hochdeutsch und nicht Mundart durchgeführt werden, da dies das Transkribieren erschwert, da digitale Tools nicht gut funktionieren.}{\image{Beleg0406.png}{Screenshot des Anfangs unserer Transkription des Interviews mit Frau Suter.}}

% 04.06.2025
\JournalDay{04.06.2025 - 10.06.2025}
\JournalEntry[true]{Interviews Transkribieren}{Ca. 9 Stunden (Simon)}{Beide Interviews fertig transkribiert - insgesamt ca. 13 Stunden für die je $\sim$1-stündigen Interviews.}{Beide Interviews fertig transkribiert.}{Die Transkription wird aber definitiv sehr nützlich sein, um sowohl direkte Zitierungen wie um auch anderes (Bspw. für unsere «Daily Tipps») daraus zu nehmen.}{\image{Beleg1006.png}{Screenshot des Anfangs unserer Transkription des Interviews mit Herrn Schmocker.}}

% 11.06.2025
\JournalDay{11.06.2025}
\JournalEntry[true]{Arbeitsjournal-Übertragung und Umfrage-Auswertung}{Ohne Stundenangabe (Simon \& Aryan)}{Arbeitsjournal erstmals auf Word übertragen \& ausgefüllt, finale Änderungen an Dokumenten. Umfrageresultate angefangen, auszuwerten in Excel.}{Arbeitsjournal-Übertragung und finale Dokumentenänderungen abgeschlossen. Auswertung der 83 Umfrageantworten in Excel begonnen.}{Wir haben 83 Antworten erhalten auf die Umfrage, was viel positiver ist, als wir erwartet hätten. Aber: Wenn man es auf die einzelnen Stufen aufteilt, ist es immer noch nicht so viel.}{\image{11062025.png}{Screenshot des Arbeitsjournals in Word}}

% 16.06.2025 - 17.06.2025
\JournalDay{16.06.2025 - 17.06.2025}
\JournalEntry[true]{Vorbereitung und Design der PowerPoint-Präsentation}{Ca. 3 Stunden (Simon)}{Design der PowerPoint-Präsentation ausgearbeitet (Farbschema, Layout und Design-Regeln). Daten von Umfrage weiter ausgewertet und visuell dargestellt (Pie-Charts). Farbschema: Helle, leicht nicht saturierte Farbtöne gewählt (Weisser Hintergrund, (dunkel)grauer Text, türkise Flächen, Blau-Schema für Graphen). Layout-Regeln: Möglichst einheitlich zwischen slides. Titelseite: «KantiKoala» gross und fett, türkise/blaue Bänder, Name und Klasse. Reguläre Seiten: Titel links vom Strich, Inhalt möglichst kurz gefasst, Titel hochgesetzt, Erkenntnisse unter Titel. Datenauswertung: Graphen so weit wie möglich, um prozentuelle Auslegung darzustellen (Bspw. ob gewisse Stressoren zelteten). Bar-Graphen für einfache Nummern-Vergleiche.}{Ein vollständiges Designkonzept (Farb- und Layout-Schema) für die Präsentation ist erstellt. Die Umfragedaten wurden weiter ausgewertet und für die visuelle Darstellung (Pie-Charts, Bar-Graphen) vorbereitet. Die Struktur der Titelseite und der regulären Seiten ist definiert.}{Das Design ist auf eine einfache, helle und visuell ansprechende Darstellung ausgerichtet.}{\image{Beleg1706.png}{Eine der Slides aus unserer Powerpoint-Präsentation}}

% 13.08.2025
\JournalDay{13.08.2025}
\JournalEntry[true]{Interview-Analyse und Algorithmus-Grundlagen}{\makecell{Ca. 3 Stunden (Simon) \\ zusätzliche Zeit für Aryan \& Simon für Algorithmus}}{Simon: Zentrale Aussagen der Interviews mit Herr Schmocker \& Frau Suter auffassen. Erkenntnisse werden nach Fragen geordnet, und auch generelle Tipps/Empfehlungen zum Lernen und für die Applikation gesammelt. Dokument erstellt \& Layout. Erste Erkenntnisse gesammelt in Lerntechniken \& Lernmethoden. Aryan \& Simon: Algorithmus-Gedanken für die Planung ('wichtig' und 'advanced').}{Simon: Dokument zur Interview-Analyse erstellt und erste Erkenntnisse aus den Interviews in Lerntechniken und -methoden gesammelt und geordnet. Aryan \& Simon: Grundlegende Anforderungen für den Planungsalgorithmus definiert (Lernstoff gut ausbreiten, gute Zeiten vorschlagen, Zeitplan nicht überfüllen, Wochenenden depriorisieren, Prioritäten setzen).}{Der Fokus liegt auf der Umsetzung der theoretischen Erkenntnisse aus den Interviews in konkrete Empfehlungen für die Web-Applikation. Die Algorithmus-Gedanken stellen die Basis für die Zeitplanungsfunktionalität dar.}{\image{13082025.png}{Algorithmus-Gedanken auf einem Whiteboard}}

% 14.08.2025
\JournalDay{14.08.2025}
\JournalEntry[true]{Interview-Analyse Herr Schmocker}{Ca. 2h (Simon)}{Erkenntnisse von Herr Schmocker auffassen, wenn möglich (nahezu) fertig.}{Einige Fragen vollständig bearbeitet. Viele Tipps zur Lernapp ergänzt.}{Das Ziel, die Erkenntnisse von Herrn Schmocker nahezu fertig aufzufassen, wurde gut erreicht, indem viele Tipps ergänzt und Fragen bearbeitet wurden.}{\image{Beleg1408.png}{Lernerkenntnisse aus der Interviewanalyse von Herr Schmocker}}

% 16.08.2025
\JournalDay{16.08.2025}
\JournalEntry[true]{Interview-Analyse Herr Schmocker fortgesetzt}{Ca. 1h (Simon)}{Weiterhin Erkenntnisse von Herr Schmocker auffassen.}{Weiter an Herrn Schmockers Kommentaren gearbeitet.}{Die Arbeit an den Erkenntnissen aus dem Interview mit Herrn Schmocker wurde fortgesetzt.}{\image{Beleg1608.png}{Web-Applikationserkenntnisse aus der Interviewanalyse von Herr Schmocker}}

% 18.08.2025
\JournalDay{18.08.2025}
\JournalEntry[true]{Interview-Analyse von Herr Schmocker abgeschlossen}{Ca. 1.5h (Simon)}{Herr Schmockers Erkenntnisse fertig bearbeiten. Mit den Erkenntnissen von Frau Suter anfangen.}{Die Analyse der Erkenntnisse aus dem Interview mit Herrn Schmocker ist fertiggestellt. Die Aufarbeitung der Erkenntnisse von Frau Suter wurde begonnen.}{Das Hauptziel, die Erkenntnisse von Herrn Schmocker fertigzustellen, wurde erreicht. Die Arbeit an Frau Suters Interview-Daten konnte planmässig begonnen werden.}{\image{Beleg1808.png}{Die Erkenntnisse zu eine der fertiggestellten Fragen}}

% 20.08.2025
\JournalDay{20.08.2025}
\JournalEntry[true]{Implementierung von Ganztags-Events und Notenorganisation}{3h (Aryan)}{Implementierung der "All day/multi day" Event-Funktionalität im Kalender. Einen Anfang für die Notenorganisation machen.}{Die Funktionalität für ganztägige und mehrtägige Events (`All day/multi day`) wurde implementiert. Ein Grundstein für die Notenorganisation wurde gelegt.}{Ich musste ein bisschen mehr als erwartet mit KI arbeiten, weil ich mich nicht so gut in JavaScript auskenne. Aber es lief trotzdem gut.}{\gitcommit{b7a8e2d2ea1e4f96ffa8c9cad6fb89e8876cba27}}

% 28.08.2025
\JournalDay{28.08.2025}
\JournalEntry[true]{Weiterarbeit an der Notenorganisation}{1.5h (Aryan)}{Notenorganisation weiterbearbeiten und überlegen, welche weiteren Funktionen dort implementiert werden sollen.}{Die Notenorganisation ist funktional im gewünschten Umfang implementiert.}{Die Hauptfunktionalität der Notenorganisation ist erreicht, es muss jedoch noch entschieden werden, welche optionalen Features hinzugefügt werden. Das vollständige Testing der Funktion steht noch aus.}{\gitcommit{0280f3217c764a679fb8340fe647d18c4ca7d902}}

% 02.09.2025
\JournalDay{02.09.2025}
\JournalEntry[true]{Interview-Analyse Frau Suter fortgesetzt}{Ca. 2.5h (Simon)}{Weiter an Frau Suters Erkenntnissen arbeiten, um die zentralen Aussagen aus dem Interview zu erfassen und in Empfehlungen für die Applikation umzuwandeln.}{Mehr Erkenntnisse aus dem Interview mit Frau Suter gesammelt und für die Applikation aufbereitet.}{Die Aufarbeitung der Interview-Erkenntnisse schreitet gut voran und liefert wertvolle Informationen für die App-Funktionalität.}{\image{Beleg0209.png}{Zusammengefasste Antworten von Frau Suter zu einer Frage über spezifische Lernmethoden}}

% 04.09.2025
\JournalDay{04.09.2025}
\JournalEntry[true]{Interview-Analyse Frau Suter}{Ca. 1.5h (Simon)}{Weiter an Frau Suters Erkenntnissen arbeiten. Sektionen "Lerntechniken", "Lernmethoden" und Teile von "Pausenmanagement" fertigstellen. Nebendran: Interviewdokument schöner formatieren.}{Die Abschnitte zu Lerntechniken und Lernmethoden sowie die ersten beiden Unterabschnitte von Pausenmanagement sind fertiggestellt. Das Interviewdokument wurde zusätzlich optisch aufbereitet.}{Die Analyse der Erkenntnisse von Frau Suter wurde weitgehend abgeschlossen, und die Formatierung des Interviewdokuments verbessert die Lesbarkeit.}{\image{Beleg0409.png}{Erkenntnisse aus der Interviewanalyse von Frau Suter zu einer Frage im Pausenmanagement.}}

% 09.09.2025
\JournalDay{09.09.2025}
\JournalEntry[true]{Implementierung Pomodoro-Timer und Algorithmus-Grundlagen}{1.5h (Aryan)}{Implementierung des Lerntimers und Weiterentwicklung der Algorithmus-Gedanken. Für den Pomodoro-Timer wurde ein Grundgerüst von GPT geholt. Weiterhin wurden Algorithmus-Überlegungen gemacht, basierend auf folgenden Input-Daten: Agenda, separater Prüfungsplan (aus Agenda extrahieren), maximale Lerntage (7, extrem: 14). Die Priorität soll der Prüfungs-Lernzeit entsprechen (z.B. hoch heisst 10 Stunden). Pro Tag sollen alle Aktivitäten addiert werden, um die verbleibende Zeit für Lernaktivitäten zu bestimmen (24-10 = Schlaf plus Essen). Dann soll durch alle Tage vor der Prüfung gegangen werden. Wenn genug Zeit, einfach durchschnittliche Lernzeit pro Tag subtrahieren. Sonst die ganze Zeit vom Vortag subtrahieren. Wenn der Schluss innerhalb von 1h der Estimierung ist, ist es ok, sonst muss man 2 Wochen vorher anfangen. Der Algorithmus soll jedes Mal, wenn etwas geändert wird, von heute nach vorne schauen.}{Der Pomodoro-Timer ist implementiert (nach Korrektur des GPT-Grundgerüsts). Der Algorithmus für die Prüfungsplanung ist theoretisch ausgereift und bereit zur Implementierung.}{GPT gab zwar ein gutes Grundgerüst für den Pomodoro-Timer, welches aber noch geflickt werden musste. Der Algorithmus sieht theoretisch gut aus und kann nun implementiert werden.}{\gitcommit{8d4697660abb1785c911464087a7dca260afb417}}

% 10.09.2025
\JournalDay{10.09.2025}
\JournalEntry[true]{Algorithmus-Design und Prioritätensetzung}{3h (Aryan)}{Berichtserweiterungen, Algorithmus-Implementierung und Prioritätensetzung. Der Algorithmus wurde weiter überarbeitet: Die Prüfungspriorisierung soll 4 Fälle umfassen (Sehr wichtig \& viel: 5-7 Tage, 2h pro Tag; Wichtig \& guter Stoff: 5-7 Tage, 1h pro Tag; Nicht so wichtig: 3-5 Tage, 0.5-1h pro Tag; Unwichtig \& fast kein Stoff: 1-2 Tage, 0.5-1h pro Tag). Grundsätzlich sollen 7 Tage nicht überschritten werden (maximal 2 Wochen im Extremfall). Pro Tag ist ein Maximum von 3h Lernzeit akkumuliert. Funktionsweise: Rückwärts von der Prüfung zurückgehen und Zeit finden. Tags sollen implementiert werden (Schule, Freizeit, Lernzeit, Termin, ...). Verbleibende Zeit wird zu einer Variablen addiert, wenn man nicht viel Zeit hat. Ein anderer Schüler wurde nach Input gefragt.}{Ein Draft des Algorithmus liegt vor. Weitere Algorithmus-Gedanken und eine detaillierte Priorisierungsstruktur für Prüfungen wurden entwickelt. Input von einem anderen Schüler wurde eingeholt. Die Implementierung von Bericht und Algorithmus selbst, sowie die tatsächliche Prioritätensetzung (im Code) wurden noch nicht erreicht.}{Beim Implementieren fiel auf, dass die benötigte Lernzeit für verschiedene Prüfungslevel unklar ist. Eine ganze Umfrage nur dafür scheint zu viel. Der Algorithmus muss noch finalisiert werden.}{\gitcommit{65209c027b4de457c2b8aadc1e39df577180a121}}

% 17.09.2025
\JournalDay{17.09.2025}
\JournalEntry[true]{Interview-Auswertung: Pausen- und Zeitmanagement}{Ca. 1h (Simon)}{Weitere Arbeit an der Auswertung des Interviews von Frau Suter mit dem Ziel, die Abschnitte zu Pausenmanagement und Zeitmanagement fertigzustellen.}{Die Abschnitte Pausenmanagement und Zeitmanagement sind grösstenteils fertiggestellt, womit die Auswertung des Interviews fast abgeschlossen ist.}{Die Analyse ist nun nahezu fertig, was die Grundlage für die nächsten Schritte in der Umsetzung bildet.}{\image{Beleg1709.png}{Lernerkenntnisse aus der Interviewanalyse von Frau Suter}}

% 24.09.2025
\JournalDay{24.09.2025}
\JournalEntry[true]{Fertigstellung Interview-Analyse und Strukturierung der App-Empfehlungen}{Ca. 1.5h (Simon)}{Fertigstellung der Analyse von Frau Suters Aussagen. Anfang an allgemeinen, konkreten Empfehlungen für die Web-Applikation. Die Empfehlungen werden nach Kategorien geordnet (A: Daily Tipps, B: Agenda/Algorithmus, C: Pomodoro-Timer, D: Verschiedenes) und erhalten zur einfachen Referenz einen Buchstaben und eine Zahl.}{Die Interview-Analyse ist abgeschlossen und die Struktur für die konkreten App-Empfehlungen (geordnet und referenzierbar) ist erstellt.}{Die Interviews gaben viele nützliche Informationen, welche besonders für die «Daily Tipps» relevant sind. Wir haben auch persönlich viel davon gelernt.}{\image{Beleg2409.png}{Anforderungen für unsere Agenda / den LZA}}

% 25.09.2025
\JournalDay{25.09.2025}
\JournalEntry[true]{Abschluss der Interview-Auswertung und Berichtergänzung}{Ca. 3h (Simon: 2.5h, Aryan: 0.5h)}{Simon: Fertigstellung der Auswertung, indem die konkreten Empfehlungen fertiggestellt wurden. Aryan: Ergänzung des Berichts.}{Das Dokument mit den konkreten Empfehlungen ist abgeschlossen und fertig formatiert. Die Interview-Auswertung ist somit komplett und für zukünftige Referenz fertiggestellt.}{Die Interviews gaben viel Material für die Daily Tipps, aber auch Nützliches für den Algorithmus und den Pomodoro-Timer. Das Dokument ist abgeschlossen und formatiert, somit kann die Arbeit an der Umfrage-Analyse richtig anfangen. Man kann auch viel daraus persönlich lernen.}{\image{Beleg2509.png}{Ein paar der Anforderungen / Empfehlungen für die Lerntipps unserer Web-Applikation}}

% 26.09.2025
\JournalDay{26.09.2025}
\JournalEntry[true]{Beginn der Umfrage-Auswertung}{Ca. 1h (Simon)}{Dokument zur Auswertung der Umfrage erstellen und einrichten. Die ersten Schlüsse und Daten aus der Zeit vor den Sommerferien wurden erneut gesichtet.}{Das Auswertungsdokument wurde aufgestellt und die Excel-Datei mit den Rohdaten aus Microsoft Forms wurde bearbeitet.}{Die Arbeit an der Umfrage-Auswertung wurde begonnen und die notwendigen Daten und Strukturen dafür vorbereitet.}{\image{Beleg2609.png}{Beginn der Umfrage-Analyse.}}

% 01.10.2025
\JournalDay{01.10.2025}
\JournalEntry[true]{Vorarbeit für die Algorithmus-Implementierung}{5h (Aryan)}{Kommentare im Code hinzugefügt und den Code optisch aufbereitet (Verschönerung, auch mit black). Die Logik für die Priorität von Events wurde überarbeitet und implementiert: Rot (Sehr wichtig), Orange (Wichtig, braucht aber noch etwas), Grün (Relativ unwichtig), Farbe frei wählbar (Keine Prüfung). Die Farben sind vom Nutzer wählbar. Einstellungen (`Settings`) und ein detaillierter Pseudo-Code für den Algorithmus wurden erstellt.}{Der Code wurde verschönert und kommentiert. Die Logik für die Event-Priorität ist implementiert. Detaillierte Settings und der Pseudo-Code für den Lern-Algorithmus sind fertiggestellt.}{Die Code-Verschönerung und die Prioritäts-Logik liefen gut. Der eigentliche Algorithmus und die Notenorganisation wurden noch nicht implementiert, aber die Vorarbeit dazu ist abgeschlossen. Gelernt wurde nicht viel Neues, es war primär eine Anwendung von Flask wie gewohnt.}{\gitcommit{428e45f39b890086443a2b707e4ea9c44757f71c}}

% 02.10.2025
\JournalDay{02.10.2025}
\JournalEntry[true]{Algorithmus-Implementierung (Codierung)}{4h (Aryan)}{Weiterarbeit am Algorithmus mit dem Ziel, den Pseudo-Code in tatsächlichen Python-Code zu übersetzen.}{Der Algorithmus wurde codiert, ist aber noch nicht implementiert (d.h. er existiert im Code, ist aber noch ohne Nutzung). Der Algorithmus konnte erfolgreich von Pseudo-Code nach Python umgewandelt werden.}{Die Umwandlung des Algorithmus von Pseudo-Code nach Python verlief gut. Für das nächste Mal ist das Testen und die eigentliche, funktionierende Implementierung des Algorithmus geplant.}{\image{Beleg0210.png}{Pseudo-Code vor der Umwandlung.}}

% 03.10.2025
\JournalDay{03.10.2025}
\JournalEntry[true]{Implementierung und Debugging des Lern-Algorithmus}{4h (Aryan)}{Der Algorithmus wurde implementiert. Es wurden Helfer-Funktionen für den Algorithmus erstellt. Ausserdem kann der User jetzt frei Prioritäten hinzufügen und löschen.}{Der Lern-Algorithmus ist implementiert. Der User kann Prioritäten für Lern-Events festlegen und diese nach Bedarf anpassen. Settings sind jetzt in einer separaten Datenbank für Übersichtlichkeit.}{Die Implementierung des Algorithmus verlief gut, jedoch traten einige Bugs auf, die noch behoben werden müssen.}{\gitcommit{4985619796ae89945db1d4aa9deecde6e1936fa5}}

% 04.10.2025
\JournalDay{04.10.2025}
\JournalEntry[true]{Test und Feinabstimmung des Lern-Algorithmus}{3h (Aryan)}{Der Algorithmus wurde weiter getestet und angepasst. Die Logik, wie der Algorithmus die Zeit findet, musste korrigiert werden.}{Der Algorithmus ist nun funktionstüchtig und findet freie Zeitblöcke korrekt, um Lern-Events einzutragen.}{Der Algorithmus funktioniert jetzt, nachdem es mit den Test-Cases einige Probleme gab. Die Logik, wie freie Zeit gefunden wird, musste angepasst werden.}{\gitcommit{5b509cf3f6317513a5c5d21ff45ede8edc17b627}}

% 05.10.2025
\JournalDay{05.10.2025}
\JournalEntry[true]{Lokales \LaTeX-Setup und Berichtsstart}{4h (Aryan)}{Installation von \textbf{TeXLive} auf dem lokalen Rechner gestartet. Weitere Ergänzungen vom Bericht. Versuch der Code-Verschönerung (Kommentare, Lesbarkeit).}{Installation von TeXLive initiiert. Wenige Ergänzungen am Bericht geschafft, das Quellenverzeichnis ist nicht so einfach. Code-Verschönerung nicht erreicht.}{Die Installation von TeXLive (\LaTeX) war sehr zeitaufwendig ($\sim$1.5h) und konnte wegen der langen Zeit noch nicht mit Visual Studio Code aufgestellt werden. Das Quellenverzeichnis erscheint schwierig.}{\image{05102025.png}{TeXLive in Visual Studio Code}}

% 06.10.2025
\JournalDay{06.10.2025}
\JournalEntry[true]{Implementierung der Notenorganisation und Umfrage-Auswertung}{Ca. 4.5h (Aryan: 3h, Simon: 1.5h)}{Aryan: Abschluss der Notenorganisation. Der Code wurde verschönert mithilfe KI, und am Bericht wurde weitergeschrieben. Simon: Die Umfrage-Auswertung wurde fortgesetzt, mit Fokus auf der Ausarbeitung des Abschnitts Lernmethoden.}{Aryan: Die Notenorganisation ist nun vollständig implementiert. Simon: Der Abschnitt zur Auswertung der Lernmethoden aus der Umfrage wurde fertiggestellt.}{Aryan: Die Programmierung lief besser als erwartet. Simon: Die Umfrage-Resultate sind sehr nützlich, müssen aber noch mit den Interview-Ergebnissen abgeglichen werden, da es leichte Diskrepanzen gab.}{\gitcommit{178139f7f543bf98f906af03f0ff531cebc0628a}\image{Beleg0610.png}{Analyse der Fragen über das Lernverhalten.}}

% 07.10.2025
\JournalDay{07.10.2025}
\JournalEntry[true]{Zusammenstellung KI-Nachweis und Umfrage-Auswertung}{Ca. 3.5h (Aryan: 2h, Simon: 1.5h)}{Aryan: Den KI-Nachweis für die Maturitätsarbeit zusammengestellt. Simon: Weiterarbeit an der Umfrage-Auswertung und Ausarbeitung der Abschnitte Pausen und Zeitmanagement.}{Aryan: Der KI-Nachweis wurde erstellt und die Verwendung von KI-Tools im Projekt dokumentiert. Simon: Die Abschnitte zur Auswertung von Pausen und Zeitmanagement aus der Umfrage wurden fertiggestellt.}{Die Arbeit an der Umfrage-Auswertung ist nun weit fortgeschritten. Mit dem Erstellen des KI-Nachweises ist ein wichtiger formaler Bestandteil der Dokumentation erledigt.}{\gitcommit{50cfaadb04e5cbf6ae203a007efac9fa029a4d88}\image{Beleg0710.png}{Analyse der Fragen zum Pausen- und Zeitmanagement}}

% 08.10.2025
\JournalDay{08.10.2025}
\JournalEntry[true]{Abschluss der Umfrage-Analyse und Übertragung des Arbeitsjournals}{Ca. 3.5h (Simon: 1.5h, Aryan: 2h)}{Simon: Die Umfrage-Analyse wurde abgeschlossen, indem einige Erkenntnisse \& ein paar konkrete Empfehlungen hinzugefügt wurden. Ebenso wurden die (seriösen) Tipps der Schüler:innen aus der Umfrage übertragen. Aryan: Alle bisherigen Arbeitsjournal-Einträge auf ein \LaTeX-Dokument übertragen}{Simon: Die Umfrage-Analyse ist nun fertiggestellt und kann so als Referenz gebraucht werden. Aryan: Arbeitsjournal wurde aufgearbeitet und in \LaTeX\space jetzt eingefügt, mit einem einfach wiederverwendbaren Template.}{Simon: Die Analyse der Umfrage begründet \& validiert unsere Web-Applikation sehr gut. Sie zeigt auf u.a. dass viele Schüler nur sehr spät mit dem Lernen anfangen, oft Prüfungsstress haben und auch an so einer Lernapp interessiert wären. Leider kam die fertiggestellte Analyse erst sehr spät, aber die Resultate waren auch ohne die formelle Analyse schon relativ sichtbar. Aryan: In \LaTeX\space ist die Formattierung sehr einfach, und in Zukunft wird das Erstellen von neuen Einträgen auch sehr einfach sein.}{\gitcommit{144429aa23d2e7dd27ba7044053e717a66e7d79f}\image{Beleg0810.png}{Ein paar Empfehlungen / Anforderungen aus der Umfrage-Analyse}}

% 09.10.2025
\JournalDay{09.10.2025}
\JournalEntry[true]{Arbeitsjournal Korrektur, Lerntipps-Seite und Arbeit am Bericht}{Ca. 4.5h (Simon: 2.5h, Aryan: 2h)}{Aryan: Das Arbeitsjournal wurde auf Fehler überprüft und korrigiert. Die Lerntipps-Seite wurde erstellt. Simon: Am Bericht weitergearbeitet, ein paar Tipps erstellt basiert auf die Interview-Analyse}{Aryan: Korrigiertes Arbeitsjournal und eine neue Seite für Lerntipps sind fertiggestellt. Simon: Weitere Arbeit am Recherche-Teil des Berichts, u.a. an den Büchern geschrieben, und die ersten Lern- und Daily Tipps ergänzt.}{Aryan: Die Korrektur des Arbeitsjournals verlief gut, und die Erstellung der Lerntipps-Seite ist ein wichtiger Schritt zur Fertigstellung der Webseite. Der Inhalt wird bald ergänzt. Simon: Die Arbeit am Bericht ging gut, hatte ein wenig mehr Zeit mit Syntax verschwendet als gewollt wegen einer unpassenden Skalierung der Bilder. Das Schreiben der ersten Tipps verlief gut, ich konnte dafür auch die Interview-Analyse zum Gebrauch nehmen.}{\gitcommit{a9f54c7b1d157fe32dd19df5f7c6c4329b33fb08}}

% 10.10.2025
\JournalDay{10.10.2025}
\JournalEntry[true]{Serverhosting und Arbeit am Bericht}{Ca. 7h (Aryan: 2h, Simon: 5h)}{Aryan: Hosting auf Railway.app eingerichtet. Mit KI habe ich herausgefunden, welche Serveroptionen es gibt. Simon: Weiter am Bericht geschriben - Fokus auf die Recherche}{Aryan: Die Web-Applikation läuft nun auf Railway.app. Simon: Der Teil des Berichts über die Internet Recherche ist praktisch fertig, habe am Segment über die Interviews angefangen.}{Aryan: Die Einrichtung des Hostings verlief gut, obwohl es einige Herausforderungen gab. Ausserdem werden SMTP Mails geblockt von Railway, also muss ich dazu noch eine Lösung finden. Simon: Die Arbeit am Bericht verlief gut, Quellenmanagement kann aufwendig sein, vor allem wenn der Teil, über den man schreibt, schon vor Monaten gemacht wurde.}{\gitcommit{c63f8579fe316802c545f4328934c0eddc7828e9}}

% 11.10.2025
\JournalDay{11.10.2025}
\JournalEntry[true]{Server-Migration, E-Mail-Lösungsansatz und Arbeit am Bericht}{Ca. 11.5h (Aryan: 8h, Simon: 3.5h)}{Aryan: Vorbereitung und Start der Server-Migration der Web-Applikation zu DigitalOcean. Dazu gehörte die Einrichtung des neuen App Platform-Dienstes und die Überprüfung der DNS-Einstellungen für die Domain \texttt{kantikoala.app}. Weiterhin die Evaluierung verschiedener E-Mail-API-Anbieter (z.B. Resend, Postmark) als notwendiger Ersatz für den blockierten SMTP-Port des aktuellen Hosters und da wir dann unsere eigene Domain für den E-Mail brauchen können. Simon: Weitere Arbeit am Interviewteil des Berichts}{Die DNS-Einstellungen für den Wechsel zu DigitalOcean sind geprüft und die neue Umgebung ist vorbereitet. Ich habe eine klare Vorstellung der notwendigen Schritte für die Wiederherstellung der E-Mail-Funktionalität durch eine API-Lösung. Simon: Hauptsächlich am 'Interviewfragebogen' Segment gearbeitet, dies kommt gut voran.}{Die Einrichtung des Servers auf DigitalOcean war sehr schwierig, ich fand fast keine Informationen dafür. Ich musste also vieles austesten, bis es funktionierte, und das dauerte sehr lange. Simon: Arbeit geht gut voran.}{\gitcommit{ee80ed9cf814d0264176f7ccb0310c00bcc6a39d}}

% 20.10.2025
\JournalDay{20.10.2025}
\JournalEntry[true]{Tailwind CSS Installation}{Ca. 2h (Simon)}{Installation und Konfiguration von Tailwind CSS und anderen benötigten Modulen, damit Arbeit am Design der Web-Applikation fortlaufen kann.}{Nun kann mit Tailwind CSS gearbeitet werden, welches ein optimierteren Framework für CSS darstellt.}{Die Installation verlief hauptsächlich gut, und ich musste noch einige JS-Module installieren, damit dies alles funktionierte. Ebenso präsentierten sich leider einige Schwierigkeiten, die Web-Applikation auf meinem Laptop laufen zu lassen, welche ich noch ausbügeln muss.}{\gitcommit{bf0f70481fa07ee0282cc669e50d0bf09ad3f361}}

% 21.10.2025
\JournalDay{21.10.2025}
\JournalEntry[true]{Deployment Fix, UI-Arbeit, E-Mail-Integration und CSRF-Schutz}{Ca. 6.5h (Aryan: 3.5h, Simon: 3h)}{Aryan: Behebung eines Problems mit den Requirements im Deployment-Prozess. Integration der E-Mail-API von Resend als schnellen Lösungsansatz. Zusätzlich die Einrichtung des CSRF-Schutzes (Cross-Site Request Forgery) für die Applikation, unter Zuhilfenahme von Gemini. Simon: Erste preliminäre Arbeit an der Homepage-UI der Website}{Aryan: Der Deployment-Fehler wurde behoben. Die E-Mail-Funktionalität über Resend ist integriert, funktioniert und war einfach einzurichten. Der grundlegende CSRF-Schutz ist nun aktiv. Simon: Ein erstes Design für Homepage eingesetzt \& ein einfaches (wenn auch sehr farbloses) Farbschema ausgewählt, welches mit TailwindCSS vordefiniert wurde. }{Aryan: Das Problem mit den Deployment-Requirements war ein unerwarteter Stolperstein. Die Einrichtung des CSRF-Schutzes erforderte Recherche, wobei Gemini bei der Implementierung unterstützte. Simon: Ich musste mich weiterhin noch vertraut machen mit TailwindCSS, die Implementation ist gewöhnungsbedürftig, u.a. besonders mit Themen wie Elemente positionieren, welches wohl noch einiges an Übung verlangt}{\gitcommit{98e36e36d9ae92fc7be088c4b0956a078a8f67ec}}

% 22.10.2025
\JournalDay{22.10.2025}
\JournalEntry[true]{Weiterarbeit am Bericht und UI-Arbeit}{Ca. 5h (Aryan: 2h, Simon: 3h)}{Aryan: Schrieb weiter am Bericht, ergänzte die Features, die noch nicht drin waren. Simon: Weiterhin an der UI gearbeitet, dynamisches Resizing}{Aryan: Der Bericht wurde um die fehlenden Features ergänzt. Simon: Ich habe weiterhin an der UI für die Homepage und der Login / Registrations-Page gearbeitet. Auch habe ich mich heute darauf geachtet, wie die Elemente sich bei Resizing vom Bildschirm verhalten und mithilfe von Breakpoints dynamische Design-Änderung, vor allem für Mobile-Screens integriert. Somit bekommt die Website bspw. ein neues Layout wenn der Bildschirm klein genug ist, welches mehr für Handys ausgelegt ist.}{Aryan: Die Arbeit am Bericht verlief gut, ich konnte die fehlenden Teile effizient hinzufügen. Simon: Die Arbeit an der UI verlief grundsätzlich sehr gut. Dank Flask's debug mode kann ich Änderungen sehr schnell vornehmen und testen.}{\gitcommit{e3585dbb83d54b6a9749cdcafc14816c48e2f901}}

% 23.10.2025
\JournalDay{23.10.2025}
\JournalEntry[true]{Testing der Applikation}{Ca. 6.5h (Aryan: 4h, Simon: 2.5h)}{Aryan: Ich erstellte die manuellen QA-Tests für die Applikation. Dabei testete ich alle Features durch und dokumentierte die Resultate. Simon: Weitere Arbeit am UI}{Aryan: Die manuellen QA-Tests sind abgeschlossen und dokumentiert. Simon: Die Homepages und Login / Registrations-Pages sind voraussichtlich in diesem ersten Pass fertig, und ich habe begonnen, mich auf die spezifischen Feature-Pages zu fokussieren, beginnend mit den Einstellungen.}{Aryan: Das Testing verlief gut, ich konnte alle Features gründlich überprüfen und die Ergebnisse klar festhalten. Simon: Die UI-Arbeit verläuft weiterhin gut. Die Einstellungen-Page ist bisher die komplexeste, u.a. wegen der Option, mehrere Priorities für die Agenda hinzuzufügen / entfernen. Dies erfordert ein noch dynamischeres Layout als die Homepage.}{\gitcommit{8dbe973583142d3f28a7cfaaa21a02e83926b7a9}}

% 24.10.2025
\JournalDay{24.10.2025}
\JournalEntry[true]{Implementierung von Best Practices in der Applikation, weitere Arbeit am UI Design}{Ca. 5h (Aryan: 3h, Simon: 2h)}{Aryan: Ich habe den Code aufgespalten in verschiedene Dateien, damit es jetzt alles in einem Application Factory Pattern läuft. Dies ist eine Best Practice für Flask Applikationen. Simon: Ich habe weiter an der Login- und Registrierungspage UI gearbeitet}{Aryan: Der Code wurde erfolgreich in ein Application Factory Pattern umgewandelt, was die Struktur der Applikation verbessert, und welches ein Best Practice ist. Simon: Die Login- und Registrierungspages kommen gut voran, sind jetzt hauptsächlich komplett.}{Aryan: Die Umstellung bereitete einige Schwierigkeiten, da viele Importe und Abhängigkeiten angepasst werden mussten. Zudem habe ich noch nie mit so einer Struktur gearbeitet, also gab es immer wieder Fehler. Simon: Arbeit am UI Design läuft gut, vor allem da ich jetzt langsam vertraut werde mit TailwindCSS}{\gitcommit{725d3c0f4ae81ee70795788978e6ee43b4e58e7e}\gitcommit{e86c95da20a8f68c6badc4f84da4c471af3bd44f}}

% 25.10.2025
\JournalDay{25.10.2025}
\JournalEntry[true]{"Forgot Password" UI}{Ca. 3h (Simon)}{Arbeit am UI Design für die "Forgot Password"-Page}{Ich konnte die "Forgot Password"-Page erfolgreich praktisch fertigstellen und auch noch ein paar kleine Änderungen an der "Register"-Page machen, ebenso habe ich einige Begriffe auf der Website auf Deutsch übersetzt, damit dies einheitlicher dargestellt wird.}{Die Arbeit am Design verläuft weiterhin gut.}{\gitcommit{1f6c07ea900ac948ad420e84a8975aa49ee0d7ec}}

% 27.10.2025
\JournalDay{27.10.2025}
\JournalEntry[true]{Arbeit an der Settings-UI \& Migration von UI Design }{Ca. 3.5h (Simon)}{Ich habe an der UI für die Settings-Page gearbeitet und die HTML-Dateien zum neuen Directory migriert.}{Settings-Page ist gut vorangekommen, aber noch nicht fertig.}{Die Settings-Page erweist sich als überraschend schwierig wegen dem dynamischen Layout der "Lernprioritäten", aber ich denke ich habe eine gute Lösung gefunden.}{\gitcommit{6440de9b5764c37cce9686121eed5ce7491b19fc}}

% 29.10.2025
\JournalDay{29.10.2025}
\JournalEntry[true]{Testing der Applikation}{Ca. 2h (Aryan)}{Anhand unseren manuellen Tests habe ich unsere Webapplikation neu getestet, da die ganze Codestruktur sich verändert hat. Nicht alle Tests wurden bestanden.}{Neues Testing der Applikation. Mehrere Tests nicht erfolgreich, jedoch konnten viele auch gerade geflickt werden.}{Das neue Testing war eine gute Idee, da eben viele Sachen nicht mehr funktionierten nach der Umstellung auf das Application Factory Pattern. Ich konnte viele Bugs fixen, aber einige sind noch offen.}{\gitcommit{d46f0decfc87881ff7855ea8401e731231ac4567}}

% 30.10.2025
\JournalDay{30.10.2025}
\JournalEntry[true]{Arbeit am Bericht}{Keine genaue Zeitangabe (Simon)}{Weitere Arbeit am Recherche-Teil, vor allem am Segment der Interviewfragen.}{Stressmanagement-Sektion der Interviewfragen ist fertig}{Es ist schwierig zu entscheiden, was für den Bericht relevant ist und nicht. Das Schreiben des Recherche-Teils stellt sich als überraschend schwierig heraus.}{\image{Beleg-30.10.png}{Arbeit an der Interviewfragen-Sektion des Berichts}}

% 31.10.2025
\JournalDay{31.10.2025} 
\JournalEntry[true]{Berichtergänzung vom Programmieren}{Ca. 3.5h (Aryan)}{Ich habe die Abschnitte zu der Datenbankstruktur und der Codestruktur ergänzt. Zusätzlich auch noch die Sicherheitsmassnahmen. Dabei hat mir KI mit dem Schreiben geholfen. Zudem habe ich im \LaTeX formale Anforderungen noch ergänzt.}{Erfolgreiche Ergänzung des Berichts. Formale Angaben sind erfüllt.}{Sowohl das Schreiben wie auch die formalen Anpassungen liefen gut.}{\gitcommit{2191711502f77994aa2d610c5f1b3087b6768567}}

% 02.11.2025
\JournalDay{02.11.2025}
\JournalEntry[true]{Bugfixing und Bericht-Arbeit}{Ca. 5h (Aryan)}{Weiteres Bugfixing basierend auf dem letzten Testing, vorallem mit dem Algorithmus gab es Probleme. Am Bericht habe ich auch noch weitergeschrieben (mit Hilfe von KI).}{Der Algorithmus funktioniert wieder wie erwartet. Der Bericht kommt gut voran.}{Das Bugfixing verlief gut, ich konnte die Probleme mit dem Algorithmus beheben. Die Arbeit am Bericht ging ebenfalls zügig voran. Jetzt muss ich wieder alles testen.}{\gitcommit{bd81cb73d9205df04d684b8e3791ef2cac608b4c}}

% 04.11.2025
\JournalDay{04.11.2025}
\JournalEntry[true]{Bericht-Arbeit}{Ca. 7h (Aryan: 4h, Simon: 3h)}{Aryan: Am Bericht habe ich auch noch weitergeschrieben (mit Hilfe von KI). Hinweise von Herr Schneider berücksichtigt. Simon: Aufgrund der Hinweise von Herr Schneider habe ich angefangen, die Recherche umzuformulieren.}{Aryan: Der Programmier-Teil des Berichts ist grob abgeschlossen. Simon: Der Anfangsabschnitt der Recherche ist fertig.}{Aryan: Es lief heute gut mit dem Schreiben, und es fehlt nicht mehr viel vom Programmier-Teil. Ich muss einfach noch alle KI-Nachweise angeben, und dann halt noch die Schlussfolgerung schreiben. Simon: Dank des Feedbacks von Herr Schneider, auch wenn es ein wenig hart ist da nun vieles der bisherigen Arbeit daran neu geschrieben werden muss, ist sehr hilfreich.}{\gitcommit{01cca9a0bcfada94a2fe53b67db231be9cc2de74}\gitcommit{225dfa025d4876b4cdf0a643efb3a5cb3a28e51e}}

% 05.11.2025
\JournalDay{05.11.2025}
\JournalEntry[true]{Weiterarbeit am Bericht}{Ca. 3.5h (Aryan: 3h, Simon: 0.5h)}{Aryan: Ich schrieb weiter am Bericht, setzte Bermerkungen von Herr Schneider um von der heutigen Sitzung. Simon: Ich schrieb auch ein wenig weiter am Bericht, vor allem am Recherche-Segment.}{Aryan: Der Bericht sieht nun gut aus, aber weit von fertig. Simon: Ich habe weiter an der Internet-Recherche gearbeitet.}{Aryan: Es gibt noch einige offene Punkte, die ich klären muss. Jedoch fehlt eigentlich nichts grosses, sondern ein bisschen umschreiben und KI Nachweise. Simon: Aufgrund meiner Militär-Rekrutierung hatte ich nicht viel Zeit, um weiter zu arbeiten, aber dennoch konnte ich ein wenig an der Recherche arbeiten. Es kommt voran.}{\gitcommit{f3e5aef663aaa4760d2ff1d8dc75f176356af72c}\image{Beleg-5.11.png}{Arbeit an der Internet-Recherche des Berichts}}

% 06.11.2025
\JournalDay{06.11.2025}
\JournalEntry[true]{Bericht-Arbeit und Bug-Fixing}{Ca. 8h (Aryan: 4h, Simon: 4h)}{Aryan: Ich habe den Bericht fast ganz fertiggestellt, d. h. ich habe die Teile umgeschrieben, die noch umgeschrieben werden mussten, inklusive aller KI-Nachweise und der Schlussfolgerung. Jedoch muss man das alles nochmals durchlesen, und eventuell noch Änderungen vornehmen. Es gab noch einige kleinere Bugs, die ich flicken konnte. Simon: Ich habe weiter am Bericht gearbeitet, vor allem der Literaturstudie.}{Aryan: Der Bericht ist nun fast fertig. Die kleineren Bugs sind geflickt. Simon: Ich habe weiter am Recherche-Segment des Berichts gearbeitet, vor allem an der Literaturstudie. Diese kommt gut voran und ich kann bald am Feedback für die Interviewfragen arbeiten.}{Aryan: Die Fertigstellung des Berichts verlief gut, ich konnte alle offenen Punkte klären und den Bericht soweit wie möglich abschliessen. Das Bug-Fixing lief ebenfalls gut, ich konnte alle gefundenen Bugs beheben. Simon: Mein Segment des Berichts kommt gut voran, auch wenn das Schreiben ein wenig schwierig ist.}{\gitcommit{851b349318dbd905ec4d51ca5718947064baff1f}\gitcommit{2c285adeb22b951c3295a013e8ab000acde34006}}

% 07.11.2025
\JournalDay{07.11.2025}
\JournalEntry[true]{Berichtserweiterungen/Tests, Code-Anpassungen, Design und Lerntipps}{Ca. 17h (Aryan: 9h, Simon: 8h)}{Aryan: Ich habe weitere Tests eingefügt, die man noch machen muss. Zudem habe ich noch den schriftlichen Kommentar ein bisschen poliert. Am Code gab es ein paar unvollständige Sachen mit den default Farben, die ich noch anpassen musste. Schliesslich habe ich noch den ganzen Design für die Seite mithilfe von KI erstellt. Simon: Ich habe weiterhin am Bericht gearbeitet und einige Lerntipps aufgeschrieben basiert auf den Empfehlungen.}{Aryan: Bericht ist schöner, es hat mehr Tests, und die default Farben der Erreignisse sind jetzt auch noch veränderbar (und in der Dokumentation begründet). Das Design für die Webseite ist inkl. Dark-Mode fertig. Simon: Der Interview-Bereich der Recherche im Bericht ist (beinahe) fertig, weitere Lerntipps wurden hinzugefügt.}{Aryan: Die Arbeit an den Tests und der Code-Anpassungen verlief gut. Das Design sieht gut aus, ich müsste evtl. noch die Farbwahl begründen. Simon: Ich bin zufrieden, konnte meine Schreibblockade überwinden und ein paar Lerntipps schreiben \& an meinem Bereich des Berichts viel schreiben.}{\gitcommit{93d4a97011a9d3edd5fe7972654dd2c767cdfeaa}\gitcommit{701dcf5992a6a4d70195c3ad4c5f5c9bd52ca872}}

% 08.11.2025
\JournalDay{08.11.2025}
\JournalEntry[true]{Usability-Tests, Bericht-Arbeit, Lerntipps und Codeanpassungen}{Ca. 15h (Aryan: 7h, Simon: 8h)}{Aryan: Ich habe die Usability-Tests geschrieben und habe auch gleich die Rückmeldungen bekommen. Ich musste noch ein paar Design Anpassungen machen und habe schon mal angefangen ein paar Rückmeldungen umzusetzen. Am Bericht habe ich auch noch weitergeschrieben. Simon: Ich habe viele Lerntipps aus den Interview- und Umfrage-Analysen integriert und am Recherche-Segment des Berichts gearbeitet.}{Aryan: Usability-Tests sind geschrieben und erste Rückmeldungen sind eingearbeitet. Bericht ist weitergeschrieben. Der Code ist soweit eigentlich fertig. Simon: Die Arbeit an der Recherche im Bericht kommt gut voran. Ich habe etliche (Lern-)tipps für die "Lerntipps" Seite geschrieben, welche hauptsächlich aus der Interview- und der Umfrage-Analyse stammen, dabei bspw. eine detailierte Erklärung der Matura-Noten.}{Aryan: Die Usability-Tests verliefen gut, ich konnte einige nützliche Rückmeldungen sammeln und bereits einige davon umsetzen. Die Arbeit am Bericht ging ebenfalls gut voran. Simon: Die Arbeit kommt gut voran.}{\gitcommit{7cce95944700455f9c1c80691b7dfa7255b4b642}\gitcommit{8453725b40cafd97a0fc68af567d4b42692b09f6}}

% 09.11.2025
\JournalDay{09.11.2025}
\JournalEntry[true]{Abschluss der Usability-Tests, Bericht-Arbeit und Zusätzliche Features}{Ca. 20h (Aryan: 9h, Simon: 11h)}{Aryan: Ich habe alle Rückmeldungen der Usability-Tests erhalten und die letzten Rückmeldungen eingearbeitet. Zudem schrieb ich noch alle fehlenden Bericht-Teile fertig. Bei About-Us habe ich noch ein Easter-Egg eingebaut, auf dem Home-Screen ein schönes Dashboard, und noch eine Hilfsseite (dank den Rückmeldungen) Simon: Ich habe, bis auf kleine Additionen an gewissen Teilen ausserhalb der Recherche, meinen Teil des Berichts fertiggestellt und angefangen, gründlich zu korrigieren / verbessern. Ebenso habe ich angefangen, Feedback zu Aryan's Bericht-Segmente zu geben.}{Aryan: Home-Screen Dashboard und Hilf-Seite implementiert. Bericht ergänzt. Simon: Bericht ergänzt, Korrektur \& Feedback für Aryan angefangen.}{Aryan: Die Usability-Tests verliefen gut, ich konnte alle Rückmeldungen einarbeiten und die Applikation weiter verbessern. Der Bericht ist nun vollständig. Simon: Mein Teil vom Bericht ist nun inhaltlich vollständig und muss nur noch fertig korrigiert werden.}{\gitcommit{d98040d5f3ce1368d9796e43b3d92ddec24dff30}\gitcommit{bcc1964311c9aa46e28c1933bb9e62f65147fb80}}

% 10.11.2025
\JournalDay{10.11.2025}
\JournalEntry[true]{Abschlussbericht Korrektur und Feedback, App-Ergänzungen, Lerntipps, QoL}{Ca. 20h (Aryan: 10h, Simon: 10h)}{Aryan: Ich habe den Bericht nochmals komplett durchgelesen und Korrekturen vorgenommen. Schlussfolgerung und Reflexion neugeschrieben/revidiert. Bei der App eine Todoliste hinzugefügt, nach Empfehlung und Datenschutzerklärung. Diese dann auch im Bericht ergänzt. Simon: Ich habe den Recherche-Teil des Berichts korrigiert, weitere Lerntipps eingefügt, einige Bugs gefunden und QoL-Vorschläge gemacht.}{Aryan: Der Bericht ist nun fertig korrigiert und bereit für die Abgabe (nach Simons Ergänzungen). Die App ist soweit fertig. Simon: Meine Segmente zum Bericht sind korrigiert und fertig, ich muss jetzt nur noch einige wenige Sachen ergänzen. Ich habe alle Lerntipps, welche von den Umfrage-Anforderungen empfohlen wurden, ergänzt und korrigiert. Ebenso habe ich mir Zeit genommen, um einige QoL-Vorschläge zu machen und, so gut wie möglich, ein paar bugs zu finden. Ich machte sogar ein paar kleinere QoL-Änderungen selbst.}{Aryan: Die Ergänzungen und Revidierungen verliefen gut, die Additionen zur App auch. Ich muss morgen vor der Abgabe alles nochmals durchschauen, um zu sehen ob etwas übrig bleibt. Simon: Ich muss nur noch ein paar Bereiche des Berichts etwas ergänzen, und dann ist der Bericht vollständig und abgabebereit. Mir bleibt daneben nur noch eine grössere Aufgabe übrig, nämlich ein paar Tipps für den sicheren Gebrauch von KI in der Schule aufzuschreiben.}{\gitcommit{78682d88316ab884a0e50a97f589a32d6a443153}\gitcommit{ba5f1698c110baf8c8034c92c3af7d5275962867}}

\JournalDay{11.11.2025}
\JournalEntry[true]{Abschluss der Maturaarbeit}{Ca. 8h (Aryan: 3h, Simon: 5h)}{Aryan: Ich habe den Bericht nochmals komplett durchgelesen und Korrekturen vorgenommen. Den KI-Nachweis, sowie die funktionellen Tests ergänzt. Simon: Ich habe alle meiner Arbeitsjournaleinträge mit den nötigen Belegen ausgestattet und die Maturaarbeit auf die Abgabe vorbereitet.}{Es ist alles abgegeben!}{Aryan: Ich finde die Arbeit verlief heute gut, etwas Stress gab es aber natürlich schon. Simon: Ich stimme Aryan zu, aber wir wurden gut fertig und ich bin mit dem Zustand der Arbeit sehr zufrieden.}{\gitcommit{127ae0368bd0325c67803cd4565c121b1fb767f4}}

\textbf{Bermerkung}\\
Die Belege haben oft Git Commit SHAs. Diese sollten auch verlinkt sein, das heisst, man sollte sie auch klicken können, um zum entsprechenden Commit auf GitHub zu gelangen. Falls das nicht automatisch funktioniert, müsste man einfach diesen Link vor dem SHA einfügen: \url{https://github.com/aryan/lernapp/commit/}. Zusätzlich ist noch wichtig, dass der Commit, welcher gelinkt ist, meistens nicht der einzige Commit ist, an dem jeweiligen Tag. Es stellt nur einen repräsentativen Commit dar. Auf unserem GitHub-Repository sind alle Commits vorhanden, wo man dann auch die anderen Commits vom Tag sehen kann.
\vspace{1cm}
\hrule

\section*{Standortbestimmungen}

% 19.03.2025
\StandortHeader{19.03.2025}
\subsubsection*{Was haben wir heute erreicht?}
\paragraph{Simon}
\begin{itemize}
    \item Alle Fragen für die PPP-Lehrpersonen überlegt und einen ersten Draft für die E-Mail geschrieben.
    \item Weitere Arbeit am Arbeitsvertrag.
    \item Consulting bei Lizenzwahl.
\end{itemize}

\paragraph{Aryan}
\begin{itemize}
    \item Aufstellung des Github-Repositories + Directory tree darin für Flask.
    \item Login/Register Authentication in Flask mit SQL-Alchemy (Email, Username, Passwort für Register, nur Username und Passwort für Login).
    \item Settings page, momentan nur Account deletion button.
    \item Account deletion.
\end{itemize}

\subsubsection*{Reflexion}
Wir sind gut vorangekommen, mittlerweile noch gut im Zeitplan. Basic Account Management features waren einfacher aufzustellen, als erwartet.

\subsubsection*{Totaler Zeitaufwand}
Je 3 Lektionen
\StandortFooter

% 21.03.2025
\StandortHeader{21.03.2025}
\subsubsection*{Was haben wir heute erreicht?}
\begin{itemize}
    \item Lagebesprechung mit Herr Schneider, Theorie-Dokumente für Interviews \& Umfragedesign gesammelt.
    \begin{itemize}
        \item Konkretes Feedback zum Arbeitsvertrag.
    \end{itemize}
\end{itemize}
\StandortFooter

% 02.04.2025
\StandortHeader{02.04.2025}
\subsubsection*{Was haben wir heute erreicht?}
\begin{itemize}
    \item Einen Interviewfragebogen mit \LaTeX\ erstellt, welchen wir nach einer Zusage zum Interview den Lehrpersonen schicken können.
    \item Am Vertrag gearbeitet, warten nun auf Feedback von Herr Schneider.
    \item Mehr Recherche um Themen, welche wichtig für das Interview \& die Umfragen sind.
\end{itemize}

\subsubsection*{Reflexion}
\begin{itemize}
    \item Wir haben gemerkt, dass Umfragedesign nicht im Interview vorkommen muss, da das Internet und die Theoriedokumente bereits genügend gute Antworten liefern. D.h. dass wir da mehr Recherche machen als konkret fragen. So ist die Umfrage auch weniger direkt abhängig vom Interview, welches später paralleles Arbeiten an beidem ermöglicht.
    \item Wir haben gemerkt, dass die Zeit doch schneller vergeht, als wir gemeint hatten, vor allem wenn sonst noch Sachen los sind.
\end{itemize}

\subsubsection*{Totaler Zeitaufwand}
Je 3 Lektionen während den vorgesehenen PRO Lektionen + noch etwa 1h Arbeit am Morgen.
\StandortFooter

% 14.05.2025
\StandortHeader{14.05.2025}
\subsubsection*{Programmieren: Funktionierende Features}
\begin{itemize}
    \item Full Authentication System (Login, Signup, Forgot/Reset Password).
    \item Daily Tipps (ohne die eigentlichen Tipps, bis jetzt nur Filler Tipps).
    \item Agenda, mit:
    \begin{itemize}
        \item Color Coding
        \item Priority
        \item Repeatable Events
        \item Natürlich auch Edit und Delete
        \item Importieren von .ical Dateien
    \end{itemize}
\end{itemize}

\subsubsection*{Was noch fehlt}
Was noch fehlt, welches bis zu der Zwischenpräsentation noch gemacht werden muss ist die Serververbindung. Das heisst, die SQL Datei muss im Cloud gespeichert sein.

\subsubsection*{Zeitaufwand}
Der gesamte Zeitaufwand bis jetzt für das Programmieren beträgt etwa 18.25h.

\subsubsection*{Reflexion}
Bis jetzt lief alles perfekt und ich verschwendete relativ wenig Zeit bei Bugs. Hoffentlich geht das so weiter. Ich weiss jetzt auch ziemlich gut wie ich mit Copilot arbeite, und welche Tasks ich diesem geben kann, welches es dann auch gut erledigt.

\subsubsection*{Recherche}
Wir konnten die beiden Interviews erfolgreich durchführen und haben eine Umfrage gestaltet, welche bald abgeschickt werden kann. Der gesamte Prozess hat aber länger gedauert als bisher angenommen, unter anderem wegen der zusätzlich benötigten Buchrecherche und unter anderem einem Feedback-Prozess für die Umfrage. Dennoch haben die Interviews und die Recherche gute, nutzbare Resultate geliefert.

\subsubsection*{Reflexion Recherche}
Obwohl der gesamte Recherche-Prozess länger gebraucht hatte als gedacht, haben wir gute Resultate erzielt, welche wir hoffentlich später erfolgreich umsetzen könnten. In der Zukunft sollten wir aber mehr Zeit für Recherche einplanen.
\StandortFooter

% 05.10.2025
\StandortHeader{05.10.2025}
\subsubsection*{Programmieren: Funktionierende Features}
\begin{itemize}
    \item Full Authentication System (Login, Signup, Forgot/Reset Password).
    \item Daily Tipps (ohne die eigentlichen Tipps, bis jetzt nur Filler Tipps).
    \item Agenda, mit:
    \begin{itemize}
        \item Color Coding
        \item Priority
        \item Repeatable Events
        \item Natürlich auch Edit und Delete
        \item Importieren von .ical Dateien
        \item All Day, Multi Day Events
    \end{itemize}
    \item Lernzeitalgorithmus.
    \item Settings:
    \begin{itemize}
        \item Verschiedene Settings für den LZA.
        \item Auch Allgemeine Settings (Change Password, Delete Account).
    \end{itemize}
    \item Lerntimer.
    \item Notenorganisation (nicht komplett fertig, aber eine gute Basis).
    \item Daily Tipps (mit Filler Tipps für den Moment).
\end{itemize}

\subsubsection*{Was noch fehlt}
\begin{itemize}
    \item UI Design.
    \item Server fix (momentan wird der Server von der Schule geblockt).
    \item Lerntipps.
    \item Verschiedene Tests für den Code.
    \item Notenorganisation ganz fertig programmieren.
    \item Evtl. noch einen Filter für den .ical Import; schauen, ob man einen solchen Filter überhaupt braucht.
\end{itemize}

\subsubsection*{Zeitaufwand}
48.25h (excl. heute, da ich die StaBe am Anfang mache)

\subsubsection*{Reflexion}
Bis jetzt lief alles immer noch sehr gut. Mit Bugs konnte ich gut umgehen, manchmal mithilfe KI, manchmal auch ohne. Was ich noch gut gelernt habe bis jetzt, ist welche KI ich am besten brauchen kann (verschiedene KIs für verschiedene Nutzen). Es bleibt nicht viel Zeit übrig, aber es bleiben auch nicht viele Features übrig. Der schwierigste Feature, der LZA, ist programmiert, und wahrscheinlich das nächste, das viel Zeit brauchen wird, ist das Design. Am Bericht wird auch gut parallel gearbeitet, sodass das nicht am Schluss zu viel Stress verursacht.
\StandortFooter

% 10.10.2025
\StandortHeader{10.10.2025}
\subsubsection*{Was haben wir bis jetzt erreicht?}
\begin{itemize}
    \item Die Interview- und Umfrage-Analyse sind beide fertiggestellt, das heisst sie können nun für die Web-Applikation vollständig gebraucht werden.
    \begin{itemize}
        \item Dies markiert das Ende der Recherche-Phase, alles was noch bleibt ist dies im Schriftlichen Kommentar detailliert zu beschreiben.
    \end{itemize}
\end{itemize}

\subsubsection*{Reflexion}
\begin{itemize}
    \item Die Recherche war, allen in allem, ein voller Erfolg - sie brachte uns viele nützliche Informationen und validierte das Grundkonzept unserer Web-Applikation sehr stark. Somit können wir unseren Entwicklungsprozess auch gut begründen, da dieser tatsächlich
    \item Dieser Abschluss kommt aber sehr spät - wir haben wohl am Anfang der Arbeit wirklich unterschätzt, wie viel Arbeit dies sein würde, bzw. wir hätten zum Teil auch mehr daran arbeiten können. Diese Arbeit hat sich aber total gelohnt.
\end{itemize}

\subsubsection*{Totaler Zeitaufwand}
Siehe Arbeitsjournaleinträge von Simon $\rightarrow$ insgesamt: 67 nachweisbare Stunden (manchmal wurde die Stundenreferenz vergessen)
\StandortFooter

\section*{Fazit}
% Was hat sich am meisten verändert? Welche Erkenntnisse bleiben?
Im Verlauf dieser Maturitätsarbeit haben wir viel gelernt über die Entwicklung einer Web-Applikation. Vorallem haben wir gemerkt, wie wichtig die Datensicherheit wirklich ist. Wir haben gelernt, wie man eine Web-Applikation mit Flask aufbaut, und wie man verschiedene Features implementiert. Ebenso haben wir gelernt, wie man KI-Modelle effektiv in den Entwicklungsprozess integriert, um die Produktivität zu steigern.

\end{document}