\documentclass[a4paper,12pt]{article}
\usepackage[utf8]{inputenc}
\usepackage[T1]{fontenc}
\usepackage{geometry}
\usepackage{ifthen}
\usepackage{makecell}
\usepackage{graphicx}
\geometry{left=2.5cm,right=2.5cm,top=2.5cm,bottom=2.5cm}

% Command for a journal entry (no separator)
\newcommand{\JournalEntry}[6][false]{
    \subsection*{#2}
    \begin{tabular}{|l|l|}
        \hline
        \textbf{Gearbeitete Stunden} & #3 \\ \hline
    \end{tabular}

    \subsubsection*{\textbf{Arbeitsschritte}}
    #4

    \subsubsection*{\textbf{Ergebnisse}}
    #5

    \subsubsection*{\textbf{Begründung und Reflexion}}
    #6

    \vspace{1cm}

    % Optional: Add a horizontal line to separate entries
    \ifthenelse{\equal{#1}{true}}{
        \hrule
    
    \vspace{1cm}
    }{}
}

% Command for a day block with separator
\newcommand{\JournalDay}[1]{
    \section*{Datum: #1}
    \vspace{0.3cm}
}

\begin{document}

\makeatletter
\begin{titlepage}
    \centering
    \vspace*{1cm}
       { \includegraphics[width=6cm]{img/kanti-baden.png}}\\[1cm]

    {\LARGE \textbf{Kanti Koala}}\\
    {\textbf{Die Lern- und Studienhilfsapp für Schüler:innen der Kantonsschule Baden}}\\[1cm]

    {Maturitätsarbeit, Kantonsschule Baden}\\
    {Arbeitsjournal}\\[1cm]
    
    \textbf{Erstbetreuer: }{Michael Schneider}\\
    \textbf{Zweitbetreuerin: }{Julia Smits}\\[1cm]
    
    \textbf{Geschrieben von: }{Aryan Anand (G22b), Simon Haddon (G22b)}\\[1cm]
 
    \date{\large Datum: 11. November 2025}
    {\@date\\}
\end{titlepage}
\makeatother

% 12.03.2025
\JournalDay{12.03.2025}
\JournalEntry[true]{Projektstart: Vertrag und Interviewvorbereitung}{Je $\sim$3 Lektionen (Simon \& Aryan)}{Vertrag ausgefüllt. Projektplan entworfen. INTERVIEW: Angefangen, Themen und Fragen zu entwerfen (z.T. mit Hilfe von ChatGPT Ideen für Fragen geholt).}{Erster Projektplan und Vertragsentwurf erstellt. Erste Ideen für Interviewfragen gesammelt.}{Ideen von ChatGPT sind eigentlich okay, aber wahrscheinlich nicht spezifisch genug.}

% 19.03.2025
\JournalDay{19.03.2025}
\JournalEntry[true]{Vertrag, Interviewfragen und Github-Initialisierung}{Je $\sim$3 Lektionen (Simon \& Aryan)}{Weiter am Vertrag gearbeitet. Interview-Fragen gesammelt und begründet. Eine erste Mail für die PPP-Lehrpersonen entworfen. Lizenz für Webseite gewählt. Github Repository initialisiert. Flask aufgestellt.}{Vertragsentwurf, Interviewfragen, E-Mail-Draft und Lizenzwahl abgeschlossen. Github Repository mit Flask-Grundgerüst aufgesetzt. Basic Account Management Features sind funktionsfähig.}{Wir sind gut vorangekommen, mittlerweile noch gut im Zeitplan. Basic Account Management Features waren einfacher aufzustellen, als erwartet.}

% 21.03.2025
\JournalDay{21.03.2025}
\JournalEntry[true]{Theorie und Lagebesprechung}{Simon (ohne Stundenangabe)}{Theorie-Dokumente zu Interviews angeschaut, welche von Herr Schneider zur Verfügung gestellt wurden. Mail für Frau Suter / Herr Schmocker ausgearbeitet. Lagebesprechung mit Herr Schneider.}{Theoretische Grundlagen für Interviews und konkretes Feedback zum Arbeitsvertrag erhalten.}{Konkretes Feedback erhalten und die theoretische Grundlage für die Interviews erarbeitet.}

% 26.03.2025
\JournalDay{26.03.2025}
\JournalEntry[true]{Code und Recherche}{Je $\sim$3 Lektionen (Simon \& Aryan)}{Aryan: Code, «Forgot Password» feature, Kalender mit Flask, E-Mail erstellt. Simon: Recherche.}{«Forgot Password»-Funktion und Flask-Kalender-Grundgerüst begonnen.}{Gute Fortschritte in der Programmierung und Recherche gemacht.}

% 28.03.2025
\JournalDay{28.03.2025}
\JournalEntry[true]{Programmierung}{Etwa 90 Minuten (Aryan)}{Programmieren.}{Fortschritte in der Programmierung erzielt.}{}

% 02.04.2025
\JournalDay{02.04.2025}
\JournalEntry[true]{Arbeitsvertrag, Interviewfragebogen und Agenda-Funktionalität}{Je $\sim$3 Lektionen (Aryan \& Simon)}{Simon: Arbeitsvertrag fast fertig, Interviewfragebogen angefangen (in \LaTeX\space / Overleaf). Aryan: Agenda Events bearbeiten, erstellen, und löschen; Strategie für eine .ical implementation.}{Arbeitsvertrag fast fertig. Interviewfragebogen in \LaTeX\space begonnen. Funktionen zum Bearbeiten, Erstellen und Löschen von Agenda-Events implementiert, sowie eine Strategie für die .ical-Funktionalität entwickelt.}{Wir haben gemerkt, dass das Umfragedesign nicht im Interview vorkommen muss, da das Internet und die Theoriedokumente bereits genügend gute Antworten liefern. D.h., dass wir da mehr Recherche machen als konkret fragen, was paralleles Arbeiten ermöglicht. Wir haben gemerkt, dass die Zeit schneller vergeht, als wir meinten. Copilot ist nützlich für zeitaufwendige Aufgaben wie Kommentieren und Grundlagen aufbauen.}

% 04.04.2025
\JournalDay{04.04.2025}
\JournalEntry[true]{Bug Fixes und Wiederholbare Events}{Aryan: 2h (Aryan \& Simon)}{Simon: Interviewfragebogen ausgearbeitet, Mail zum Abschicken weiter vorbereitet. Aryan: Bug fix von Account deletion, Logik von wiederholbaren Events.}{Interviewfragebogen weiter ausgearbeitet und Mail vorbereitet. Bugfix für Account Deletion implementiert. Logik für wiederholbare Events begonnen.}{SQL hat "cascade delete", welches sehr nützlich ist um zwei Datenbanken zu verbinden. Der Bug fix lief viel besser als erwartet.}

% 06.04.2025
\JournalDay{06.04.2025}
\JournalEntry[true]{Repeatable Events Implementierung}{2h (Aryan)}{Repeatable Events (ausser löschen) implementiert. Europäisches Datum Format implementiert. Anderer View von Agenda (Monat view) implementiert. Implementieren von repeatable Events Logik mithilfe von ChatGPT.}{Repeatable Events (ausser Löschen), europäisches Datumsformat und Monatsansicht der Agenda implementiert.}{Ich habe gelernt, wie Fullcalendar extra Infos speichert, auch die `datetime`/`dateutil`-Logik für repeatable events (wurde primär von Copilot gemacht).}

% 08.04.2025
\JournalDay{08.04.2025}
\JournalEntry[true]{Beginn des schriftlichen Kommentars}{1.5h (Aryan)}{Fing an, den schriftlichen Kommentar zu schreiben.}{Der schriftliche Kommentar wurde begonnen.}{Lief viel besser als erwartet (kein chatGPT!). Ich weiss jedoch nicht, wo ich Quellen angeben muss.}

% 10.04.2025
\JournalDay{10.04.2025}
\JournalEntry[true]{Bericht und .ical Feature}{2.5h (Aryan)}{Alles im Bericht, das bisher programmiert wurde, dokumentiert. .ical import feature implementiert.}{Der Bericht wurde um die bisherige Programmierung ergänzt. Das .ical Import Feature ist funktionsfähig.}{Lief alles sehr gut, das Schreiben vom Bericht ist schneller als erwartet.}

% 13.04.2025
\JournalDay{13.04.2025}
\JournalEntry[true]{Umfrage, Server und Prioritäten}{Aryan: 0.5h (Aryan \& Simon)}{Simon: Arbeit an der Umfrage (via Microsoft Forms). Aryan: Überlegung vom Server, Priorität von Events implementiert.}{Arbeit an der Umfrage begonnen. Gedanken zur Serverimplementierung gemacht. Priorität von Events implementiert.}{Der Server wird vielleicht schwer zu implementieren und kann Kosten verursachen.}

% 14.04.2025
\JournalDay{14.04.2025}
\JournalEntry[true]{Umfrage und Farbentheorie}{~2 Stunden (Simon)}{Weitere Arbeit an der Umfrage – Layout, Fragen, Technisches. Farbentheorie für das Umfragedesign recherchiert.}{Die Umfrage ist sehr weit fortgeschritten. Farbentheorie-Recherche für das Design durchgeführt.}{}

% 15.04.2025
\JournalDay{15.04.2025}
\JournalEntry[true]{Logo-Skizzen}{Ohne Stundenangabe (Simon)}{Erste Arbeit am Logo, erste Skizzen und Überlegungen gemacht.}{Erste Skizzen und Überlegungen zum Logo-Design erstellt.}{}

% 21.04.2025
\JournalDay{21.04.2025 - 24.04.2025}
\JournalEntry[true]{Recherche Lerntheorie}{$\sim$5 Stunden (Simon)}{Bücher über Lerntheorie vor den Interviews ausgeliehen (Auf Empfehlung von Herr Schmocker) und gelesen. Wichtigste Punkte daraus notiert.}{Wichtige Erkenntnisse aus der Buchrecherche zur Lerntheorie gewonnen.}{Das Ausleihen ging sehr einfach, hätten wir vielleicht schon vorher machen sollen.}

% 24.04.2025
\JournalDay{24.04.2025}
\JournalEntry[true]{Erstes Interview}{Etwa 1 $\frac{1}{4}$ Stunden (Simon \& Aryan)}{Erstes Interview durchgeführt, mit Frau Suter, sehr positiv verlaufen.}{Das erste Interview mit Frau Suter wurde erfolgreich abgeschlossen.}{}

% 30.04.2025
\JournalDay{30.04.2025}
\JournalEntry[true]{Recherche Daily Tipps}{0.5h (Aryan)}{Ein bisschen für Daily Tipps recherchiert.}{Recherche für Daily Tipps begonnen.}{Im MA-Archiv gibt es gute Arbeiten, aber es ist schwierig daraus Tipps zu finden.}

% 07.05.2025
\JournalDay{07.05.2025}
\JournalEntry[true]{Programmierung und Recherche}{Je $\sim$3 Lektionen (Simon \& Aryan)}{Simon: weitere Bücherrecherche, Erkenntnisse notiert. Aryan: Forgot password implementiert (ohne Testing), repeatable events löschen.}{Bücherrecherche fortgesetzt. «Forgot password»-Funktion implementiert. Logik zum Löschen von wiederholbaren Events implementiert.}{Die Logik für «Forgot Password» war viel komplexer als gedacht, jedoch die für «repeatable events» viel einfacher.}

% 08.05.2025
\JournalDay{08.05.2025}
\JournalEntry[true]{Zweites Interview}{~1 Stunde (Simon \& Aryan)}{Zweites Interview – jetzt mit Herr Schmocker – durchgeführt.}{Das zweite Interview mit Herr Schmocker wurde erfolgreich durchgeführt.}{Die Interviews auf Mundart durchzuführen war ein Fehler. ABER wir haben sowohl von Frau Suter wie auch Herr Schmocker sehr gutes Feedback erhalten, u.a. auch auf unsere Professionalität $\rightarrow$ Das ist gut.}

% 09.05.2025
\JournalDay{09.05.2025}
\JournalEntry[true]{Umfrage, Testing und Daily Tipps}{3h (Simon \& Aryan)}{Simon: Umfrage-Feedback von Herr Schmocker ergänzt, Umfrage überarbeitet. Bis dahin: Auch noch mehr Bücherrecherche durchgeführt. Aryan: Testing und fixing von Forgot password, Daily tipps template.}{Umfrage überarbeitet, Bücherrecherche abgeschlossen. «Forgot password»-Funktion getestet und gefixt. Daily Tipps Template erstellt.}{Lief alles perfekt.}

% 14.05.2025
\JournalDay{14.05.2025}
\JournalEntry[true]{Umfrage Testläufe und Serververbindung}{90 min (Aryan \& Simon)}{Simon: Umfrage nochmals überarbeitet, Klassenkameraden um erste Testläufe gefragt, um dies «praktisch» zu testen. Aryan: Probierte Serververbindung, aber nur Security stuff geschafft.}{Umfrage überarbeitet und erste Testläufe mit Klassenkameraden gestartet. Erste Schritte zur Serververbindung (Security) unternommen.}{Obwohl der gesamte Recherche-Prozess länger gebraucht hatte als gedacht, haben wir gute Resultate erzielt. In der Zukunft sollten wir aber mehr Zeit für Recherche einplanen. Serververbindung ist viel schwieriger als erwartet, aber trotzdem wichtiges Security Zeugs gemacht.}

% 16.05.2025
\JournalDay{16.05.2025}
\JournalEntry[true]{Server-Implementierung}{2h (Aryan)}{Immernoch Server am versuchen.}{Fortschritte bei der Serverimplementierung.}{Noch nicht funktioniert, aber bin sehr nah an einer Lösung.}

% 18.05.2025
\JournalDay{18.05.2025}
\JournalEntry[true]{Serververbindung und Quellenformatierung}{2.5h (Aryan)}{Endlich Serververbindung implementiert. Quellen für das schriftliche Kommentar formatiert (\LaTeX).}{Funktionierende Serververbindung implementiert. Quellen für den Bericht in \LaTeX\space formatiert.}{Gut, dass es funktionierte.}

% 20.05.2025
\JournalDay{20.05.2025}
\JournalEntry[true]{Umfrage-Feedback-Verarbeitung}{~30 min. (Simon)}{Weiteres Feedback von Frau Suter ergänzt/umgestaltet.}{Umfrage nach Feedback von Klassenkameraden und Frau Suter umgestaltet.}{Leider haben wir von unseren Klassenkameraden nur 3 Antworten bekommen, aber trotzdem wertvolles Feedback. Auswertung des Feedbacks unserer Klassenkameraden hat einiges gezeigt, und wir sind hauptsächlich mit den Resultaten zufrieden, wenn auch sie nicht immer die tiefgründigsten Antworten gegeben haben, weswegen wir nun versucht haben, dies ein wenig mit Umformulierungen entgegenzuwirken.}

% 21.05.2025
\JournalDay{21.05.2025}
\JournalEntry[true]{Umfrage-Abschluss}{~1h (Simon)}{Finale Änderungen an der Umfrage gemacht, an Frau Hoffmann geschickt.}{Die finalen Änderungen an der Umfrage wurden vorgenommen und sie wurde zur Freigabe geschickt.}{Die Arbeit an der Umfrage \& Recherche hat viel länger gedauert, als erhofft / zuerst angenommen.}

% 22.05.2025
\JournalDay{22.05.2025}
\JournalEntry[true]{Logo-Design}{Ohne Stundenangabe (Simon)}{Weitere Arbeit am Logo, von Skizzen weitergearbeitet und angefangen, erste Versionen auszuarbeiten.}{Erste Versionen des Logos basierend auf Skizzen ausgearbeitet.}{Auseinandersetzung mit «Was macht ein gutes Logo aus», wir haben, glaube ich, noch keine endgültige Version getroffen.}

% 27.05.2025
\JournalDay{27.05.2025}
\JournalEntry[true]{Umfrage-Versand}{Ohne Stundenangabe (Simon)}{Nach letztem Feedback von Frau Hoffmann wurde die Umfrage (ENDLICH) losgeschickt.}{Die Umfrage wurde nach Freigabe versandt.}{}

% 28.05.2025
\JournalDay{28.05.2025 - 04.06.2025}
\JournalEntry[true]{Interviews Transkribieren}{Ca. 4 Stunden (Simon)}{Interviews werden transkribiert und direkt auf Hochdeutsch übersetzt. Ebenso werden, so gut wie möglich, sprachliche Aussetzer («ähm», bspw.) entfernt, sollte aber sonst so getreu wie möglich bleiben.}{Transkription und Übersetzung der Interviews ins Hochdeutsche begonnen.}{SEHR aufwendig, in der Zukunft sollten die Interviews auf Hochdeutsch und nicht Mundart durchgeführt werden, da dies das Transkribieren erschwert, da digitale Tools nicht gut funktionieren.}

% 04.06.2025
\JournalDay{04.06.2025 - 10.06.2025}
\JournalEntry[true]{Interviews Transkribieren}{Ca. 9 Stunden (Simon)}{Beide Interviews fertig transkribiert – insgesamt ca. 13 Stunden für die je $\sim$1-stündigen Interviews.}{Beide Interviews fertig transkribiert.}{Die Transkription wird aber definitiv sehr nützlich sein, um sowohl direkte Zitierungen wie um auch anderes (Bspw. für unsere «Daily Tipps») daraus zu nehmen.}

% 11.06.2025
\JournalDay{11.06.2025}
\JournalEntry[true]{Arbeitsjournal-Übertragung und Umfrage-Auswertung}{Ohne Stundenangabe (Simon \& Aryan)}{Arbeitsjournal erstmals auf Word übertragen \& ausgefüllt, finale Änderungen an Dokumenten. Umfrageresultate angefangen, auszuwerten in Excel.}{Arbeitsjournal-Übertragung und finale Dokumentenänderungen abgeschlossen. Auswertung der 83 Umfrageantworten in Excel begonnen.}{Wir haben 83 Antworten erhalten auf die Umfrage, was viel positiver ist, als wir erwartet hätten. Aber: Wenn man es auf die einzelnen Stufen aufteilt, ist es immer noch nicht so viel.}

% 16.06.2025 - 17.06.2025
\JournalDay{16.06.2025 - 17.06.2025}
\JournalEntry[true]{Vorbereitung und Design der PowerPoint-Präsentation}{Ca. 3 Stunden (Simon)}{Design der PowerPoint-Präsentation ausgearbeitet (Farbschema, Layout und Design-Regeln). Daten von Umfrage weiter ausgewertet und visuell dargestellt (Pie-Charts). Farbschema: Helle, leicht nicht saturierte Farbtöne gewählt (Weisser Hintergrund, (dunkel)grauer Text, türkise Flächen, Blau-Schema für Graphen). Layout-Regeln: Möglichst einheitlich zwischen slides. Titelseite: «KantiKoala» gross und fett, türkise/blaue Bänder, Name und Klasse. Reguläre Seiten: Titel links vom Strich, Inhalt möglichst kurz gefasst, Titel hochgesetzt, Erkenntnisse unter Titel. Datenauswertung: Graphen so weit wie möglich, um prozentuelle Auslegung darzustellen (Bspw. ob gewisse Stressoren zelteten). Bar-Graphen für einfache Nummern-Vergleiche.}{Ein vollständiges Designkonzept (Farb- und Layout-Schema) für die Präsentation ist erstellt. Die Umfragedaten wurden weiter ausgewertet und für die visuelle Darstellung (Pie-Charts, Bar-Graphen) vorbereitet. Die Struktur der Titelseite und der regulären Seiten ist definiert.}{Das Design ist auf eine einfache, helle und visuell ansprechende Darstellung ausgerichtet.}

% 13.08.2025
\JournalDay{13.08.2025}
\JournalEntry[true]{Interview-Analyse und Algorithmus-Grundlagen}{\makecell{Ca. 3 Stunden (Simon) \\ zusätzliche Zeit für Aryan \& Simon für Algorithmus}}{Simon: Zentrale Aussagen der Interviews mit Herr Schmocker \& Frau Suter auffassen. Erkenntnisse werden nach Fragen geordnet, und auch generelle Tipps/Empfehlungen zum Lernen und für die Applikation gesammelt. Dokument erstellt \& Layout. Erste Erkenntnisse gesammelt in Lerntechniken \& Lernmethoden. Aryan \& Simon: Algorithmus-Gedanken für die Planung ('wichtig' und 'advanced').}{Simon: Dokument zur Interview-Analyse erstellt und erste Erkenntnisse aus den Interviews in Lerntechniken und -methoden gesammelt und geordnet. Aryan \& Simon: Grundlegende Anforderungen für den Planungsalgorithmus definiert (Lernstoff gut ausbreiten, gute Zeiten vorschlagen, Zeitplan nicht überfüllen, Wochenenden depriorisieren, Prioritäten setzen).}{Der Fokus liegt auf der Umsetzung der theoretischen Erkenntnisse aus den Interviews in konkrete Empfehlungen für die Web-Applikation. Die Algorithmus-Gedanken stellen die Basis für die Zeitplanungsfunktionalität dar.}

% 14.08.2025
\JournalDay{14.08.2025}
\JournalEntry[true]{Interview-Analyse Herr Schmocker}{Ca. 2h (Simon)}{Erkenntnisse von Herr Schmocker auffassen, wenn möglich (nahezu) fertig.}{Einige Fragen vollständig bearbeitet. Viele Tipps zur Lernapp ergänzt.}{Das Ziel, die Erkenntnisse von Herrn Schmocker nahezu fertig aufzufassen, wurde gut erreicht, indem viele Tipps ergänzt und Fragen bearbeitet wurden.}

% 16.08.2025
\JournalDay{16.08.2025}
\JournalEntry[true]{Interview-Analyse Herr Schmocker fortgesetzt}{Ca. 1h (Simon)}{Weiterhin Erkenntnisse von Herr Schmocker auffassen.}{Weiter an Herrn Schmockers Kommentaren gearbeitet.}{Die Arbeit an den Erkenntnissen aus dem Interview mit Herrn Schmocker wurde fortgesetzt.}

% 18.08.2025
\JournalDay{18.08.2025}
\JournalEntry[true]{Interview-Analyse abgeschlossen}{Ca. 1.5h (Simon)}{Herr Schmockers Erkenntnisse fertig bearbeiten. Mit den Erkenntnissen von Frau Suter anfangen.}{Die Analyse der Erkenntnisse aus dem Interview mit Herrn Schmocker ist fertiggestellt. Die Aufarbeitung der Erkenntnisse von Frau Suter wurde begonnen.}{Das Hauptziel, die Erkenntnisse von Herrn Schmocker fertigzustellen, wurde erreicht. Die Arbeit an Frau Suters Interview-Daten konnte planmässig begonnen werden.}

% 20.08.2025
\JournalDay{20.08.2025}
\JournalEntry[true]{Implementierung von Ganztags-Events und Notenorganisation}{3h (Aryan)}{Implementierung der "All day/multi day" Event-Funktionalität im Kalender. Einen Anfang für die Notenorganisation machen.}{Die Funktionalität für ganztägige und mehrtägige Events (`All day/multi day`) wurde implementiert. Ein Grundstein für die Notenorganisation wurde gelegt.}{Ich musste ein bisschen mehr als erwartet mit KI arbeiten, weil ich mich nicht so gut in JavaScript auskenne. Aber es lief trotzdem gut.}

% 28.08.2025
\JournalDay{28.08.2025}
\JournalEntry[true]{Weiterarbeit an der Notenorganisation}{1.5h (Aryan)}{Notenorganisation weiterbearbeiten und überlegen, welche weiteren Funktionen dort implementiert werden sollen.}{Die Notenorganisation ist funktional im gewünschten Umfang implementiert.}{Die Hauptfunktionalität der Notenorganisation ist erreicht, es muss jedoch noch entschieden werden, welche optionalen Features hinzugefügt werden. Das vollständige Testing der Funktion steht noch aus.}

% 02.09.2025
\JournalDay{02.09.2025}
\JournalEntry[true]{Interview-Analyse Frau Suter fortgesetzt}{Ca. 2.5h (Simon)}{Weiter an Frau Suters Erkenntnissen arbeiten, um die zentralen Aussagen aus dem Interview zu erfassen und in Empfehlungen für die Applikation umzuwandeln.}{Mehr Erkenntnisse aus dem Interview mit Frau Suter gesammelt und für die Applikation aufbereitet.}{Die Aufarbeitung der Interview-Erkenntnisse schreitet gut voran und liefert wertvolle Informationen für die App-Funktionalität.}

% 04.09.2025
\JournalDay{04.09.2025}
\JournalEntry[true]{Interview-Analyse Frau Suter abgeschlossen}{Ca. 1.5h (Simon)}{Weiter an Frau Suters Erkenntnissen arbeiten. Sektionen "Lerntechniken", "Lernmethoden" und Teile von "Pausenmanagement" fertigstellen. Nebendran: Interviewdokument schöner formatieren.}{Die Abschnitte zu Lerntechniken und Lernmethoden sowie die ersten beiden Unterabschnitte von Pausenmanagement sind fertiggestellt. Das Interviewdokument wurde zusätzlich optisch aufbereitet.}{Die Analyse der Erkenntnisse von Frau Suter wurde weitgehend abgeschlossen, und die Formatierung des Interviewdokuments verbessert die Lesbarkeit.}

% 09.09.2025
\JournalDay{09.09.2025}
\JournalEntry[true]{Implementierung Pomodoro-Timer und Algorithmus-Grundlagen}{1.5h (Aryan)}{Implementierung des Lerntimers und Weiterentwicklung der Algorithmus-Gedanken. Für den Pomodoro-Timer wurde ein Grundgerüst von GPT geholt. Weiterhin wurden Algorithmus-Überlegungen gemacht, basierend auf folgenden Input-Daten: Agenda, separater Prüfungsplan (aus Agenda extrahieren), maximale Lerntage (7, extrem: 14). Die Priorität soll der Prüfungs-Lernzeit entsprechen (z.B. hoch heisst 10 Stunden). Pro Tag sollen alle Aktivitäten addiert werden, um die verbleibende Zeit für Lernaktivitäten zu bestimmen (24-10 = Schlaf plus Essen). Dann soll durch alle Tage vor der Prüfung gegangen werden. Wenn genug Zeit, einfach durchschnittliche Lernzeit pro Tag subtrahieren. Sonst die ganze Zeit vom Vortag subtrahieren. Wenn der Schluss innerhalb von 1h der Estimierung ist, ist es ok, sonst muss man 2 Wochen vorher anfangen. Der Algorithmus soll jedes Mal, wenn etwas geändert wird, von heute nach vorne schauen.}{Der Pomodoro-Timer ist implementiert (nach Korrektur des GPT-Grundgerüsts). Der Algorithmus für die Prüfungsplanung ist theoretisch ausgereift und bereit zur Implementierung.}{GPT gab zwar ein gutes Grundgerüst für den Pomodoro-Timer, welches aber noch geflickt werden musste. Der Algorithmus sieht theoretisch gut aus und kann nun implementiert werden.}

% 10.09.2025
\JournalDay{10.09.2025}
\JournalEntry[true]{Algorithmus-Design und Prioritätensetzung}{3h (Aryan)}{Berichtserweiterungen, Algorithmus-Implementierung und Prioritätensetzung. Der Algorithmus wurde weiter überarbeitet: Die Prüfungspriorisierung soll 4 Fälle umfassen (Sehr wichtig \& viel: 5-7 Tage, 2h pro Tag; Wichtig \& guter Stoff: 5-7 Tage, 1h pro Tag; Nicht so wichtig: 3-5 Tage, 0.5-1h pro Tag; Unwichtig \& fast kein Stoff: 1-2 Tage, 0.5-1h pro Tag). Grundsätzlich sollen 7 Tage nicht überschritten werden (maximal 2 Wochen im Extremfall). Pro Tag ist ein Maximum von 3h Lernzeit akkumuliert. Funktionsweise: Rückwärts von der Prüfung zurückgehen und Zeit finden. Tags sollen implementiert werden (Schule, Freizeit, Lernzeit, Termin, ...). Verbleibende Zeit wird zu einer Variablen addiert, wenn man nicht viel Zeit hat. Ein anderer Schüler wurde nach Input gefragt.}{Ein Draft des Algorithmus liegt vor. Weitere Algorithmus-Gedanken und eine detaillierte Priorisierungsstruktur für Prüfungen wurden entwickelt. Input von einem anderen Schüler wurde eingeholt. Die Implementierung von Bericht und Algorithmus selbst, sowie die tatsächliche Prioritätensetzung (im Code) wurden noch nicht erreicht.}{Beim Implementieren fiel auf, dass die benötigte Lernzeit für verschiedene Prüfungslevel unklar ist. Eine ganze Umfrage nur dafür scheint zu viel. Der Algorithmus muss noch finalisiert werden.}

% 17.09.2025
\JournalDay{17.09.2025}
\JournalEntry[true]{Interview-Auswertung: Pausen- und Zeitmanagement}{Ca. 1h (Simon)}{Weitere Arbeit an der Auswertung des Interviews von Frau Suter mit dem Ziel, die Abschnitte zu Pausenmanagement und Zeitmanagement fertigzustellen.}{Die Abschnitte Pausenmanagement und Zeitmanagement sind grösstenteils fertiggestellt, womit die Auswertung des Interviews fast abgeschlossen ist.}{Die Analyse ist nun nahezu fertig, was die Grundlage für die nächsten Schritte in der Umsetzung bildet.}

% 24.09.2025
\JournalDay{24.09.2025}
\JournalEntry[true]{Fertigstellung Interview-Analyse und Strukturierung der App-Empfehlungen}{Ca. 1.5h (Simon)}{Fertigstellung der Analyse von Frau Suters Aussagen. Anfang an allgemeinen, konkreten Empfehlungen für die Web-Applikation. Die Empfehlungen werden nach Kategorien geordnet (A: Daily Tipps, B: Agenda/Algorithmus, C: Pomodoro-Timer, D: Verschiedenes) und erhalten zur einfachen Referenz einen Buchstaben und eine Zahl.}{Die Interview-Analyse ist abgeschlossen und die Struktur für die konkreten App-Empfehlungen (geordnet und referenzierbar) ist erstellt.}{Die Interviews gaben viele nützliche Informationen, welche besonders für die «Daily Tipps» relevant sind. Wir haben auch persönlich viel davon gelernt.}

% 25.09.2025
\JournalDay{25.09.2025}
\JournalEntry[true]{Abschluss der Interview-Auswertung und Berichtergänzung}{Ca. 3h (Simon: 2.5h, Aryan: 0.5h)}{Simon: Fertigstellung der Auswertung, indem die konkreten Empfehlungen fertiggestellt wurden. Aryan: Ergänzung des Berichts.}{Das Dokument mit den konkreten Empfehlungen ist abgeschlossen und fertig formatiert. Die Interview-Auswertung ist somit komplett und für zukünftige Referenz fertiggestellt.}{Die Interviews gaben viel Material für die Daily Tipps, aber auch Nützliches für den Algorithmus und den Pomodoro-Timer. Das Dokument ist abgeschlossen und formatiert, somit kann die Arbeit an der Umfrage-Analyse richtig anfangen. Man kann auch viel daraus persönlich lernen.}

% 26.09.2025
\JournalDay{26.09.2025}
\JournalEntry[true]{Beginn der Umfrage-Auswertung}{Ca. 1h (Simon)}{Dokument zur Auswertung der Umfrage erstellen und einrichten. Die ersten Schlüsse und Daten aus der Zeit vor den Sommerferien wurden erneut gesichtet.}{Das Auswertungsdokument wurde aufgestellt und die Excel-Datei mit den Rohdaten aus Microsoft Forms wurde bearbeitet.}{Die Arbeit an der Umfrage-Auswertung wurde begonnen und die notwendigen Daten und Strukturen dafür vorbereitet.}

% 01.10.2025
\JournalDay{01.10.2025}
\JournalEntry[true]{Vorarbeit für die Algorithmus-Implementierung}{5h (Aryan)}{Kommentare im Code hinzugefügt und den Code optisch aufbereitet (Verschönerung, auch mit black). Die Logik für die Priorität von Events wurde überarbeitet und implementiert: Rot (Sehr wichtig), Orange (Wichtig, braucht aber noch etwas), Grün (Relativ unwichtig), Farbe frei wählbar (Keine Prüfung). Die Farben sind vom Nutzer wählbar. Einstellungen (`Settings`) und ein detaillierter Pseudo-Code für den Algorithmus wurden erstellt.}{Der Code wurde verschönert und kommentiert. Die Logik für die Event-Priorität ist implementiert. Detaillierte Settings und der Pseudo-Code für den Lern-Algorithmus sind fertiggestellt.}{Die Code-Verschönerung und die Prioritäts-Logik liefen gut. Der eigentliche Algorithmus und die Notenorganisation wurden noch nicht implementiert, aber die Vorarbeit dazu ist abgeschlossen. Gelernt wurde nicht viel Neues, es war primär eine Anwendung von Flask wie gewohnt.}

% 02.10.2025
\JournalDay{02.10.2025}
\JournalEntry[true]{Algorithmus-Implementierung (Codierung)}{4h (Aryan)}{Weiterarbeit am Algorithmus mit dem Ziel, den Pseudo-Code in tatsächlichen Python-Code zu übersetzen.}{Der Algorithmus wurde codiert, ist aber noch nicht implementiert (d.h. er existiert im Code, ist aber noch ohne Nutzung). Der Algorithmus konnte erfolgreich von Pseudo-Code nach Python umgewandelt werden.}{Die Umwandlung des Algorithmus von Pseudo-Code nach Python verlief gut. Für das nächste Mal ist das Testen und die eigentliche, funktionierende Implementierung des Algorithmus geplant.}

% 03.10.2025
\JournalDay{03.10.2025}
\JournalEntry[true]{Implementierung und Debugging des Lern-Algorithmus}{4h (Aryan)}{Der Algorithmus wurde implementiert. Es wurden Helfer-Funktionen für den Algorithmus erstellt. Ausserdem kann der User jetzt frei Prioritäten hinzufügen und löschen.}{Der Lern-Algorithmus ist implementiert. Der User kann Prioritäten für Lern-Events festlegen und diese nach Bedarf anpassen. Settings sind jetzt in einer separaten Datenbank für Übersichtlichkeit.}{Die Implementierung des Algorithmus verlief gut, jedoch traten einige Bugs auf, die noch behoben werden müssen.}

% 04.10.2025
\JournalDay{04.10.2025}
\JournalEntry[true]{Test und Feinabstimmung des Lern-Algorithmus}{3h (Aryan)}{Der Algorithmus wurde weiter getestet und angepasst. Die Logik, wie der Algorithmus die Zeit findet, musste korrigiert werden.}{Der Algorithmus ist nun funktionstüchtig und findet freie Zeitblöcke korrekt, um Lern-Events einzutragen.}{Der Algorithmus funktioniert jetzt, nachdem es mit den Test-Cases einige Probleme gab. Die Logik, wie freie Zeit gefunden wird, musste angepasst werden.}

% 05.10.2025
\JournalDay{05.10.2025}
\JournalEntry[true]{Lokales \LaTeX-Setup und Berichtsstart}{4h}{Installation von \textbf{TeXLive} auf dem lokalen Rechner gestartet. Weitere Ergänzungen vom Bericht. Versuch der Code-Verschönerung (Kommentare, Lesbarkeit).}{Installation von TeXLive initiiert. Wenige Ergänzungen am Bericht geschafft, das Quellenverzeichnis ist nicht so einfach. Code-Verschönerung nicht erreicht.}{Die Installation von TeXLive (\LaTeX) war sehr zeitaufwendig ($\sim$1.5h) und konnte wegen der langen Zeit noch nicht mit Visual Studio Code aufgestellt werden. Das Quellenverzeichnis erscheint schwierig.}

% 06.10.2025
\JournalDay{06.10.2025}
\JournalEntry[true]{Implementierung der Notenorganisation und Umfrage-Auswertung}{Ca. 4.5h (Aryan: 3h, Simon: 1.5h)}{Aryan: Abschluss der Notenorganisation. Der Code wurde verschönert mithilfe KI, und am Bericht wurde weitergeschrieben. Simon: Die Umfrage-Auswertung wurde fortgesetzt, mit Fokus auf der Ausarbeitung des Abschnitts Lernmethoden.}{Aryan: Die Notenorganisation ist nun vollständig implementiert. Simon: Der Abschnitt zur Auswertung der Lernmethoden aus der Umfrage wurde fertiggestellt.}{Aryan: Die Programmierung lief besser als erwartet. Simon: Die Umfrage-Resultate sind sehr nützlich, müssen aber noch mit den Interview-Ergebnissen abgeglichen werden, da es leichte Diskrepanzen gab.}

% 07.10.2025
\JournalDay{07.10.2025}
\JournalEntry[true]{Zusammenstellung KI-Nachweis und Umfrage-Auswertung}{Ca. 3.5h (Aryan: 2h, Simon: 1.5h)}{Aryan: Den KI-Nachweis für die Maturitätsarbeit zusammengestellt. Simon: Weiterarbeit an der Umfrage-Auswertung und Ausarbeitung der Abschnitte Pausen und Zeitmanagement.}{Aryan: Der KI-Nachweis wurde erstellt und die Verwendung von KI-Tools im Projekt dokumentiert. Simon: Die Abschnitte zur Auswertung von Pausen und Zeitmanagement aus der Umfrage wurden fertiggestellt.}{Die Arbeit an der Umfrage-Auswertung ist nun weit fortgeschritten. Mit dem Erstellen des KI-Nachweises ist ein wichtiger formaler Bestandteil der Dokumentation erledigt.}

% 08.10.2025
\JournalDay{08.10.2025}
\JournalEntry[true]{Abschluss der Umfrage-Analyse und Übertragung des Arbeitsjournals}{Ca. 3.5h (Simon: 1.5h, Aryan: 2h)}{Simon: Die Umfrage-Analyse wurde abgeschlossen, indem einige Erkenntnisse \& ein paar konkrete Empfehlungen hinzugefügt wurden. Ebenso wurden die (seriösen) Tipps der Schüler:innen aus der Umfrage übertragen. Aryan: Alle bisherigen Arbeitsjournal-Einträge auf ein \LaTeX-Dokument übertragen}{Simon: Die Umfrage-Analyse ist nun fertiggestellt und kann so als Referenz gebraucht werden. Aryan: Arbeitsjournal wurde aufgearbeitet und in \LaTeX\space jetzt eingefügt, mit einem einfach wiederverwendbaren Template.}{Simon: Die Analyse der Umfrage begründet \& validiert unsere Web-Applikation sehr gut. Sie zeigt auf u.a. dass viele Schüler nur sehr spät mit dem Lernen anfangen, oft Prüfungsstress haben und auch an so einer Lernapp interessiert wären. Leider kam die fertiggestellte Analyse erst sehr spät, aber die Resultate waren auch ohne die formelle Analyse schon relativ sichtbar. Aryan: In \LaTeX\space ist die Formattierung sehr einfach, und in Zukunft wird das Erstellen von neuen Einträgen auch sehr einfach sein.}

\JournalDay{09.10.2025}


\end{document}